\documentclass[11pt, english, fleqn, DIV=15, headinclude, BCOR=1cm]{scrartcl}

\usepackage[bibatend]{../header}

\usepackage{lastpage}
\usepackage{multicol}
\usepackage{simplewick}
\usepackage{slashed}

\newcommand\timeorder{\mathscr T}
\newcommand\normorder{\mathscr N}
\newcommand\eye{\mat 1_4}
\newcommand\myslash[1]{\underline{\slashed{\vec{#1}}}}

\hypersetup{
    pdftitle=
}

\graphicspath{{build/}}

\newcounter{totalpoints}
\newcommand\punkte[1]{#1\addtocounter{totalpoints}{#1}}

\newcounter{problemset}
\setcounter{problemset}{1}

\subject{physics7501 -- Advanced Quantum Field Theory}
\ihead{physics7501 -- Problem Set \arabic{problemset}}

\title{Problem Set \arabic{problemset}}

\newcommand\thegroup{Tutor: Thorsten Schimmanek}

\publishers{\thegroup}
\ofoot{\thegroup}

\author{
    Martin Ueding \\ \small{\href{mailto:mu@martin-ueding.de}{mu@martin-ueding.de}}
}
\ifoot{Martin Ueding}

\ohead{\rightmark}

\begin{document}

\maketitle

\vspace{3ex}

\begin{center}
    \begin{tabular}{rrr}
        Problem & Achieved points & Possible points \\
        \midrule
        \nameref{homework:1} & & \punkte{15} \\
        \midrule
        Total & & \arabic{totalpoints}
    \end{tabular}
\end{center}

\vspace{3ex}

\begin{center}
    \begin{small}
        This document consists of \pageref{LastPage} pages.
    \end{small}
\end{center}

\section{Higgs correction to $g - 2$}
\label{homework:1}

\subsection{Gordon identity}

The $\mat\sigma^{\mu\nu}$ are defined by the antisymmetric part of the Dirac
matrices:
\[
    - \iup \mat\sigma^{\mu\nu} := \mat\gamma^{[\mu} \mat\gamma^{\nu]} \,.
\]
This uses the idempotent variant of the antisymmetrization brackets. They can
be expanded using the anticommunitation relation.
\begin{align*}
    \iup \mat\sigma^{\mu\nu}
    &= - \mat\gamma^{[\mu} \mat\gamma^{\nu]} \\
    &= - \frac12 \sbr{
        \mat\gamma^{\mu} \mat\gamma^{\nu}
        - \mat\gamma^{\nu} \mat\gamma^{\mu}
    }
    \intertext{%
        We use the anticommunitation relation and yield
    }
    &= \eta^{\mu\nu} \eye - \mat\gamma^{\mu} \mat\gamma^{\nu} \,.
\end{align*}

We will also need the anticommutation of a slashed vector with a Dirac matrix.
This works as follows:
\[
    - \mat\gamma^\mu \myslash p'
    = - \mat\gamma^\mu \mat\gamma^\nu p'_\nu
    = \sbr{\mat\gamma^\nu \mat\gamma^\mu - 2 \eta^{\mu\nu}} p'_\nu
    = \myslash p' \mat\gamma^\mu - 2 p'^\mu \,.
\]

Then the Gordon identity can be shown. We start with and right side and insert
the previously derived expression of $\iup \mat\sigma^{\mu\nu}$:
\begin{align*}
    \bar u(\four p') \sbr{\frac{p'^\mu + p^\mu}{2m} + \frac{\iup
    \mat\sigma^{\mu\nu} q_v}{2m}} u(\four p)
    &= \bar u(\four p') \sbr{\frac{p'^\mu + p^\mu}{2m} + \frac{
    \sbr{\eta^{\mu\nu} - \mat\gamma^{\mu} \mat\gamma^{\nu}} q_v}{2m}} u(\four
    p) \,.
    \intertext{%
        We have dropped the $\eye$ here. Then we just factor out the innermost
        bracket and obtain
    }
    &= \bar u(\four p') \sbr{\frac{p'^\mu + p^\mu}{2m} + \frac{ q^\mu -
    \mat\gamma^{\mu} \slashed{\vec q}}{2m}} u(\four p) \,.
    \intertext{%
        The definition of the passed momentum~$\four q$ is $\four p' - \four
        p$. Expanding those will get us to
    }
    &= \bar u(\four p') \frac{1}{2m} \sbr{p'^\mu + p^\mu + p'^\mu - p^\mu -
    \mat\gamma^\mu \myslash p' + \mat\gamma^\mu \myslash p}
    u(\four p) \,.
    \intertext{%
        Some terms can be cancelled easily.
    }
    &= \bar u(\four p') \frac{1}{2m} \sbr{2 p'^\mu -
    \mat\gamma^\mu \myslash p' + \mat\gamma^\mu \myslash p}
    u(\four p)
    \intertext{%
        Then we apply the anticommunitation of the slashed outgoing momentum.
    }
    &= \bar u(\four p') \frac{1}{2m} \sbr{2 p'^\mu + \myslash p' \mat\gamma^\mu
    - 2 p'^\mu + \mat\gamma^\mu \myslash p} u(\four p) \\
    &= \bar u(\four p') \frac{1}{2m} \sbr{\myslash p' \mat\gamma^\mu +
    \mat\gamma^\mu \myslash p} u(\four p)
    \intertext{%
        The Dirac equations of motion are $[\myslash p - m] u(\four p) = 0$ and
        $\bar u(\four p') [\myslash p' - m] = 0$. We can apply those to get rid
        of the momenta.
    }
    &= \bar u(\four p') \frac{1}{2} \sbr{\mat\gamma^\mu + \mat\gamma^\mu} u(\four p)
    \intertext{%
        Then we just simplify and obtain the desired result.
    }
    &= \bar u(\four p') \mat\gamma^\mu u(\four p)
\end{align*}

\subsection{Higgs vertex correction}

\paragraph{Drawing}

In first order of $\lambda$ (which we assume to be asked here) the QED vertex
correction by the scalar Higgs boson looks like shown in
Figure~\ref{fig:higgs-vertex}.

\begin{figure}[htbp]
    \centering
    \includegraphics{higgs-vertex}
    \caption{%
        Fermion scattering on a photon with first-order vertex correction by a
        Higgs-boson.
    }
    \label{fig:higgs-vertex}
\end{figure}

This is just like the vertex correction by the additional photon. The
differences are the changed propagator as well as the changed Higgs-fermion
vertex. Those are given by \textcite[(D.49), (D.61)]{romao/aqt} as:
\[
    \frac{\iup}{\four p^2 - m^2 + \iup \epsilon}
    \qquad\text{and}\qquad
    - \iup \frac{g}{2} \frac{m_\mathrm f}{m_\mathrm W} \,.
\]
From the Lagrangian density we deduce that in our case the Higgs vertex is just
$- \iup \lambda$.

Our task here is to compute the form factor $F_2(\four q^2)$ from the Feynman
diagram in Figure~\ref{fig:higgs-vertex}. Then, using the form factor, we shall
compute the $g - 2$ of the electron.

\paragraph{Assembly}

We start with the components of the
expression by assembling the terms. Those matrix elements are always assembled
along the fermion lines. We start from the outgoing fermion, go to the photon
vertex and to the incoming fermion. As a last step we add the Higgs propagator.
\begin{align*}
    \iup \mathcal M
    &= \int \frac{\dif^4 k}{[2 \piup]^4}
    \bar u(\four p')
    [-\iup \lambda]
    \frac{\iup [\myslash k' + m]}{\four k'^2 - m^2 + \iup \epsilon}
    \mat\gamma^\mu
    \frac{\iup [\myslash k + m]}{\four k^2 - m^2 + \iup \epsilon}
    [-\iup \lambda]
    u(\four p)
    \frac{\iup}{\four q^2 - m_\mathrm h^2 + \iup \epsilon}
    \intertext{%
        Then we can reorder this expression and move the coupling constants up
        front.
    }
    &= \iup \lambda^2
    \int \frac{\dif^4 k}{[2 \piup]^4}
    \bar u(\four p')
    \frac{\myslash k' + m}{\four k'^2 - m^2 + \iup \epsilon}
    \mat\gamma^\mu
    \frac{\myslash k + m}{\four k^2 - m^2 + \iup \epsilon}
    u(\four p)
    \frac{1}{\four q^2 - m_\mathrm h^2 + \iup \epsilon}
\end{align*}

Just as done by \textcite[189--196]{Peskin/QFT/1995} we will start with the
denominator. There the use of Feynman parameters will make the denominator more
symmetric.

\paragraph{Denominator}

There are three factors in the denominator, therefore we will have three
Feynman parameters. The form will be
\[
    \int_0^1 \dif x \dif y \dif z \, \delta(x + y + z - 1)
    \frac{3-1}{D^3}
\]
where $D$ is the denominator which has to be built up now.
\begin{align*}
    D
    &= x[\four k'^2 - m^2] + y[\four k^2 - m^2] + z[\four q'^2 - m_\mathrm h^2]
    + [x + y + z] \iup \epsilon
    \intertext{%
        We will expand $\four k' = \four k + \four q$ and $\four q' = \four p -
        \four k$.
    }
    &= x\sbr{[\four k + \four q]^2 - m^2} + y[\four k^2 - m^2] + z\sbr{[\four p
    - \four k]^2 - m_\mathrm h^2} + [x + y + z] \iup \epsilon \\
    &= x[\four k^2 + 2 \four k \cdot \four q + \four q^2 - m^2] + y[\four k^2 -
    m^2] + z[\four p^2 - 2 \four p \cdot \four k + \four k^2 - m_\mathrm h^2] + [x
    + y + z] \iup \epsilon
    \intertext{%
        The terms can be regrouped. Since $x + y + z = 1$, there is just one
        $\four k^2$ and one $\iup \epsilon$ in total.
    }
    &= \four k^2 + 2 \four k \cdot [x \four q - z \four p]
    - m^2 [x + y]
    + x\four q^2 + z[\four p^2 - m_\mathrm h^2] + \iup \epsilon
    \intertext{%
        This is already in a form where we can complete the square in $\four
        k$.
    }
    &= \sbr{\four k^2 + [x \four q - z \four p]}^2 - [x \four q - z \four p]^2
    - m^2 [x + y]
    + x\four q^2 + z\four p^2 - z m_\mathrm h^2 + \iup \epsilon
    \intertext{%
        We will introduce $\four l$ with the same definition as in the book:
        \[
            \four l := \four k + x \four q - z \four p \,.
        \]
        Using this and expanding the second bracket will give us
    }
    &= \four l^2 - x^2 \four q^2 + 2 xz \four q \cdot \four p - z^2 \four p^2
    - m^2 [x + y]
    + x\four q^2 + z\four p^2 - z m_\mathrm h^2 + \iup \epsilon \,.
    \intertext{%
        All the middle terms are put into a term called $\Delta$.
    }
    &= \four l^2 - \Delta + \iup \epsilon
\end{align*}

We can then further simplify that term $\Delta$. Note that all the signs are
flipped since $\Delta$ is subtracted in the denominator.
\begin{align*}
    \Delta
    &= x^2 \four q^2 - 2 xz \four q \cdot \four p + z^2 \four p^2
    + m^2 [x + y]
    - x\four q^2 - z\four p^2 + z m_\mathrm h^2
    \intertext{%
        We factor out the squares.
    }
    &= \four q^2 [x^2 - x]
    + \four p^2 [z^2 - z]
    - 2 xz \four q \cdot \four p
    + m^2 [x + y]
    + z m_\mathrm h^2
    \intertext{%
        The term $\four q \cdot \four p$ can be simplified like follows:
        \[
            \four p + \four q = \four p'
            \iff
            \four p^2 + 2 \four p \cdot \four q + \four q^2 = \four p'^2
            \iff
            2 \four p \cdot \four q
            = \four p'^2 - \four p^2 - \four q^2
            = m^2 - m^2 - \four q^2
            = - \four q^2
        \]
        Therefore we can write the $-2 \four q \cdot \four p$ as $- \four q^2$.
        We obtain
    }
    &= \four q^2 [x^2 - x]
    + \four p^2 [z^2 - z]
    + xz \four q^2
    + m^2 [x + y]
    + z m_\mathrm h^2 \,, \\
    \intertext{%
        which we can further factor out into the first term like so:
    }
    &= \four q^2 [x^2 - x + xz]
    + \four p^2 [z^2 - z]
    + m^2 [x + y]
    + z m_\mathrm h^2 \,.
    \intertext{%
        Then we still have not used $\four p^2 = m^2$. This will reduce the
        number of factors even further.
    }
    &= x \four q^2 [x - 1 + z]
    + m^2 [z^2 - z + x + y]
    + z m_\mathrm h^2
    \intertext{%
        There is even more to simplify. In the first bracket we have
        \[
            x - 1 + z = x + y + z - 1 - y = - y \,.
        \]
        The second bracket contains something similar:
        \[
            z^2 + x + y - z = 
            z^2 + x + y + z - 2z =
            z^2 + 1 - 2z =
            [z - 1]^2 \,.
        \]
        Inserting this will give us our final result for $\Delta$:
    }
    &= - xy \four q^2
    + m^2 [z - 1]^2
    + z m_\mathrm h^2 \,.
\end{align*}
Except for the additional term with the mass of the Higgs-boson, this term
matches the one given by \textcite[(6.44)]{Peskin/QFT/1995}.

\paragraph{Numerator}

The next step is to massage the numerator into a form which can utilize the
Gordon identity. The book tells us that the integral now is symmetric in $\four
l$ (which it really is) and therefore all terms which are odd in $\four l$ can
be dropped directly. That will make it easier.

It starts off innocent-looking:
\begin{align*}
    \bar u(\four p') N u(\four p)
    &= \bar u(\four p') [\myslash k' + m] \mat\gamma^\mu [\myslash k + m]
    u(\four p)
\end{align*}
Then we will concentrate on the numerator itself keeping in mind that it is
sandwiched between Dirac spinors.
\begin{align*}
    N
    &= [\myslash k' + m] \mat\gamma^\mu [\myslash k + m]
    \intertext{%
        We start by expanding all the terms.
    }
    &= \myslash k' \mat\gamma^\mu \myslash k
    + m \myslash k' \mat\gamma^\mu
    + m \mat\gamma^\mu \myslash k
    + m^2 \mat\gamma^\mu
    \intertext{%
        Then we can replace $\four k'$ with $\four k + \four q$.
    }
    &= [\myslash k + \myslash q] \mat\gamma^\mu \myslash k
    + m [\myslash k + \myslash q] \mat\gamma^\mu
    + m \mat\gamma^\mu \myslash k
    + m^2 \mat\gamma^\mu
    \intertext{%
        And to obtain even more terms we replace $\four k$ with
        $\four l - x \four q + z \four p$.
    }
    &= [\myslash l - x \myslash q + z \myslash p + \myslash q] \mat\gamma^\mu [\myslash l - x \myslash q + z \myslash p]
    + m [\myslash l - x \myslash q + z \myslash p + \myslash q] \mat\gamma^\mu
    + m \mat\gamma^\mu [\myslash l - x \myslash q + z \myslash p]
    + m^2 \mat\gamma^\mu
    \intertext{%
        From the first product, only the terms with an even number of $\four
        l$ will contribute, we can drop the other terms. Same with the other
        terms where $\four l$ only occurs once.
    }
    &= \myslash l \mat\gamma^\mu \myslash l + [- x \myslash q + z \myslash p + \myslash q] \mat\gamma^\mu [- x \myslash q + z \myslash p]
    + m [- x \myslash q + z \myslash p + \myslash q] \mat\gamma^\mu
    + m \mat\gamma^\mu [- x \myslash q + z \myslash p]
    + m^2 \mat\gamma^\mu
    \intertext{%
        The expression $\myslash l \mat\gamma^\mu \myslash l$ can be simplified
        using the anticommutation relation.
        \[
            \myslash l \mat\gamma^\mu \myslash l
            = [2 l^\mu - \mat\gamma^\mu \myslash l] \myslash l
            = 2 l^\mu \myslash l - \mat\gamma^\mu \four l^2
        \]
        The first term is odd in $l^\mu$ and we can therefore drop that. All
        the $\myslash p$ on the right side can be replaced by $m$ from the
        equations of motions.
    }
    &= \mat\gamma^\mu \four l^2 + [- x \myslash q + z \myslash p + \myslash q]
    \mat\gamma^\mu [- x \myslash q + z m]
    + m [- x \myslash q + z \myslash p + \myslash q] \mat\gamma^\mu
    + m \mat\gamma^\mu [- x \myslash q + z m]
    + m^2 \mat\gamma^\mu
\end{align*}

\end{document}

% vim: spell spelllang=en tw=79
