\documentclass[11pt, english, fleqn, DIV=15, headinclude, BCOR=1cm]{scrartcl}

\usepackage[bibatend]{../header}

\usepackage{lastpage}
\usepackage{multicol}
\usepackage{simplewick}
\usepackage{slashed}

\newcommand\timeorder{\mathscr T}
\newcommand\normorder{\mathscr N}
\newcommand\eye{\mat 1_4}
\newcommand\myslash[1]{\underline{\slashed{\vec{#1}}}}

\hypersetup{
    pdftitle=
}

\graphicspath{{build/}}

\newcounter{totalpoints}
\newcommand\punkte[1]{#1\addtocounter{totalpoints}{#1}}

\newcounter{problemset}
\setcounter{problemset}{1}

\subject{physics7501 -- Advanced Quantum Field Theory}
\ihead{physics7501 -- Problem Set \arabic{problemset}}

\title{Problem Set \arabic{problemset}}

\newcommand\thegroup{Tutor: Thorsten Schimmanek}

\publishers{\thegroup}
\ofoot{\thegroup}

\author{
    Martin Ueding \\ \small{\href{mailto:mu@martin-ueding.de}{mu@martin-ueding.de}}
}
\ifoot{Martin Ueding}

\ohead{\rightmark}

\begin{document}

\maketitle

\vspace{3ex}

\begin{center}
    \begin{tabular}{rrr}
        Problem & Achieved points & Possible points \\
        \midrule
        \nameref{homework:1} & & \punkte{15} \\
        \midrule
        Total & & \arabic{totalpoints}
    \end{tabular}
\end{center}

\vspace{3ex}

\begin{center}
    \begin{small}
        This document consists of \pageref{LastPage} pages.
    \end{small}
\end{center}

\vspace{5ex}

I would like to scan and upload the problem sets with your corrections to my
website \href{http://martin-ueding.de}{martin-ueding.de}. There, the original
problem set as well as the reviewed one will be licensed under the
“\href{http://creativecommons.org/licenses/by-sa/4.0/}{Creative Commons
Attribution-ShareAlike 4.0 International License}”. Is that okay with you?

Yes $\Box$ \hspace{2cm} No $\Box$

\vspace{3ex}

I started writing this in the “we”-style as I expected that somebody would join
me. Since that has not happened, it will look a little weird now.

\section{Higgs correction to $g - 2$}
\label{homework:1}

\subsection{Gordon identity}

The $\mat\sigma^{\mu\nu}$ are defined by the antisymmetric part of the Dirac
matrices:
\[
    - \iup \mat\sigma^{\mu\nu} := \mat\gamma^{[\mu} \mat\gamma^{\nu]} \,.
\]
This uses the idempotent variant of the antisymmetrization brackets. They can
be expanded using the anticommutation relation.
\begin{align*}
    \iup \mat\sigma^{\mu\nu}
    &= - \mat\gamma^{[\mu} \mat\gamma^{\nu]} \\
    &= - \frac12 \sbr{
        \mat\gamma^{\mu} \mat\gamma^{\nu}
        - \mat\gamma^{\nu} \mat\gamma^{\mu}
    }
    \intertext{%
        We use the anticommutation relation and yield
    }
    &= \eta^{\mu\nu} \eye - \mat\gamma^{\mu} \mat\gamma^{\nu} \,.
\end{align*}

We will also need the anticommutation of a slashed vector with a Dirac matrix.
This works as follows:
\[
    - \mat\gamma^\mu \myslash p'
    = - \mat\gamma^\mu \mat\gamma^\nu p'_\nu
    = \sbr{\mat\gamma^\nu \mat\gamma^\mu - 2 \eta^{\mu\nu}} p'_\nu
    = \myslash p' \mat\gamma^\mu - 2 p'^\mu \,.
\]

Then the Gordon identity can be shown. We start with and right side and insert
the previously derived expression of $\iup \mat\sigma^{\mu\nu}$:
\begin{align*}
    \bar u(\four p') \sbr{\frac{p'^\mu + p^\mu}{2m} + \frac{\iup
    \mat\sigma^{\mu\nu} q_v}{2m}} u(\four p)
    &= \bar u(\four p') \sbr{\frac{p'^\mu + p^\mu}{2m} + \frac{
    \sbr{\eta^{\mu\nu} - \mat\gamma^{\mu} \mat\gamma^{\nu}} q_v}{2m}} u(\four
    p) \,.
    \intertext{%
        We have dropped the $\eye$ here. Then we just factor out the innermost
        bracket and obtain
    }
    &= \bar u(\four p') \sbr{\frac{p'^\mu + p^\mu}{2m} + \frac{ q^\mu -
    \mat\gamma^{\mu} \slashed{\vec q}}{2m}} u(\four p) \,.
    \intertext{%
        The definition of the passed momentum~$\four q$ is $\four p' - \four
        p$. Expanding those will get us to
    }
    &= \bar u(\four p') \frac{1}{2m} \sbr{p'^\mu + p^\mu + p'^\mu - p^\mu -
    \mat\gamma^\mu \myslash p' + \mat\gamma^\mu \myslash p}
    u(\four p) \,.
    \intertext{%
        Some terms can be cancelled easily.
    }
    &= \bar u(\four p') \frac{1}{2m} \sbr{2 p'^\mu -
    \mat\gamma^\mu \myslash p' + \mat\gamma^\mu \myslash p}
    u(\four p)
    \intertext{%
        Then we apply the anticommutation of the slashed outgoing momentum.
    }
    &= \bar u(\four p') \frac{1}{2m} \sbr{2 p'^\mu + \myslash p' \mat\gamma^\mu
    - 2 p'^\mu + \mat\gamma^\mu \myslash p} u(\four p) \\
    &= \bar u(\four p') \frac{1}{2m} \sbr{\myslash p' \mat\gamma^\mu +
    \mat\gamma^\mu \myslash p} u(\four p)
    \intertext{%
        The Dirac equations of motion are $[\myslash p - m] u(\four p) = 0$ and
        $\bar u(\four p') [\myslash p' - m] = 0$. We can apply those to get rid
        of the momenta.
    }
    &= \bar u(\four p') \frac{1}{2} \sbr{\mat\gamma^\mu + \mat\gamma^\mu} u(\four p)
    \intertext{%
        Then we just simplify and obtain the desired result.
    }
    &= \bar u(\four p') \mat\gamma^\mu u(\four p)
\end{align*}

\subsection{Higgs vertex correction}

\paragraph{Drawing}

In first order of $\lambda$ (which we assume to be asked here) the QED vertex
correction by the scalar Higgs boson looks like shown in
Figure~\ref{fig:higgs-vertex}.

\begin{figure}[htbp]
    \centering
    \includegraphics{higgs-vertex}
    \caption{%
        Fermion scattering on a photon with first-order vertex correction by a
        Higgs-boson. The momentum sums get the following shorter names: $\four
        k' = \four k + \four q$ and $\four q' = \four p - \four k$.
    }
    \label{fig:higgs-vertex}
\end{figure}

This is just like the vertex correction by the additional photon. The
differences are the changed propagator as well as the changed Higgs-fermion
vertex. Those are given by \textcite[(D.49), (D.61)]{romao/aqt} as:
\[
    \frac{\iup}{\four p^2 - m^2 + \iup \epsilon}
    \qquad\text{and}\qquad
    - \iup \frac{g}{2} \frac{m_\mathrm f}{m_\mathrm W} \,.
\]
From the Lagrangian density we deduce that in our case the Higgs vertex is just
$- \iup \lambda$. Perhaps then one given by \textcite{romao/aqt} is closely
related to the result we should attain in this homework problem.

Our task here is to compute the form factor $F_2(\four q^2)$ from the Feynman
diagram in Figure~\ref{fig:higgs-vertex}. Then, using the form factor, we shall
compute the $g - 2$ of the electron.

\paragraph{Assembly}

We start with the components of the
expression by assembling the terms. Those matrix elements are always assembled
along the fermion lines. We start from the outgoing fermion, go to the photon
vertex and to the incoming fermion. As a last step we add the Higgs propagator.
\begin{align*}
    \iup \mathcal M
    &= \int \frac{\dif^4 k}{[2 \piup]^4}
    \bar u(\four p')
    [-\iup \lambda]
    \frac{\iup [\myslash k' + m]}{\four k'^2 - m^2 + \iup \epsilon}
    \mat\gamma^\mu
    \frac{\iup [\myslash k + m]}{\four k^2 - m^2 + \iup \epsilon}
    [-\iup \lambda]
    u(\four p)
    \frac{\iup}{\four q^2 - m_\mathrm h^2 + \iup \epsilon}
    \intertext{%
        Then we can reorder this expression and move the coupling constants up
        front.
    }
    &= \iup \lambda^2
    \int \frac{\dif^4 k}{[2 \piup]^4}
    \bar u(\four p')
    \frac{\myslash k' + m}{\four k'^2 - m^2 + \iup \epsilon}
    \mat\gamma^\mu
    \frac{\myslash k + m}{\four k^2 - m^2 + \iup \epsilon}
    u(\four p)
    \frac{1}{\four q^2 - m_\mathrm h^2 + \iup \epsilon}
\end{align*}

Just as done by \textcite[189--196]{Peskin/QFT/1995}, we will start with the
denominator. There the use of Feynman parameters will make the denominator more
symmetric.

\paragraph{Denominator}

There are three factors in the denominator, therefore we will have three
Feynman parameters. The form will be
\[
    \int_0^1 \dif x \dif y \dif z \, \delta(x + y + z - 1)
    \frac{3-1}{D^3}
\]
where $D$ is the denominator which has to be built up now. It is
\begin{align*}
    D
    &= x[\four k'^2 - m^2] + y[\four k^2 - m^2] + z[\four q'^2 - m_\mathrm h^2]
    + [x + y + z] \iup \epsilon \,.
    \intertext{%
        We will expand $\four k' = \four k + \four q$ and $\four q' = \four p -
        \four k$.
    }
    &= x\sbr{[\four k + \four q]^2 - m^2} + y[\four k^2 - m^2] + z\sbr{[\four p
    - \four k]^2 - m_\mathrm h^2} + [x + y + z] \iup \epsilon \\
    &= x[\four k^2 + 2 \four k \cdot \four q + \four q^2 - m^2] + y[\four k^2 -
    m^2] + z[\four p^2 - 2 \four p \cdot \four k + \four k^2 - m_\mathrm h^2] + [x
    + y + z] \iup \epsilon
    \intertext{%
        The terms can be regrouped. Since $x + y + z = 1$, there is just one
        $\four k^2$ and one $\iup \epsilon$ in total.
    }
    &= \four k^2 + 2 \four k \cdot [x \four q - z \four p]
    - m^2 [x + y]
    + x\four q^2 + z[\four p^2 - m_\mathrm h^2] + \iup \epsilon
    \intertext{%
        This is already in a form where we can complete the square in $\four
        k$.
    }
    &= \sbr{\four k + [x \four q - z \four p]}^2 - [x \four q - z \four p]^2
    - m^2 [x + y]
    + x\four q^2 + z\four p^2 - z m_\mathrm h^2 + \iup \epsilon
    \intertext{%
        We will introduce $\four l$ with the same definition as in the book:
        \[
            \four l := \four k + x \four q - z \four p \,.
        \]
        Using this and expanding the second bracket will give us
    }
    &= \four l^2 - x^2 \four q^2 + 2 xz \four q \cdot \four p - z^2 \four p^2
    - m^2 [x + y]
    + x\four q^2 + z\four p^2 - z m_\mathrm h^2 + \iup \epsilon \,.
    \intertext{%
        All the middle terms are put into a term called $\Delta$.
    }
    &= \four l^2 - \Delta + \iup \epsilon
\end{align*}

We can then further simplify that term $\Delta$. Note that all the signs are
flipped since $\Delta$ is subtracted in the denominator.
\begin{align*}
    \Delta
    &= x^2 \four q^2 - 2 xz \four q \cdot \four p + z^2 \four p^2
    + m^2 [x + y]
    - x\four q^2 - z\four p^2 + z m_\mathrm h^2
    \intertext{%
        We factor out the squares.
    }
    &= \four q^2 [x^2 - x]
    + \four p^2 [z^2 - z]
    - 2 xz \four q \cdot \four p
    + m^2 [x + y]
    + z m_\mathrm h^2
    \intertext{%
        The term $\four q \cdot \four p$ can be simplified like follows:
        \[
            \four p + \four q = \four p'
            \iff
            \four p^2 + 2 \four p \cdot \four q + \four q^2 = \four p'^2
            \iff
            2 \four p \cdot \four q
            = \four p'^2 - \four p^2 - \four q^2
            = m^2 - m^2 - \four q^2
            = - \four q^2
        \]
        Therefore we can write the $-2 \four q \cdot \four p$ as $\four q^2$.
        We obtain
    }
    &= \four q^2 [x^2 - x]
    + \four p^2 [z^2 - z]
    + xz \four q^2
    + m^2 [x + y]
    + z m_\mathrm h^2 \,, \\
    \intertext{%
        which we can further factor out into the first term like so:
    }
    &= \four q^2 [x^2 - x + xz]
    + \four p^2 [z^2 - z]
    + m^2 [x + y]
    + z m_\mathrm h^2 \,.
    \intertext{%
        Then we still have not used $\four p^2 = m^2$. This will reduce the
        number of factors even further.
    }
    &= x \four q^2 [x - 1 + z]
    + m^2 [z^2 - z + x + y]
    + z m_\mathrm h^2
    \intertext{%
        There is even more to simplify. In the first bracket we have
        \[
            x - 1 + z = x + y + z - 1 - y = - y \,.
        \]
        The second bracket contains something similar:
        \[
            z^2 + x + y - z = 
            z^2 + x + y + z - 2z =
            z^2 + 1 - 2z =
            [z - 1]^2 \,.
        \]
        Inserting this will give us our final result for $\Delta$:
    }
    &= - xy \four q^2
    + m^2 [z - 1]^2
    + z m_\mathrm h^2 \,.
\end{align*}
Except for the additional term with the mass of the Higgs-boson, this term
matches the one given by \textcite[(6.44)]{Peskin/QFT/1995}.

\paragraph{Numerator}

The next step is to massage the numerator into a form which can utilize the
Gordon identity. The book tells us that the integral now is symmetric in $\four
l$ (which it really is) and therefore all terms which are odd in $\four l$ can
be dropped directly. That will make it easier.

It starts off innocent-looking:
\begin{align*}
    \bar u(\four p') N u(\four p)
    &= \bar u(\four p') [\myslash k' + m] \mat\gamma^\mu [\myslash k + m]
    u(\four p)
\end{align*}
Then we will concentrate on the numerator itself keeping in mind that it is
sandwiched between Dirac spinors.
\begin{align*}
    N
    &= [\myslash k' + m] \mat\gamma^\mu [\myslash k + m] \\
    \intertext{%
        Then we can replace $\four k'$ with $\four k + \four q$.
    }
    &= [\myslash k + \myslash q + m] \mat\gamma^\mu [\myslash k + m] \\
    \intertext{%
        And to obtain even more terms we replace $\four k$ with
        $\four l - x \four q + z \four p$.
    }
    &= [\myslash l - x \myslash q + z \myslash p + \myslash q + m] \mat\gamma^\mu [\myslash l - x \myslash q + z \myslash p + m]
    \intertext{%
        From the first product, only the terms with an even number of $\four
        l$ will contribute, we can drop the other terms. Same with the other
        terms where $\four l$ only occurs once.
    }
    &= \myslash l \mat\gamma^\mu \myslash l + [- x \myslash q + z \myslash p + \myslash q + m] \mat\gamma^\mu [- x \myslash q + z \myslash p + m]
    \intertext{%
        The expression $\myslash l \mat\gamma^\mu \myslash l$ can be simplified
        using the anticommutation relation.
        \[
            \myslash l \mat\gamma^\mu \myslash l
            = [2 l^\mu - \mat\gamma^\mu \myslash l] \myslash l
            = 2 l^\mu \myslash l - \mat\gamma^\mu \four l^2
        \]
        The first term is odd in $l^\mu$ and we can therefore drop that. All
        the $\myslash p$ on the right side can be replaced by $m$ from the
        equations of motions.
    }
    &= - \mat\gamma^\mu \four l^2 + \sbr{[1 - x] \myslash q + z \myslash p + m}
    \mat\gamma^\mu \sbr{- x \myslash q + [z + 1] m}
    \intertext{%
        The form we aim for has the Dirac matrix up front. We therefore
        anticommute the first big bracket beyond the Dirac matrix.
    }
    &= - \mat\gamma^\mu \four l^2 +
    \cbr{
        2 \sbr{[1 - x] q^\mu + z p^\mu + m}
        - \mat\gamma^\mu
        \sbr{[1 - x] \myslash q + z \myslash p + m}
    } \sbr{- x \myslash q + [z + 1] m} \\
    \intertext{%
        We factor the curly brace out to separate terms with and without the
        Dirac matrix.
    }
    &= - \mat\gamma^\mu \four l^2 +
    2 \sbr{[1 - x] q^\mu + z p^\mu + m} \sbr{- x \myslash q + [z + 1] m}
    \\&\quad
    - \mat\gamma^\mu
        \sbr{[1 - x] \myslash q + z \myslash p + m}
    \sbr{- x \myslash q + [z + 1] m}
    \intertext{%
        A $\myslash q$ sandwiched between Dirac spinors gives zero. Therefore
        we can drop that term from the first summand. In the second there still
        is the Dirac matrix which makes things more complicated.
    }
    &= - \mat\gamma^\mu \four l^2 + 2 \sbr{[1 - x] q^\mu + z p^\mu + m} [z + 1] m
    - \mat\gamma^\mu
        \sbr{[1 - x] \myslash q + z \myslash p + m}
    \sbr{- x \myslash q + [z + 1] m}
    \intertext{%
        We will now focus on the last summand. There we can factor out the six
        terms.
    }
    &= - \mat\gamma^\mu \four l^2 + 2 \sbr{[1 - x] q^\mu + z p^\mu + m} [z + 1] m
    \\&\quad
    - \mat\gamma^\mu
    \cbr{
        - [1 - x] \myslash q x \myslash q
        + z \myslash p [z + 1] m
        - m x \myslash q
        + [1 - x] \myslash q [z + 1] m
        - z \myslash p x \myslash q
        + m [z + 1] m
    }
    \intertext{%
        This can be simplified, of course. The slashed squares can be replaced
        by normal squares. The mixed scalar product is replaced with the
        identity we previously showed. Then the stray $\myslash p$ is used to
        yield an $m$.
    }
    &= - \mat\gamma^\mu \four l^2 + 2 \sbr{[1 - x] q^\mu + z p^\mu + m} [z + 1] m
    \\&\quad
    - \mat\gamma^\mu
    \cbr{
        - [1 - x] x \four q^2
        + z [z + 1] m^2
        - x m \myslash q
        + [1 - x] [z + 1] m \myslash q
        + \frac12 x z \four q^2
        + [z + 1] m^2
    }
    \intertext{%
        We can combine the two $m^2$ terms now. We also combine the terms with
        $\myslash q$ and $\four q^2$.
    }
    &= - \mat\gamma^\mu \four l^2 + 2 \sbr{[1 - x] q^\mu + z p^\mu + m} [z + 1] m
    \\&\quad
    - \mat\gamma^\mu
    \cbr{
        \sbr{\frac12 xz - [1 - x] x} \four q^2
        + \sbr{[1 - x] [z + 1] - x} m \myslash q
        + [z + 1]^2 m^2
    }
    \intertext{%
        The $\myslash q$ has to be disposed still. We will do this by inserting
        $\myslash p' - \myslash p$, anticommuting the first summand to the
        other side of the Dirac matrix and let it act on $\bar u$. In order to
        make it shorter, we will show that $\mat\gamma^\mu \myslash q$
        simplifies a lot in a Dirac spinor sandwich:
        \[
            \begin{aligned}
                \bar u \mat\gamma^\mu \myslash q u
                &= \bar u \mat\gamma^\mu [\myslash p' - \myslash p] u \\
                &= \bar u \mat\gamma^\mu \myslash p' u
                - \bar u \mat\gamma^\mu \myslash p u \\
                &= \bar u [- p'^\mu + \myslash p'] \mat\gamma^\mu u
                - \bar u \mat\gamma^\mu \myslash p u \\
                &= \bar u [- p'^\mu + m] \mat\gamma^\mu u
                - \bar u \mat\gamma^\mu m u \\
                &= - \bar u p'^\mu u \,.
            \end{aligned}
        \]
        We will use that now to simplify our numerator.
    }
    &= - \mat\gamma^\mu \four l^2 + 2 \sbr{[1 - x] q^\mu + z p^\mu + m} [z + 1] m
        + \sbr{[1 - x] [z + 1] - x} m p'^\mu
    \\&\quad
    - \mat\gamma^\mu
    \cbr{
        \sbr{\frac12 xz - [1 - x] x} \four q^2
        + [z + 1]^2 m^2
    } \\
    \intertext{%
        Then we can group those terms by pre-factors.
    }
    &=
    \mat\gamma^\mu
    \cbr{
        - \four l^2
        - \sbr{\frac12 xz - [1 - x] x} \four q^2
        - [z + 1]^2 m^2
    }
    + p'^\mu \sbr{[1 - x] [z + 1] - x} m
    \\&\quad
    + p^\mu 2 z [z+1] m
    + q^\mu 2 [1 - x] [z + 1] m
    + 2 [z+1]m^2
\end{align*}

This does not nicely fall into the desired form (given in the book) of
\[
    \mat\gamma^\mu A + [p'^\mu + p^\mu] B + q^\mu C \,.
\]
The constant summand should not be there by this scheme. In principle, the
symmetric and antisymmetric part in $p'^\mu$ and $p^\mu$ should be decomposable
from anything.
In general we can write the symmetric and antisymmetric part as:
\[
    [p' + p] B + [p' - p] C
    = p' [B + C] + p [B - C] \,.
\]
So far we have some multiple of $p'$ which we will now call $X$. The multiple
of $p$ is called $Y$. Then we have:
\begin{align*}
    X &= B + C \,, \\
    Y &= B - C \,.
\end{align*}
Those equations can be solved for $B$ and $C$ by $B = [X + Y]/2$ and $C = [X -
Y]/2$.

In our example we have
\[
    X = \sbr{[1 - x] [z + 1] - x} m
    \quad
    Y = 2 z [z+1] m \,.
\]
We ignore the multiple of $q$ since we will just get more $q$ anyway. The
solutions of $B$ and $C$ are then:
\[
    B = \frac m2 \sbr{\sbr{[1 - x] [z + 1] - x} + 2 z [z+1]}
    \quad
    C = \frac m2 \sbr{\sbr{[1 - x] [z + 1] - x} - 2 z [z+1]}
\]
$q^\mu C$ is then added to the multiple of $q^\mu$ that we already have. We are
only interested in $B$ here as that is the part that will give the $g-2$. The
coefficient of $q^\mu$ has to vanish due to the Ward identity, so we will still
compute that to see whether we have made a mistake so far.
\begin{align*}
    q^\mu \cdot \ldots
    &= q^\mu m \cbr{
        2 [1 - x] [z + 1] + \frac 12 \sbr{\sbr{[1 - x] [z + 1] - x} - 2 z [z+1]}
    } \\
    &= q^\mu m \cbr{
    \frac 32 z + \frac 52 - \frac 52 xz - 3x - z^2
    }
\end{align*}
That does not seem to vanish. If it was antisymmetric in the variables $x$ and
$y = 1 - x - z$, it would vanish; the denominator is symmetric in $x$ and $y$
and the integration bounds are the same for each Feynman parameter.

It does not work out. Therefore we must have made some mistake along the way.
We have checked this calculation and improved it, but it still does not work
out.

\paragraph{Gordon identity}

The part which is of interest to us is the $B$. Writing it with the Gordon
identity term give us
\[
    \frac{\iup \mat\sigma^{\mu\nu} q_\nu}{2m}
    m^2 \sbr{\sbr{[1 - x] [z + 1] - x} + 2 z [z+1]}
\]

\end{document}

% vim: spell spelllang=en tw=79
