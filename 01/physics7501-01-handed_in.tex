\documentclass[11pt, english, fleqn, DIV=15, headinclude, BCOR=1cm]{scrartcl}

\usepackage[bibatend]{../header}

\usepackage{lastpage}
\usepackage{multicol}
\usepackage{simplewick}
\usepackage{slashed}

\newcommand\timeorder{\mathscr T}
\newcommand\normorder{\mathscr N}
\newcommand\eye{\mat 1_4}
\newcommand\myslash[1]{\underline{\slashed{\vec{#1}}}}

\hypersetup{
    pdftitle=
}

\graphicspath{{build/}}

\newcounter{totalpoints}
\newcommand\punkte[1]{#1\addtocounter{totalpoints}{#1}}

\newcounter{problemset}
\setcounter{problemset}{1}

\subject{physics7501 -- Advanced Quantum Field Theory}
\ihead{physics7501 -- Problem Set \arabic{problemset}}

\title{Problem Set \arabic{problemset}}

\newcommand\thegroup{Tutor: Thorsten Schimmanek}

\publishers{\thegroup}
\ofoot{\thegroup}

\author{
    Martin Ueding \\ \small{\href{mailto:mu@martin-ueding.de}{mu@martin-ueding.de}}
}
\ifoot{Martin Ueding}

\ohead{\rightmark}

\begin{document}

\maketitle

\vspace{3ex}

\begin{center}
    \begin{tabular}{rrr}
        Problem & Achieved points & Possible points \\
        \midrule
        \nameref{homework:1} & & \punkte{15} \\
        \midrule
        Total & & \arabic{totalpoints}
    \end{tabular}
\end{center}

\vspace{3ex}

\begin{center}
    \begin{small}
        This document consists of \pageref{LastPage} pages.
    \end{small}
\end{center}

\section{Higgs correction to $g - 2$}
\label{homework:1}

\subsection{Gordon identity}

The $\mat\sigma^{\mu\nu}$ are defined by the antisymmetric part of the Dirac
matrices:
\[
    - \iup \mat\sigma^{\mu\nu} := \mat\gamma^{[\mu} \mat\gamma^{\nu]} \,.
\]
This uses the idempotent variant of the antisymmetrization brackets. They can
be expanded using the anticommunitation relation.
\begin{align*}
    \iup \mat\sigma^{\mu\nu}
    &= - \mat\gamma^{[\mu} \mat\gamma^{\nu]} \\
    &= - \frac12 \sbr{
        \mat\gamma^{\mu} \mat\gamma^{\nu}
        - \mat\gamma^{\nu} \mat\gamma^{\mu}
    }
    \intertext{%
        We use the anticommunitation relation and yield
    }
    &= \eta^{\mu\nu} \eye - \mat\gamma^{\mu} \mat\gamma^{\nu} \,.
\end{align*}

We will also need the anticommutation of a slashed vector with a Dirac matrix.
This works as follows:
\[
    - \mat\gamma^\mu \myslash p'
    = - \mat\gamma^\mu \mat\gamma^\nu p'_\nu
    = \sbr{\mat\gamma^\nu \mat\gamma^\mu - 2 \eta^{\mu\nu}} p'_\nu
    = \myslash p' \mat\gamma^\mu - 2 p'^\mu \,.
\]

Then the Gordon identity can be shown. We start with and right side and insert
the previously derived expression of $\iup \mat\sigma^{\mu\nu}$:
\begin{align*}
    \bar u(\four p') \sbr{\frac{p'^\mu + p^\mu}{2m} + \frac{\iup
    \mat\sigma^{\mu\nu} q_v}{2m}} u(\four p)
    &= \bar u(\four p') \sbr{\frac{p'^\mu + p^\mu}{2m} + \frac{
    \sbr{\eta^{\mu\nu} - \mat\gamma^{\mu} \mat\gamma^{\nu}} q_v}{2m}} u(\four
    p) \,.
    \intertext{%
        We have dropped the $\eye$ here. Then we just factor out the innermost
        bracket and obtain
    }
    &= \bar u(\four p') \sbr{\frac{p'^\mu + p^\mu}{2m} + \frac{ q^\mu -
    \mat\gamma^{\mu} \slashed{\vec q}}{2m}} u(\four p) \,.
    \intertext{%
        The definition of the passed momentum~$\four q$ is $\four p' - \four
        p$. Expanding those will get us to
    }
    &= \bar u(\four p') \frac{1}{2m} \sbr{p'^\mu + p^\mu + p'^\mu - p^\mu -
    \mat\gamma^\mu \myslash p' + \mat\gamma^\mu \myslash p}
    u(\four p) \,.
    \intertext{%
        Some terms can be cancelled easily.
    }
    &= \bar u(\four p') \frac{1}{2m} \sbr{2 p'^\mu -
    \mat\gamma^\mu \myslash p' + \mat\gamma^\mu \myslash p}
    u(\four p)
    \intertext{%
        Then we apply the anticommunitation of the slashed outgoing momentum.
    }
    &= \bar u(\four p') \frac{1}{2m} \sbr{2 p'^\mu + \myslash p' \mat\gamma^\mu
    - 2 p'^\mu + \mat\gamma^\mu \myslash p} u(\four p) \\
    &= \bar u(\four p') \frac{1}{2m} \sbr{\myslash p' \mat\gamma^\mu +
    \mat\gamma^\mu \myslash p} u(\four p)
    \intertext{%
        The Dirac equations of motion are $[\myslash p - m] u(\four p) = 0$ and
        $\bar u(\four p') [\myslash p' + m] = 0$. We can apply those to get rid
        of the momenta.
    }
    &= \bar u(\four p') \frac{1}{2} \sbr{- \mat\gamma^\mu + \mat\gamma^\mu} u(\four p)
\end{align*}
If it were not for some sign error somewhere, this would be the right side of
the given Gordon identity.

\end{document}

% vim: spell spelllang=en tw=79
