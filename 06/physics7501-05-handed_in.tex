\documentclass[11pt, english, fleqn, DIV=15, headinclude]{scrartcl}

\usepackage[bibatend]{../header}
\usepackage{../my-boxes}

\usepackage{lastpage}
\usepackage{multicol}
\usepackage{simplewick}
\usepackage{multicol}
\usepackage{slashed}
\usepackage{subcaption}
\usepackage{cancel}

\newcommand\timeorder{\mathscr T}
\newcommand\normorder{\mathscr N}
\newcommand\eye{\mat 1_4}
\newcommand\fourslash[1]{\slashed{\four{#1}}}
\newcommand\T{\mathrm T}

\hypersetup{
    pdftitle=
}

\graphicspath{{build/}}

\newcounter{totalpoints}
\newcommand\punkte[1]{#1\addtocounter{totalpoints}{#1}}

\newcounter{problemset}
\setcounter{problemset}{5}

\subject{physics7501 -- Advanced Quantum Field Theory}
\ihead{physics7501 -- Problem Set \arabic{problemset}}

\title{Problem Set \arabic{problemset}}

\newcommand\thegroup{Tutor: Thorsten Schimannek}

\publishers{\thegroup}
\ofoot{\thegroup}

\author{
    Martin Ueding \\ \small{\href{mailto:mu@martin-ueding.de}{mu@martin-ueding.de}}
}
\ifoot{Martin Ueding}

\ohead{\rightmark}

\begin{document}

\maketitle

\vspace{3ex}

\begin{center}
    \begin{tabular}{rrr}
        Problem & Achieved points & Possible points \\
        \midrule
        \nameref{homework:1} & & \punkte{15} \\
        \midrule
        Total & & \arabic{totalpoints}
    \end{tabular}
\end{center}

\vspace{3ex}

\begin{center}
    \begin{small}
        This document consists of \pageref{LastPage} pages.
    \end{small}
\end{center}

\section{Grassmann numbers and the fermionic path integral}
\label{homework:1}

\subsection{Taylor series and shift}

The Grassmann numbers themselves are the generators of the algebra. It probably
makes sense to add 1 to the generators also in order to have $\C$  be part of
the algebra as well. We will call that set $\setname G$.\footnote{%
    The notation $\mathbb C$ and $\mathbb G$ might be more common for fields. I like to go
    with the ISO 80000-2 standard (which allows both variants) and also take
    the heritage of those letters (like $\mathbb C$) into account. In \LaTeX\
    they are called “blackboard bold” which means this way of writing letters
    is the attempt to write bold on the blackboard. Since we have more powerful
    typesetting in \LaTeX\ than on the board, I will sure make use of them and
    display the letters in bold. Same goes for underline for emphasis, which is
    a no-go in typed documents. Perhaps it is not the best idea to deviate too
    much from common notation even though its usage in a powerful typesetting
    system feels rather backwards.
}
The function $f$ then is
\[
    f \colon \setname G \to \setname G \,.
\]
Since the Grassmann numbers anticommute they square to zero.\footnote{%
    On the Wikipedia article about Grassmann numbers they are called “non-zero
    square roots of zero” which I find a quite intriguing thought. Almost ten
    years ago the concept of $\sqrt{-1}$ has baffled me, now this is the next
    step. This time I know that field axioms are not god-given and therefore
    algebra structures can be defined like one wishes to furnish those axioms.
    Still, it is an interesting way of stating the squaring to zero feature.
}
This also means that a function of a single Grassmann variable cannot have
arbitrary form. Analytic functions can always be written as a power series,
functions can at most be linear in Grassmann variables which really limits
their complexity. Any function can be written as
\[
    f(\theta) = A + B \theta \,,
\]
where we use the same notation as \textcite[299]{Peskin/QFT/1995}. Higher order
terms in $\theta$ vanish directly.

\paragraph{Taylor series}

The Taylor series of such a function in $\theta$ just has two terms as higher
powers of $\theta$ will vanish. Also higher derivatives in $\theta$ will also
vanish when one looks at the general form motivated above.

\paragraph{Integral}

The integration must be invariant under shifts in the integration variable. We
have
\begin{align*}
    \int \dif \theta \, f(\theta)
    &= \int \dif \theta \, [A + B \theta]
    \intertext{%
        as the general form. Now shifting $\theta$ by $\eta$ does not change
        $\dif \theta$ and we have
    }
    &= \int \dif \theta \, [A + B [\theta + \eta]]
    \intertext{%
        which we can expand to yield
    }
    &= \int \dif \theta \, [A + B \theta + B \eta] \,.
    \intertext{%
        It is best to regroup the terms such that the constant and linear terms
        clearly pop out:
    }
    &= \int \dif \theta \, [\underbrace{A + B \eta}_{A'} + B \theta] \,.
\end{align*}
The result of the integration must not be changed when $A \to A'$ it performed.
Therefore the integral must not depend on $A$ as it is a linear function and
has to be rather simple. The only dependence can be $B$ then. A definition that
fulfils this is
\[
    \int \dif \theta \, [A + B\theta] = k B \,.
\]

Could this $k$ be a Grassmann number, i.e.\ $k \in \setname G$? $B$ is
definitely a regular complex number, so the product would not directly
vanish. However, the integral would then be a $\setname G$-linear function
which does not play well with integrations in $\C$. It therefore makes sense to
define $k \in \C$ and therefore $kB = c$ is a regular complex number.

\paragraph{Integration and Differentiation}

On the problem set it is noted that integration and differentiation are
equivalent. Since there are only two possible monoms of Grassmann variables
($A$ and $B \theta$), this is easy to check. The integral and derivative with
respect to $\theta$ of the first term is zero. The integral and derivative of
the second is just $B$. Setting $k = 1$ makes this truly the same thing which
is probably very handy along the road.

\subsection{Unitary transformation}

The unitary transformation is probably a \emph{special} unitary transformation.
In the case of a general unitary transformation the determinant might be a
complex phase factor which does change the integrals unless complex conjugated
counter-terms are involved as well. So $U$ is taken from $\SU(n)$ and $\mathrm
U(1)$ is excluded here? Since the problem talks about matrices, $\SU(n)$ would
not really be a problem. So this just applies when there are multiple fields?
\Textcite[301]{Peskin/QFT/1995} show that the integral with $\dif \theta^* \dif
\theta$ is invariant under $\mathrm U(n)$ but integrals with $\dif \theta$
depend on $\det(U)$ which might have a phase factor.

We will now show that the transformation introduces a factor on $\det(U)$. The
invariance has to be evaluated with respect of that determinant then which is
unity for the special transformations.

\begin{table}
    \begin{question}
        $\mathrm U(1)$ only has one generator, 1. Therefore the group $\SU(1) =
        \{ 1 \}$ is a quite trivial group. The groups $\SU(n)$ all have $n^2-1$
        generators and $\mathrm U(n)$ should have $n^2$ generators. Is it in
        general that
        \[
            \mathrm U(n) \simeq \mathrm U(1) \times \SU(n)
        \]
        holds?
    \end{question}
\end{table}

\end{document}

% vim: spell spelllang=en tw=79
