\documentclass[11pt, english, fleqn, DIV=15, headinclude]{scrartcl}

\usepackage[bibatend]{../header}
\usepackage{../my-boxes}

\usepackage{lastpage}
\usepackage{multicol}
\usepackage{simplewick}
\usepackage{multicol}
\usepackage{slashed}
\usepackage{subcaption}
\usepackage{cancel}

\newcommand\timeorder{\mathscr T}
\newcommand\normorder{\mathscr N}
\newcommand\eye{\mat 1_4}
\newcommand\fourslash[1]{\slashed{\four{#1}}}
\newcommand\T{\mathrm T}

\hypersetup{
    pdftitle=
}

\graphicspath{{build/}}

\newcounter{totalpoints}
\newcommand\punkte[1]{#1\addtocounter{totalpoints}{#1}}

\newcounter{problemset}
\setcounter{problemset}{5}

\subject{physics7501 -- Advanced Quantum Field Theory}
\ihead{physics7501 -- Problem Set \arabic{problemset}}

\title{Problem Set \arabic{problemset}}

\newcommand\thegroup{Tutor: Thorsten Schimannek}

\publishers{\thegroup}
\ofoot{\thegroup}

\author{
    Martin Ueding \\ \small{\href{mailto:mu@martin-ueding.de}{mu@martin-ueding.de}}
}
\ifoot{Martin Ueding}

\ohead{\rightmark}

\begin{document}

\maketitle

\vspace{3ex}

\begin{center}
    \begin{tabular}{rrr}
        Problem & Achieved points & Possible points \\
        \midrule
        \nameref{homework:1} & & \punkte{20} \\
        \midrule
        Total & & \arabic{totalpoints}
    \end{tabular}
\end{center}

\vspace{3ex}

\begin{center}
    \begin{small}
        This document consists of \pageref{LastPage} pages.
    \end{small}
\end{center}

\vspace{3ex}

\begin{multicols}{2}
    I am not sure whether I got all the renormalization straight so far. When
    we first started with Feynman diagrams it seemed that there was no need for
    those field strength renormalization factors~$Z$ at all. To sort those
    things out I would like to write down what I think I have understood such
    that you can point to where my misunderstanding starts.

    The $S$-matrix translates states from the past infinity to the future
    infinity. We use that for scattering as we assume that the time of the
    scattering is limited and the particles did not interact with each other
    before the time of the scattering. The assumption of non-interaction
    between the distinct particles seems reasonable as they are far away from
    each other. The propagation from the past infinity to the scattering was
    somewhat excluded as we only looked into amputated diagrams. The amputated
    diagrams could be computed with the Feynman rules and the various tricks
    that we have learned during the course so far.

    I assume that taking amputated diagrams means to approximate the time
    evolution from past infinity to the scattering with the free theory. If
    that is indeed the case, we only made an approximation there such that the
    $S$-matrix would be easier to compute. The disconnected diagrams (the
    vacuum bubbles) factored out of the correlation functions and therefore do
    not contribute at all. I took the latter statement as an exact statement
    (the proof looked exact). Perhaps I have misunderstood the amputated
    diagrams as being an analogous thing and thought that this was not an
    approximation.

    From the charge and mass renormalization I have the impression that the
    amputation is an approximation that we would like to get rid of in
    Chapter~7 of \emph{the book}. The propagation from past infinity (and to
    future infinity) is also perturbed by the various interactions like the
    electron self-energy diagrams to arbitrary order. Now that I see that this
    changes the mass and charge of the involved particles, it seems incorrect
    to assume that even a single particle propagates as a free particle as
    there is always the vacuum which is not empty at all.

    In order to compute the $S$-matrix elements correctly we would have to
    include all the diagrams with external leg corrections as well. This would
    increase the number of diagrams significantly. It would give the right
    result to the given order, right? It is just a lot work that we would like
    to get around. So this is where the LSZ reduction formula comes into play
    which relates the amputated diagram to the $S$-matrix element with
    additional factors, the renormalization factors~$Z$. Therefore we can still
    just compute the amputated diagrams and add some pre-computed factors $Z$
    to the final result. This result will have time evolution with the
    interacting time evolution operator built in. And the time evolution
    operator is just another word for the propagators, right? The propagators
    obtain a correction like $\Sigma$ or $\Pi$ as we have computed in class and
    in the exercises.

    Those corrections to the propagator are a bit hard to manage and we do not
    need to carry the $\Sigma$ and $\Pi$ around explicitly. The in- and
    out-states that we want to use with the $S$-matrix are on-shell particles
    which fulfil the Dirac equation $[\fourslash p - m] u(\four p) = 0$ with
    the physical mass~$m$. Therefore the fermion propagator $\iup /
    [\fourslash p - m]$ will have a pole at the physical mass. The behaviour at
    the pole is characterized by the residue at the pole. Therefore we can just
    encapsulate the difference in the free and interacting propagator \emph{for
    on-shell particles} in the residue of the propagator at the physical mass.
    The whole time evolution from the past infinity to the scattering event
    occurs with a single particle which has to me on-shell. Therefore the
    amputated diagrams is just multiplied with the residue of the pole. And
    that residue is called $Z$ (or $\sqrt Z$?) and hence enters with LSZ
    formula.

    All this means that we can continue as usual: Compute the amputated Feynman
    diagram using the various rules and tricks. Only at the end we tack on a
    number of $Z$ factors to accommodate the corrections to the propagator for
    incoming and outgoing particles. The propagators in the amputated diagram
    are not corrected as we compute the diagram to a given order using the
    normal propagators. Using the corrected propagators there as well would
    shift the order of the diagram to a higher value which we do not
    necessarily want.

    \Textcite{Peskin/QFT/1995} have Chapter~7 about all those ideas. There are
    $Z$ and $Z_1$, $Z_2$ popping up here and there. Is the $Z$ the one and only
    renormalization factor for the scalar $\phi^4$-theory as there is only one
    kind of particle? Then the $Z_1$ seem to come from the vertex diagram where
    the transition from $\mat\gamma^\mu$ to $\mat\Gamma^\mu$ is done. Then
    there is $Z_2$ from the self-energy of the electron. In the end they turn
    out to be the same anyway in order to keep the photon massless. Then I
    think I saw a $Z_3$ somewhere as well, that was the correction to the
    electric charge from the self-energy of the photon which we have computed
    in the homework, right? The notation also gets a tad overloaded by the use
    of $\Sigma$ for the exact result and $\Sigma_2$ as a second order result.
    Are any of the $Z_2$ or $Z_1$ just a second or first order approximation to
    the $Z$ or distinct quantities? In the lecture there were $Z_2^1$ and
    similar constructs, so I assume that $Z_2$ and $Z_1$ are distinct as the
    book also suggests.
\end{multicols}

\section{Yukawa corrections}
\label{homework:1}

\begin{multicols}{2}
    The contribution to $Z_1$ was second in the book, the first renormalization
    constant introduced was $Z$ and $Z_2$. The way they are introduced they are
    the probability to create a particular one-particle state from the vacuum.
    Annihilation is the same as the modulus squared is taken. That $Z_2$ was
    just added to the left side of the equation which relates the vertex
    $\mat\Gamma^\mu$ to the form factors~$F_1$ and $F_2$
    \parencite[(7.46)]{Peskin/QFT/1995}.

    We need $F_1(0) = 1$ to make the theory consistent. Currently I think that
    this needs to be fulfilled to give the Ward identity which is equivalent to
    a massless photon? The problem with adding $Z_2$ on the left side is that
    $F_1(0)$ is not necessarily unity afterwards. To remedy this one introduced
    a $Z_1$ such that it would cancel the $Z_2$ in the case of $\four q = 0$,
    the case of no momentum transfered \parencite[(7.47)]{Peskin/QFT/1995}.

    Since the exact $\mat\Gamma^\mu$ and the exact $Z_2$ and $Z_1$ depend on
    all orders of $\alpha$, we cannot compute them here. They are different
    though, since the vertex correction diagram contains a different propagator
    with mass and a different vertex without a Dirac matrix. This will alter
    the overall invariant matrix element of that vertex correction diagram in
    first order. Higher orders will be different as well. Corrections to the
    electron propagator with virtual $\phi$-particles will also differ from the
    corrections with photons.

    Perhaps we are just asked to verify the whole thing to second order in
    $\lambda^2$ which is the first interesting order? Then the computation of
    the first order vertex diagram would suffice and we could compare them to
    lowest order at least. Computing either diagram with dimensional
    regularization should be doable, especially since there are the Dirac
    matrices only from the electron propagators and not also from the
    electron-photon vertices.

    The $Z_2$ is defined as the residue of the pole in the electron propagator
    \parencite[243]{Peskin/QFT/1995}, therefore we should be able to compute
    that from the geometric series of first-order self-energy diagrams. Here I
    would proceed as we did with the self-energy of the electron with the
    photon. The 1PI-diagram could be computed, perhaps dimensional
    regularization is needed there. Then the expression $\Sigma_2$ (or was it
    $\Pi_2?$) can be put into the geometric series. Hopefully it would simplify
    enough that it can be evaluated. Then the residue would give me $Z_2$.

    I still have no idea how to compute $Z_1$. It is introduced as a “second
    rescaling factor” \parencite[230]{Peskin/QFT/1995} which has to be equal to
    $Z_2$ such that it works. They refer to the proof in section~7.4 which is
    exactly where the Ward-Takahashi identity is proven. From the problem
    statement it sounds like one should be able to extract it from the vertex
    diagram.

    However I fail to see where I could really start. Now my time runs out and
    I cannot do a Wick rotation and do this homework in imaginary time.
    Therefore I just have to leave you with three pages of questions. Sorry.
    Having read the sections 7.1--7.4 a couple times now I still have no clear
    idea how all these concepts relate. I would greatly appreciate if you could
    give me a few pointers into the right directions to clear it up.
\end{multicols}

\end{document}

% vim: spell spelllang=en tw=79
