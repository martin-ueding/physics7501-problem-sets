\documentclass[11pt, english, fleqn, DIV=15, headinclude]{scrartcl}

\usepackage[bibatend]{../header}
\usepackage{../my-boxes}

\usepackage{lastpage}
\usepackage{multicol}
\usepackage{simplewick}
\usepackage{multicol}
\usepackage{adjustbox}
\usepackage{slashed}
\usepackage{subcaption}
\usepackage{cancel}
\usepackage{tikzsymbols}
\usepackage{placeins}

\newcommand\timeorder{\mathscr T}
\newcommand\normorder{\mathscr N}
\newcommand\eye{\mat 1_4}
\newcommand\fourslash[1]{\slashed{\four{#1}}}
\newcommand\T{\mathrm T}

\hypersetup{
    pdftitle=
}

\graphicspath{{build/}}

\newcounter{totalpoints}
\newcommand\punkte[1]{#1\addtocounter{totalpoints}{#1}}

\newcounter{problemset}
\setcounter{problemset}{11}

\subject{physics7501 -- Advanced Quantum Field Theory}
\ihead{physics7501 -- Problem Set \arabic{problemset}}

\title{Problem Set \arabic{problemset}}

\newcommand\thegroup{Tutor: Thorsten Schimannek}

\publishers{\thegroup}
\ofoot{\thegroup}

\author{
    Martin Ueding \\ \small{\href{mailto:mu@martin-ueding.de}{mu@martin-ueding.de}}
}
\ifoot{Martin Ueding}

\ohead{\rightmark}

\begin{document}

\maketitle

\vspace{3ex}

\begin{center}
    \begin{tabular}{rrr}
        \toprule
        Problem & Achieved points & Possible points \\
        \midrule
        \nameref{homework:1} & & \punkte{20} \\
        \midrule
        Total & & \arabic{totalpoints} \\
        \bottomrule
    \end{tabular}
\end{center}

\vspace{3ex}

\begin{center}
    \begin{small}
        This document consists of \pageref{LastPage} pages.
    \end{small}
\end{center}

\section{Renormalization of Yukawa theory}
\label{homework:1}

\newcommand\mdim M

\subsection{Feynman rules}

The Feynman rules are
\begin{align*}
    \vcenter{\hbox{\includegraphics{scalar-propagator}}}
    &= \frac{\iup}{\four p^2 - m^2}
    \\
    \vcenter{\hbox{\includegraphics{dirac-propagator}}}
    &= \frac{\iup \fourslash p}{\four p^2 - M^2}
    \\
    \vcenter{\hbox{\includegraphics{vertex}}}
    &= g \mat\gamma^5 \,.
\end{align*}

The mass dimension $\mdim$ of $g$ is zero, it is a dimensionless coupling
constant. The rules for the superficial degree of divergence are the same as in
QED as the boson propagator has the same mass dimension. Every scalar
propagator gives $\mdim = 2$, each fermion propagator gives $\mdim = 3$. Each
vertex gives $\mdim = -4$ due to the momentum conserving $\delta$-distribution.
The empty diagram starts with $\mdim = 4$ as as we do not want to count the
$\delta$-distribution that conserves the total momentum. Therefore we have
\[
    D = 4 + 3 P_\psi + 2 P_\phi - 4 V \,.
\]
This is not helpful as we want to express this in the external legs. A little
trying makes it clear that adding fermion loops inside of the diagram does not
change $D$ at all. Therefore it is only depended on the external legs, just as
with QED\@. We looked at a couple diagrams and made the Ansatz
\[
    D = a + b E_\psi + c E_\phi
\]
and solved the linear system of equations. We obtain
\[
    D = 4 - \frac32 E_\psi - E_\phi \,.
\]

\begin{question}
    Is that the recommended way to get these sort of relations for the
    superficial degree of divergence? In the tutorial it seemed so easy for you
    to construct it. I always have a hard time figuring out the relations
    between the number of propagators, vertices and external legs, especially
    if different types of vertices are involved.
\end{question}

\subsection{Superficially divergent amplitudes}

\paragraph{List of divergent amplitudes}

It is not possible to have a diagram with an odd number of fermionic legs. If
we have more then two fermionic legs, the diagram will converge. If a diagram
has more than four scalar legs, it will converge as well. Therefore we are left
with the following selection of possible divergent diagrams shown in
Table~\ref{tab:divergent}.

\begin{table}
    \centering
    \begin{tabular}{S|SS}
        \toprule
        {$E_\phi$ \textbackslash $E_\psi$} & 0 & 2 \\
        \midrule
        0 & 4 & 1 \\
        1 & 3 & 0 \\
        2 & 2 & -1 \\
        3 & 1 & -2 \\
        4 & 0 & -3 \\
        \bottomrule
    \end{tabular}
    \caption{%
        Superficially divergent diagram configurations.
        The first row has the number of external fermion lines. The first
        column has the number of external scalar lines.
    }
    \label{tab:divergent}
\end{table}

Only two diagrams with two fermionic legs are divergent and a lot of diagrams
with no fermionic legs are divergent. All those diagrams are depicted in
Figure~\ref{fig:divergent}.

\begin{figure}
    \centering
    \begin{subfigure}[c]{0.3\linewidth}
        \centering
        \includegraphics{loop-0}
        \caption{%
        }
        \label{fig:/1}
    \end{subfigure}
    \hfill
    \begin{subfigure}[c]{0.3\linewidth}
        \centering
        \includegraphics{loop-1}
        \caption{%
        }
        \label{fig:/1}
    \end{subfigure}
    \hfill
    \begin{subfigure}[c]{0.3\linewidth}
        \centering
        \includegraphics{loop-2}
        \caption{%
        }
        \label{fig:/1}
    \end{subfigure}
    \hfill
    \begin{subfigure}[c]{0.3\linewidth}
        \centering
        \includegraphics{loop-3}
        \caption{%
        }
        \label{fig:/1}
    \end{subfigure}
    \hfill
    \begin{subfigure}[c]{0.3\linewidth}
        \centering
        \includegraphics{loop-4}
        \caption{%
        }
        \label{fig:loop-4}
    \end{subfigure}
    \hfill
    \begin{subfigure}[c]{0.3\linewidth}
        \centering
        \includegraphics{dirac-propagator}
        \caption{%
        }
        \label{fig:/1}
    \end{subfigure}
    \hfill
    \begin{subfigure}[c]{0.3\linewidth}
        \centering
        \includegraphics{vertex}
        \caption{%
        }
        \label{fig:/1}
    \end{subfigure}
    \caption{%
        Superficially divergent diagrams
    }
    \label{fig:divergent}
\end{figure}

Higher order corrections to all these diagrams would have the same superficial
degree of divergence. We hope that only the lowest order is asked here.

\paragraph{Expansion in external momenta}

Let the external momenta be of order $k$ and the internal loop momentum be $p$.
Then we can expand all the loop diagrams in $k$. Actually we would have
multiple external momenta $k_i$ but we can probably just let $k$ be the largest
one (in some particular reference frame perhaps) and expand in it. Only the
dimensional argument is important here. The diagram with zero or one external
leg is independent of the external momentum. Therefore that will diverge with
the one term there is.

The diagram with two legs will have one four dimensional integral in $p$ and
two powers of $p + k$ in the denominator. Therefore it still diverges with $D =
2$. Each differentiation with $k$ will lower the divergence in $p$ by one.
Therefore we need to include the $D = 2, 1, 0$ terms which means we have to
include $k^2$ terms.

In the diagram with three legs we already have three powers of $p + k$ in the
denominator, here we only need $k^1$ terms in order to have $D = 1, 0$ terms.

The diagram with four legs is only divergent with $D = 0$ already. There we do
not need to expand, we can just let the external momenta go to zero.

The propagator and the vertex do not have any integration really, so they are
fine although the superficial degree of divergence says otherwise. When adding
loop corrections, this might change.

\paragraph{Symmetry factors}

In all cases the symmetry factor $S$ is just one as we are not allowed to
switch the external legs. Otherwise the diagram with two legs would have $S =
2$.

\paragraph{Diverging $\phi^4$ amplitude}

The $\phi^4$ amplitude is shown in Figure~\ref{fig:loop-4} and has $D = 0$.
\begin{align*}
    \iup \mathcal M_4
    &=
    \int \frac{\dif^4 p}{[2 \piup]^4} \frac{\trdirac\del{
        \iup \fourslash p \mat\gamma^5
        \iup \fourslash p \mat\gamma^5
        \iup \fourslash p \mat\gamma^5
        \iup \fourslash p \mat\gamma^5
    }
}{[\four p^2 - m^2]^4}
\\
\intertext{%
    We can move all the $\mat\gamma^5$ to one spot. We need an even number of
    anticommutations and therefore do not have an overall sign change. The
    fifth Dirac matrix is its own inverse and therefore they all just cancel
    out. The imaginary units also cancel out. We are therefore left with:
}
&=
    \int \frac{\dif^4 p}{[2 \piup]^4} \frac{\trdirac\del{
        \fourslash p
        \fourslash p
        \fourslash p 
        \fourslash p 
    }
}{[\four p^2 - m^2]^4}
\intertext{%
    This is just a very fancy scalar product as we can just use $\fourslash p
    \fourslash p = \four p^2$ twice.
}
&= \int \frac{\dif^4 p}{[2 \piup]^4} \frac{\four p^4}{[\four p^2 - m^2]^4}
\end{align*}
This has eight powers of $\four p$ in the numerator as well as in the
denominator. We could use dimensional regularization here to compute the
amplitude.

\subsection{Counterterms}

\newcommand\phir{\phi_\text r}
\newcommand\psir{\psi_\text r}

The previous Lagrangian has been
\[
    \mathcal L
    = \frac 12 (\partial_\mu \phi)^2 - \frac12 m_0^2 \phi^2 - \bar\psi(\iup
    \slashed\partial - M_0) \psi - \iup g \bar\psi \gamma^5 \psi \phi \,.
\]
Now we add the quartic scalar interaction and have
\[
    \mathcal L
    = \frac 12 (\partial_\mu \phi)^2 - \frac12 m_0^2 \phi^2 -
    \frac{\lambda}{4!} \phi^4 - \bar\psi(\iup
    \slashed\partial - M_0) \psi - \iup g \bar\psi \gamma^5 \psi \phi \,.
\]

We define the renormalized fields with
\[
    \phi = \sqrt{Z_\phi} \phir
    \eqnsep
    \psi = \sqrt{Z_\psi} \psir
\]
and insert that into the Lagrangian. We obtain
\[
    \mathcal L
    = \frac 12 Z_\phi (\partial_\mu \phir)^2
    - \frac12 Z_\phi m_0^2 \phir^2
    - Z_\phi^2 \frac{\lambda_0}{4!} \phir^4
    - Z_\psi \bar\psir(\iup \slashed\partial - M_0) \psir
    - \iup g Z_\psi \sqrt{Z_\phi} \bar\psir \gamma^5 \psir \phir \,.
\]

From here we introduce the following counterterms:
\begin{gather*}
    \delta_\phi = Z_\phi - 1
    \eqnsep
    \delta_\psi = Z_\psi - 1
    \eqnsep
    \delta_m = Z_\phi m_0^2 - m^2
    \eqnsep
    \delta_M = Z_\psi M_0^2 - M^2
    \,, \\
    \delta_g = Z_\psi \sqrt{Z_\phi} g_0 - g
    \eqnsep
    \delta_\lambda = Z_\phi^2 \lambda_0 - \lambda \,.
\end{gather*}
This allows us to write the Lagrangian as
\begin{align*}
    \mathcal L
    &= \frac 12 (\partial_\mu \phir)^2
    - \frac12 m^2 \phir^2
    - \frac{\lambda}{4!} \phir^4
    - \bar\psir(\iup \slashed\partial - M) \psir
    - \iup g \bar\psir \gamma^5 \psir \phir
    \\
    &\quad
    + \frac 12 \delta_\phi (\partial_\mu \phir)^2
    - \frac12 \delta_m \phir^2
    - \frac{\delta_\lambda}{4!} \phir^4
    - \iup \delta_\psi \bar\psir \slashed\partial \psir
    + \delta_M \bar\psir \psir
    - \iup \delta_g \bar\psir \gamma^5 \psir \phir \,.
\end{align*}

The Feynman rules for the four counterterms are
\begin{align*}
    \vcenter{\hbox{\includegraphics{counter-phi}}}
    &= \iup \sbr{p^2 \delta_\phi - \delta_m} \,,
    &
    \vcenter{\hbox{\includegraphics{counter-psi}}}
    &= \iup \sbr{\fourslash p \delta_\psi - \delta_M} \,, \\ \\
    \vcenter{\hbox{\includegraphics{counter-quartic}}}
    &= - \iup \delta_\lambda \,,
    &
    \vcenter{\hbox{\includegraphics{counter-vertex}}}
    &= \delta_g \gamma^5 \,.
\end{align*}

\subsection{Regularization}

\paragraph{Regularization conditions}

First we need some regularization conditions. As we are working in the minimal
subtraction scheme, we must only subtract the poles. This means that the sum of
loop diagrams and the counterterm are supposed to be regular.

As usual, we require the propagators to have a pole at the physical mass and
only regular terms otherwise. This means for the fermion that
\begin{multline*}
    \vcenter{\hbox{\includegraphics{dirac-propagator}}}
    +
    \vcenter{\hbox{\includegraphics{fermion-self}}}
    +
    \vcenter{\hbox{\includegraphics{counter-psi}}}
    \\
    =
    \frac{\iup \fourslash p}{p^2 - M^2}
    + \text{(terms regular at $d = 4$)} \,.
\end{multline*}
Also the scalar propagator shall have the pole at the physical mass:
\begin{multline*}
    \vcenter{\hbox{\includegraphics{scalar-propagator}}}
    +
    \vcenter{\hbox{\includegraphics{loop-2}}}
    +
    \vcenter{\hbox{\includegraphics{tadpole-propagator}}}
    +
    \vcenter{\hbox{\includegraphics{counter-phi}}}
    \\
    =
    \frac{\iup}{p^2 - m^2}
    + \text{(terms regular at $d = 4$)} \,.
\end{multline*}
For the vertex we have
\[
    \vcenter{\hbox{\includegraphics{vertex}}}
    +
    \vcenter{\hbox{\includegraphics{vertex-loop}}}
    +
    \vcenter{\hbox{\includegraphics{counter-vertex}}}
    = \text{(terms regular at $d = 4$)} \,.
\]
And finally the quartic interaction is renormalized by
\begin{multline*}
    \vcenter{\hbox{\includegraphics{quartic}}}
    +
    \vcenter{\hbox{\includegraphics{loop-4}}}
    +
    \vcenter{\hbox{\includegraphics{s-channel}}}
    + \text{($t$-channel)}
    + \text{($u$-channel)}
    \\
    + \vcenter{\hbox{\includegraphics{counter-quartic}}}
    = \text{(terms regular at $d = 4$)} \,.
\end{multline*}

\paragraph{Regularization of the fermionic propagator}

We start with the computation of the fermion self-energy
\begin{align*}
    \iup \mathcal M_\text{FSE}
    &= \vcenter{\hbox{\includegraphics{fermion-self}}} \,. \\
    \intertext{%
        Using the Feynman rules we obtain
    }
    &= g^2 \bar u(p) \int \frac{\dif^4 k}{[2 \piup]^4} \frac{\iup}{k^2 - m^2}
    \gamma^5 \frac{\iup [\fourslash p - \fourslash k]}{[p-k]^2 - M^2} \gamma^5
    u(p) \,.
    \intertext{%
        We can just move the second $\gamma^5$ into the middle and obtain a
        sign switch while anticommuting it with $\fourslash p$. The two
        imaginary units cancel the negative sign. The square of $\gamma^5$ is
        the unit matrix. Therefore this simplifies to
    }
    &= g^2 \bar u(p) \int \frac{\dif^4 k}{[2 \piup]^4} \frac{1}{k^2 - m^2}
    \frac{\fourslash p - \fourslash k}{[p-k]^2 - M^2} u(p) \,.
    \intertext{%
        This integral diverges linearly. We can either expand in the external
        momentum $\four p$ or complete the square in the denominator. We choose
        to do the Feynman parameter route here. After all the magic we have
    }
    &= g^2 \bar u(p) \int \frac{\dif^4 k}{[2 \piup]^4}
    \int_0^i \dif x \, \dif y \delta(x + y - 1)
    \frac{x\fourslash p - \fourslash l}{\sbr{l^2 - \sbr{- xyp^2 + xm^2 +
    yM^2}}^2 + \iup \epsilon} u(p) \,.
    \intertext{%
        We can directly see that the integral with $\fourslash l$ will drop out
        due to symmetry. For the remainder we can perform dimensional
        regularization and have
    }
    &= g^2
    \int_0^i \dif x \, \dif y \delta(x + y - 1)
    x\fourslash p \frac{\iup}{[4 \piup]2} \Gammaup\del{2 - \frac d2} \sbr{\frac
    1 \Delta}^{2-2/d} \,.
    \intertext{%
        We are only interested in the residue at $\epsilon \to 0$ so we just
        end up with
    }
    &= g^2 \fourslash p \frac{\iup}{[4 \piup]^2} \frac2\epsilon
    + \text{(terms regular at $d = 4$)} \,.
\end{align*}

In the renormalised perturbation theory we already have the pole of the
propagator at the physical mass. That was the whole point. So we just need to
set the counterterm to the negative of the pole and we are done:
\[
    \vcenter{\hbox{\includegraphics{counter-psi}}}
    = \iup \sbr{\fourslash p \delta_\psi - \delta_M}
    = - g^2 \fourslash p \frac{\iup}{[4 \piup]^2} \frac2\epsilon \,.
\]
Our pole only contains the $\fourslash p$ term, therefore we have
\[
    \delta_\psi = - g^2 \frac{1}{[4 \piup]^2} \frac2\epsilon
    \eqnsep
    \delta_M = 0 \,.
\]
Somewhere along we way we have dropped the spinors. Using those, we could have
converted the $\fourslash p$ into an $M$ and gotten $\delta_\psi = 0$ and
$\delta_M \neq 0$. This is fishy and something must be wrong!

\end{document}

% vim: spell spelllang=en tw=79
