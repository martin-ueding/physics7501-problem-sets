\documentclass[11pt, english, fleqn, DIV=15, headinclude]{scrartcl}

\usepackage[bibatend]{../header}
\usepackage{../my-boxes}

\usepackage{lastpage}
\usepackage{multicol}
\usepackage{simplewick}
\usepackage{multicol}
\usepackage{adjustbox}
\usepackage{slashed}
\usepackage{subcaption}
\usepackage{cancel}
\usepackage{tikzsymbols}
\usepackage{placeins}

\newcommand\timeorder{\mathscr T}
\newcommand\normorder{\mathscr N}
\newcommand\eye{\mat 1_4}
\newcommand\fourslash[1]{\slashed{\four{#1}}}
\newcommand\T{\mathrm T}

\hypersetup{
    pdftitle=
}

\graphicspath{{build/}}

\newcounter{totalpoints}
\newcommand\punkte[1]{#1\addtocounter{totalpoints}{#1}}

\newcounter{problemset}
\setcounter{problemset}{11}

\subject{physics7501 -- Advanced Quantum Field Theory}
\ihead{physics7501 -- Problem Set \arabic{problemset}}

\title{Problem Set \arabic{problemset}}

\newcommand\thegroup{Tutor: Thorsten Schimannek}

\publishers{\thegroup}
\ofoot{\thegroup}

\author{
    Martin Ueding \\ \small{\href{mailto:mu@martin-ueding.de}{mu@martin-ueding.de}}
}
\ifoot{Martin Ueding}

\ohead{\rightmark}

\begin{document}

\maketitle

\vspace{3ex}

\begin{center}
    \begin{tabular}{rrr}
        Problem & Achieved points & Possible points \\
        \midrule
        \nameref{homework:1} & & \punkte{20} \\
        \midrule
        Total & & \arabic{totalpoints}
    \end{tabular}
\end{center}

\vspace{3ex}

\begin{center}
    \begin{small}
        This document consists of \pageref{LastPage} pages.
    \end{small}
\end{center}

\section{Renormalization of Yukawa theory in six dimensions}
\label{homework:1}

\newcommand\mdim M

\subsection{Feynman rules}

The Feynman rules are
\begin{align*}
    \vcenter{\hbox{\includegraphics{scalar-propagator}}}
    &= \frac{\iup}{\four p^2 - m^2}
    \\
    \vcenter{\hbox{\includegraphics{dirac-propagator}}}
    &= \frac{\iup \fourslash p}{\four p^2 - M^2}
    \\
    \vcenter{\hbox{\includegraphics{vertex}}}
    &= g \mat\gamma^5 \,.
\end{align*}

The mass dimension $\mdim$ of $g$ is zero, it is a dimensionless coupling
constant. The rules for the superficial degree of divergence are the same as in
QED as the boson propagator has the same mass dimension. Every scalar
propagator gives $\mdim = 2$, each fermion propagator gives $\mdim = 3$. Each
vertex gives $\mdim = -4$ due to the momentum conserving $\delta$-distribution.
The empty diagram starts with $\mdim = 4$ as as we do not want to count the
$\delta$-distribution that conserves the total momentum. Therefore we have
\[
    D = 4 + 3 P_\psi + 2 P_\phi - 4 V \,.
\]
This is not helpful as we want to express this in the external legs. A little
trying makes it clear that adding fermion loops inside of the diagram does not
change $D$ at all. Therefore it is only depended on the external legs, just as
with QED\@. We looked at a couple diagrams and made the Ansatz
\[
    D = a + b E_\psi + c E_\phi
\]
and solved the linear system of equations. We obtain
\[
    D = 4 - \frac32 E_\psi - E_\phi \,.
\]

\end{document}

% vim: spell spelllang=en tw=79
