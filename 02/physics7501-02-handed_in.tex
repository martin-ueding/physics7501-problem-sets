\documentclass[11pt, english, fleqn, DIV=15, headinclude]{scrartcl}

\usepackage[bibatend]{../header}
\usepackage{../my-boxes}

\usepackage{lastpage}
\usepackage{multicol}
\usepackage{simplewick}
\usepackage{slashed}
\usepackage{subcaption}
\usepackage{cancel}

\newcommand\timeorder{\mathscr T}
\newcommand\normorder{\mathscr N}
\newcommand\eye{\mat 1_4}
\newcommand\fourslash[1]{\slashed{\four{#1}}}

\hypersetup{
    pdftitle=
}

\graphicspath{{build/}}

\newcounter{totalpoints}
\newcommand\punkte[1]{#1\addtocounter{totalpoints}{#1}}

\newcounter{problemset}
\setcounter{problemset}{2}

\subject{physics7501 -- Advanced Quantum Field Theory}
\ihead{physics7501 -- Problem Set \arabic{problemset}}

\title{Problem Set \arabic{problemset}}

\newcommand\thegroup{Tutor: Thorsten Schimmanek}

\publishers{\thegroup}
\ofoot{\thegroup}

\author{
    Martin Ueding \\ \small{\href{mailto:mu@martin-ueding.de}{mu@martin-ueding.de}}
}
\ifoot{Martin Ueding}

\ohead{\rightmark}

\begin{document}

\maketitle

\vspace{3ex}

\begin{center}
    \begin{tabular}{rrr}
        Problem & Achieved points & Possible points \\
        \midrule
        \nameref{homework:1} & & \punkte{15} \\
        \midrule
        Total & & \arabic{totalpoints}
    \end{tabular}
\end{center}

\vspace{3ex}

\begin{center}
    \begin{small}
        This document consists of \pageref{LastPage} pages.
    \end{small}
\end{center}

We will indicate four-vectors with bold sans-serif letters (like
$\tens{abcdef}$) instead of using the underline. The underline looks rather
ugly. There should not be any confusion between four-vectors and higher-rank
tensors. This will still allow to discriminate between three-vectors ($\vec
q$,~$\vec k$) and four-vectors ($\four q$,~$\four k$), although the difference
is subtle now. Please tell if you have a preference.

\section{Vacuum polarization}
\label{homework:1}

\subsection{Form of propagator}

The photon propagator, we will call it $\pi(\four q)$ is corrected in first
order by the diagram in Figure~\ref{fig:loop}. All corrections are contained in
the corrected propagator $\Pi(\four q)$. The notation is similar to the vertex
correction where $\mat\gamma \mapsto \mat\Gamma$.

\begin{figure}[htbp]
    \centering
    \includegraphics{loop}
    \caption{%
        Vacuum polarization in first order
    }
    \label{fig:loop}
\end{figure}

The propagator is a rank-2 tensor and therefore has to transform
Lorentz-covariantly. This means it can only be a linear combination of elements
which are rank-2 tensors themselves. For this we have the metric tensor~$\tens
g$ and the product~$\four q \otimes \four q$. All this can multiply a function
which depends on a Lorentz scalar as this will be invariant. The only scalar
that we have is $\four q^2$ which is Lorentz invariant.

Therefore we can have something like
\[
    \Pi^{\mu\nu}(\four q) = \sbr{ [a \four q^2 + b] g^{\mu\nu} + c q^\mu q^\nu
    } \Pi(\four q^2) \,.
\]
The QED Ward identity then says that replacing the polarization vector of an
external photon with the photon's momentum must give zero when contracted. Here
it means that
\[
    q_\mu \Pi^{\mu\nu} = 0 \,,
\]
the relation with $\nu$ contracted directly follows from the symmetry. We only
have to check one of them.
\begin{align*}
    q_\mu \Pi^{\mu\nu}(\four q)
    &= q_\mu \sbr{ [a \four q^2 + b] g^{\mu\nu} + c q^\mu q^\nu } \Pi(\four
    q^2) \\
    &= \sbr{ [a \four q^2 + b] \, q_\mu g^{\mu\nu} + c q_\mu q^\mu q^\nu }
    \Pi(\four q^2) \\
    &= \sbr{ [a \four q^2 + b] \, q^\nu + c \four q^2 q^\nu } \Pi(\four q^2) \\
    &= [a \four q^2 + b + c \four q^2] \, q^\nu \Pi(\four q^2)
\end{align*}
From here the condition
\[
    [a + c] \four q^2 + b = 0
\]
follows. This can only be fulfilled for any $\four q$ if $a = -c$ and $b = 0$
are given. Therefore the propagator can only have a form as given on the
problem set. The multiple $a$ can be absorbed into the $\Pi(\four q^2)$
function to give Equation~(1) from the problem set.

\subsection{Chains}

In Figure~\ref{fig:split/pokemon} two diagrams are shown which are not simple
chains of the first order diagram. We would say that the full contribution
needs to take those type of diagrams into account.

\begin{figure}[htbp]
    \begin{subfigure}[c]{.5\linewidth}
        \centering
        \includegraphics{split}
        \caption{%
            The “split egg” diagram
        }
        \label{fig:split}
    \end{subfigure}
    \begin{subfigure}[c]{.5\linewidth}
        \centering
        \includegraphics{pokemon}
        \caption{%
            The “Pokéball” diagram
        }
        \label{fig:pokemon}
    \end{subfigure}
    \caption{%
        Correction diagrams which are not chains of the first-order diagram.
    }
    \label{fig:split/pokemon}
\end{figure}

The problem statement asks for \emph{one particle irreducible} diagrams. Then
those can be chained together. This makes sense as this prevent overcounting of
diagrams. The diagrams shown on the problem set are just chains of the first
order diagram. There are more complicated 1-PI diagrams that needs to be
included as well.

When we denote the 1-PI diagrams with a gray-filled blob, the equation would
like the following:
\begin{align*}
    \vcenter{\hbox{\includegraphics{blob}}}
    &=
    \vcenter{\hbox{\includegraphics{blob0}}}
    +
    \vcenter{\hbox{\includegraphics{blob1}}}
    \\&\quad
    +
    \vcenter{\hbox{\includegraphics{blob2}}}
    + \ldots \\
    &= \sum_{n=0}^\infty \sbr{
    \vcenter{\hbox{\includegraphics{blob1}}}
    }^n \,.
\end{align*}

\subsection{General form}

The form given on the problem set will become the usual photon propagator when
$\Pi \to 0$. This suggests that $\Pi$ is a correction to the propagator and not
the \emph{corrected} propagator. \Textcite[219]{Peskin/QFT/1995} show a
similar calculation with the electron self-energy corrections. There the
expression $\Sigma(\four p)$ also does not contain the propagators. The
expression is sandwiched with propagators to give the blob expression.

In the general form we have some sort of power series in $\Pi^{\mu\nu}$.
Between the gray blobs there is a photon propagator which ties the left
outgoing and the right incoming photon together. Let us denote the photon
propagator with $\mat S$. The whole contribution should be
\[
    \vcenter{\hbox{\includegraphics{blob}}}
    = \mat S \sum_{n=0}^\infty \sbr{\iup \mat\Pi \mat S}^n \,.
\]
where $\mat \Pi$ is the matrix representation of the propagator. Then the power
is just the power of matrices. The contraction happens by the metric tensor
usually present in the photon propagator.

There is a nice trick shown by the authors. The product $\iup \mat\Pi \mat S$
actually is an idempotent projector:
\begin{align*}
    \iup \mat\Pi \mat S
    &= \iup [ \four q^2 \eye - \four q \otimes \four q] \Pi(\four q^2)
    \frac{- \iup \eye}{\four q^2} \\
    \intertext{%
        One can just cancel the $\four q^2$ and obtain
    }
    &= \sbr{ \eye - \frac{\four q \otimes \four q}{\four q^2}} \Pi(\four q^2)
    \,,
    \intertext{%
        which we define to be
    }
    &= \mat P \Pi(\four q^2) \,.
\end{align*}
This is a projection operator to the subspace where the Ward identity holds. We
call this operator $\mat P$ and use the fact that it is idempotent.

Now we continue evaluating the blob.
\begin{align*}
    \mat S \sum_{n=0}^\infty \sbr{\iup \mat\Pi \mat S}^n
    &= \mat S \sum_{n=0}^\infty \sbr{\mat P \Pi(\four q^2)}^n
    \intertext{%
        We split off the $n = 0$ summand. It does not contain a projector and
        needs to be treated differently.
    }
    &= \mat S + \mat S \sum_{n=1}^\infty [\mat P \Pi(\four q^2)]^n
    \intertext{%
        The projection operator is idempotent and can be pulled out of the
        power and even in front of the sum.
    }
    &= \mat S + \mat S \mat P \sum_{n=1}^\infty \Pi(\four q^2)^n
    \intertext{%
        The sum now is the geometric series where we can just use the solution
        formula. Of course we assume that $\lim{n \to \infty} \Pi(\four q^2)^n
        = 0$.
    }
    &= \mat S + \mat S \mat P \frac{1}{1 - \Pi(\four q^2)}
    \intertext{%
        From here it is easier to introduce indices again.
    }
    \leadsto \quad [\ldots]_{\mu\nu} &=
    \frac{-\iup g_{\mu\nu}}{\four q^2}
    +
    \frac{-\iup g_{\mu\alpha}}{\four q^2}
    \sbr{\delta^\alpha_\nu - \frac{q^\alpha q_\nu}{\four q^2}}
    \frac{1}{1 - \Pi(\four q^2)}
    \intertext{%
        We first contract the indices to reduce the clutter.
    }
    &=
    \frac{-\iup g_{\mu\nu}}{\four q^2}
    +
    \frac{-\iup}{\four q^2}
    \sbr{g_{\mu\nu} - \frac{q_\mu q_\nu}{\four q^2}}
    \frac{1}{1 - \Pi(\four q^2)}
\end{align*}
Now \textcite[246]{Peskin/QFT/1995} argue that every term proportional to
$q_\mu$ or $q_\nu$ must vanish due to the Ward identity. To get rid of the
first summand they write $g_{\mu\nu}$ as $q_\mu q_\mu / \four q^2$. Then they
drop the first summand (last summand in their version (7.74)) and drop the
second term in the square bracket. They end up with
\[
    \frac{-\iup g_{\mu\nu}}{\four q^2} \frac{1}{1 - \Pi(\four q^2)}
\]
which is also the result desired here.

\begin{question}
    We are not really show why those terms can be dropped. Using the same
    argument, the remaining metric tensor could be written as $q_\mu q_\mu /
    \four q^2$ and dropped as well. There must be an additional argument here,
    what is it?
\end{question}

\subsection{Physical charge}

The first fraction does not change, so we can just leave it be. The fraction in
the parentheses has to be expanded. We start by replacing the bare charge with
the physical charge.
\begin{align*}
    \frac{e_0^2}{1 - \Pi(\four q^2)}
    &= \frac{e^2}{Z_3[1 - \Pi(\four q^2)]} \\
    &= \frac{e^2}{[1 - \Pi(0)][1 - \Pi(\four q^2)]} \\
    &= \frac{e^2}{1 - \Pi(0) - \Pi(\four q^2) + \Pi(0) \Pi(\four q^2)} \\
    \intertext{%
        And then we can perform the approximation.
    }
    &= \frac{e^2}{1 - [\Pi_2(0) + \Pi_2(\four q^2)] + \Pi_2(0) \Pi_2(\four q^2) + \mathrm O(\alpha^2)} \\
    \intertext{%
        The products of $\Pi$ should also be of order $\alpha^2$ as this term
        must be at least linear in $\alpha$.
    }
    &= \frac{e^2}{1 - [\Pi_2(0) + \Pi_2(\four q^2)] + \mathrm O(\alpha^2)}
    \intertext{%
        Except for the sign, this looks like a step into the right direction.
        We should expand the fraction such that the higher order terms are
        added next to the fraction.
    }
    &= \frac{e^2}{1 - [\Pi_2(0) + \Pi_2(\four q^2)]} + \mathrm O(\alpha^2)
\end{align*}

\subsection{Invariant amplitude}

We hope that instead of $\iup \Pi^{\mu\nu}(\four q)$ we are asked to compute
$\iup \Pi^{\mu\nu}_2(\four q)$ (added “$_2$”) which is just the diagram shown in
Figure~\ref{fig:loop}. Otherwise that would be a result to all orders of
$\alpha$ which is something we probably cannot do, especially not in a homework
problem.

Looking at Figure~\ref{fig:loop} we can assemble the terms. The catch is the
negative sign from the fermion loop. This is due to the contractions of all the
fermion fields with each other. The contractions can be disentangled by
anticommuting them such that they form a cyclic trace. An odd number of
anticommutations is needed such that an overall minus sign is left. Together
with the other terms we have
\begin{align*}
    \iup \Pi^{\mu\nu}_2
    &= - \int \frac{\dif^4 k}{[2 \piup]^4}
    \tr\del{
        [- \iup e] \, \mat\gamma^\mu
        \frac{\iup [\fourslash q + \fourslash k + m]}{[\four q + \four k]^2 - m^2}
        [- \iup e] \, \mat\gamma^\nu
        \frac{\iup [\fourslash k + m]}{\four k^2 - m^2}
    }
    \,.
    \intertext{%
        We can drop all the imaginary units, two of the negative sign and pull
        the charges up front. The trace only acts on the numerator, so we can
        shrink that to a single fraction now.
    }
    &= - e^2 \int \frac{\dif^4 k}{[2 \piup]^4}
    \frac{
        \tr\del{
            \mat\gamma^\mu
            [\fourslash q + \fourslash k + m]
            \mat\gamma^\nu
        [\fourslash k + m]}
    }
    {\sbr{[\four q + \four k]^2 - m^2} \sbr{\four k^2 - m^2}}
    \intertext{%
        For the trace we use an identity that we have shown last semester.
    }
    &= - 4 e^2 \int \frac{\dif^4 k}{[2 \piup]^4}
    \frac{
        [q^\mu + k^\mu]
        k^\nu
        +
        k^\mu
        [q^\nu + k^\nu]
        -
        g^{\mu\nu}
        \sbr{ [\four q + \four k] \cdot \four k - m^2}
    }
    {\sbr{[\four q + \four k]^2 - m^2} \sbr{\four k^2 - m^2}}
\end{align*}
Then we are already done here.

\subsection{Denominator and shift}

The denominator will attain a form of
\[
    \int_0^1 \dif x \dif y \, \deltaup(x + y - 1) \frac 1{D^2}
\]
where $D$ is the denominator with Feynman parameters to be computed.
\begin{align*}
    D
    &= x \sbr{[\four q + \four k]^2 - m^2} + y [\four k^2 - m^2] \\
    &= x [\four q^2 + 2 \four q \cdot \four k + \four k^2 - m^2]
    + y [\four k^2 - m^2] \\
    \intertext{%
        Since $x + y = 1$ we can join some of the terms which appear in both
        brackets.
    }
    &= x [\four q^2 + 2 \four q \cdot \four k] + \four k^2 - m^2 \\
    \intertext{%
        We reorder the terms for the next step.
    }
    &= \four k^2 + 2 x \four q \cdot \four k + x \four q^2 - m^2 \\
    \intertext{%
        Here we complete the square in $\four k$ and define $\four l := \four k
        + x \four q$.
    }
    &= [\four k + x \four q]^2 - x^2 \four q^2 + x \four q^2 - m^2 \\
    &= \four l^2 + x[1-x] \four q^2 - m^2
\end{align*}
With $\Delta := - x[1-x] \four q^2 + m^2$ we can write the denominator as $D =
\four l^2 - \Delta$. All the propagators contain an $\iup \epsilon$ which we
only carry to first order. Therefore the denominator also has a $+ \iup
\epsilon$ term which we have omitted here for the calculation.

The shift of variables needs to be done in the numerator as well. There we had
so far:
\begin{align*}
    N
    &= [q^\mu + k^\mu] k^\nu + k^\mu [q^\nu + k^\nu]
    - g^{\mu\nu} \sbr{ [\four q + \four k] \cdot \four k - m^2} \,.
    \intertext{%
        The calculation is straightforward, we replace $\four k$ with $\four l
        - x \four q$.
    }
    &= [[1-x] q^\mu + l^\mu] [l^\nu - x q^\nu] + [l^\mu - x q^\mu] [[1-x] q^\nu + l^\nu]
    - g^{\mu\nu} \sbr{ [[1-x] \four q + \four l] \cdot [\four l - x \four q] - m^2}
    \intertext{%
        Then we factor out all the terms.
    }
    &= l^\mu l^\nu - x[1-x] q^\mu q^\nu - \bcancel{x l^\mu q^\nu} +
    \bcancel{[1-x] q^\mu l^\nu} + l^\mu l^\nu - x[1-x] q^\mu q^\nu +
    \bcancel{[1-x] q^\mu q^\nu} - \bcancel{x q^\mu l^\nu}
    \\&\quad
    - g^{\mu\nu} \sbr{ \four l^2 - x[1-x] \four q^2 + \bcancel{[1-x] \four q
    \cdot \four l} - \bcancel{x \four l \cdot \four q} - m^2}
    \intertext{%
        Terms linear in $\four l$ drop out due to the symmetric integration
        domain. We have canceled those already. The cancel slash will aways go
        downwards such that it is distinct from the Feynman slash.
    }
    &= 2 l^\mu l^\nu - 2 x[1-x] q^\mu q^\nu
    - g^{\mu\nu} \sbr{\four l^2 - x[1-x] \four q^2 - m^2}
    \intertext{%
        This can be massaged into the form given on the problem set. Then this
        is:
    }
    &= 2 l^\mu l^\nu - g^{\mu\nu} \four l^2 - 2 x[1-x] q^\mu q^\nu
    + g^{\mu\nu} \sbr{m^2 + x[1-x] \four q^2} \,.
\end{align*}
This result can be combined with the denominator that we have computed and
obtain the full result. The Feynman parameter $y$ is not present in any of the
terms any more, therefore we can just drop the integral and the
Dirac-distribution:
\[
    \iup \Pi^{\mu\nu}_2
    = - 4 e^2 \int \frac{\dif^4 l}{[2 \piup]^4}
    \int_0^1 \dif x \,
    \frac
    {
        2 l^\mu l^\nu - g^{\mu\nu} \four l^2 - 2 x[1-x] q^\mu q^\nu
        + g^{\mu\nu} \sbr{m^2 + x[1-x] \four q^2}
    }
    {\sbr{\four l^2 - \Delta + \iup \epsilon}^2} \,.
\]
This is exactly Equation~(4) from the problem set which is nice.

\subsection{Wick rotation}

\newcommand\lE{l_\text E}
\newcommand\fourlE{\four l_\text E}

The dimensional regularization works with generalized spherical coordinates.
Therefore the signature of the metric must the all positive or negative. No
dimension must be singled out, they need to obey an $\SO(4)$ symmetry and not
an $\SO(1, 3)$ symmetry. This can be achieved by a Wick rotation which maps
$l^0 \mapsto \iup \lE^0$. This simplifies the squares of four-vectors:
\[
    \four l^2 = [l^0]^2 - \vec l^2
    = - [\iup l^0]^2 - \vec l^2
    = - [\lE^0]^2 - \vec l^2
    = - \fourlE^2 \,.
\]

In the denominator we go from $[\four l^2 - \Delta]^2$ to $[\fourlE^2 +
\Delta]^2$ and we absorb the minus sign in the square.
In the numerator we change
\[
    2 l^\mu l^\nu = \frac 2d g^{\mu\nu} \four l^2
    = - \frac 2d g^{\mu\nu} \fourlE^2 \,.
\]
The second term, involving $\four l^2$, is replaced by $- \fourlE^2$. We then
have the following integral with dimension~$d$:
\[
    \iup \Pi^{\mu\nu}_2
    = - 4 \iup e^2 \int \frac{\dif^d \lE}{[2 \piup]^d}
    \int_0^1 \dif x \,
    \frac
    {
        \sbr{1 - \frac 2d } g^{\mu\nu} \fourlE^2 - 2 x[1-x] q^\mu q^\nu
        + g^{\mu\nu} \sbr{m^2 + x[1-x] \four q^2}
    }
    {\sbr{\fourlE^2 + \Delta}^2} \,.
\]
The additional imaginary unit up front originates in the change of integration
variables.

\subsection{Evaluation of the integral}

\paragraph{Integral formula}

The integral is parametrized with arbitrary real dimension and we can start to
evaluate it. We expand the integral into multiple factors depending on their
dependence on $\lE$.
\begin{align*}
    \iup \Pi^{\mu\nu}_2
    &= - 4 \iup e^2
    \int_0^1 \dif x \,
    \int \frac{\dif^d \lE}{[2 \piup]^d}
    \sbr{
        \sbr{1 - \frac 2d} g^{\mu\nu}
        \frac{\fourlE^2}{\sbr{\fourlE^2 + \Delta}^2}
        +
        \frac{
            - 2 x[1-x] q^\mu q^\nu
            + g^{\mu\nu} \sbr{m^2 + x[1-x] \four q^2}
        }
        {\sbr{\fourlE^2 + \Delta}^2}
    }
    \intertext{%
        And then we can isolate the integrals to match the Equation~(5) and (6)
        on the problem set.
    }
    &=
    - 4 \iup e^2
    \int_0^1 \dif x \,
    \sbr{1 - \frac 2d} g^{\mu\nu}
    \int \frac{\dif^d \lE}{[2 \piup]^d}
    \frac{\fourlE^2}{\sbr{\fourlE^2 + \Delta}^2}
    \\&\quad
    - 4 \iup e^2
    \int_0^1 \dif x \,
    \sbr{
        - 2 x[1-x] q^\mu q^\nu
        + g^{\mu\nu} \sbr{m^2 + x[1-x] \four q^2}
    }
    \int \frac{\dif^d \lE}{[2 \piup]^d}
    \frac{1}{\sbr{\fourlE^2 + \Delta}^2}
    \intertext{%
        We just apply the formulas for the integrals.
    }
    &=
    - 4 \iup e^2
    \int_0^1 \dif x \,
    \sbr{1 - \frac 2d} g^{\mu\nu}
    \frac{1}{[4\piup]^{d/2}} \frac d2
    \frac{\Gammaup\del{2 - \frac d2 - 1}}{\Gammaup(2)}
    \sbr{\frac 1\Delta}^{2 - d/2 - 1}
    \\&\quad
    - 4 \iup e^2
    \int_0^1 \dif x \,
    \sbr{
        - 2 x[1-x] q^\mu q^\nu
        + g^{\mu\nu} \sbr{m^2 + x[1-x] \four q^2}
    }
    \frac{1}{[4\piup]^{d/2}} \frac{\Gammaup\del{2 - \frac d2}}{\Gammaup(2)}
    \sbr{\frac 1\Delta}^{2 - d/2}
    \intertext{%
        In the first term, we can combine the square bracket and the $d/2$
        fraction.
    }
    &=
    - 4 \iup e^2
    \int_0^1 \dif x \,
    \sbr{\frac d2 - 1} g^{\mu\nu}
    \frac{1}{[4\piup]^{d/2}}
    \frac{\Gammaup\del{1 - \frac d2}}{\Gammaup(2)}
    \sbr{\frac 1\Delta}^{2 - d/2 - 1}
    \\&\quad
    - 4 \iup e^2
    \int_0^1 \dif x \,
    \sbr{
        - 2 x[1-x] q^\mu q^\nu
        + g^{\mu\nu} \sbr{m^2 + x[1-x] \four q^2}
    }
    \frac{1}{[4\piup]^{d/2}} \frac{\Gammaup\del{2 - \frac d2}}{\Gammaup(2)}
    \sbr{\frac 1\Delta}^{2 - d/2}
    \intertext{%
        Then we can extract the minus sign. We move the resulting square
        bracket into the $\Gammaup$-function and increase its argument by one.
    }
    &=
    + 4 \iup e^2
    \int_0^1 \dif x \, g^{\mu\nu}
    \frac{1}{[4\piup]^{d/2}}
    \frac{\Gammaup\del{2 - \frac d2}}{\Gammaup(2)}
    \sbr{\frac 1\Delta}^{2 - d/2 - 1}
    \\&\quad
    - 4 \iup e^2
    \int_0^1 \dif x \,
    \sbr{
        - 2 x[1-x] q^\mu q^\nu
        + g^{\mu\nu} \sbr{m^2 + x[1-x] \four q^2}
    }
    \frac{1}{[4\piup]^{d/2}} \frac{\Gammaup\del{2 - \frac d2}}{\Gammaup(2)}
    \sbr{\frac 1\Delta}^{2 - d/2}
    \intertext{%
        After that we factor out the common terms. The first term only consists
        of $- g^{\mu\nu} \Delta$ at this point. We merge that into the term
        proportional to the metric tensor from the second summand.
    }
    &=
    - 4 \iup e^2
    \int_0^1 \dif x \,
    \frac{1}{[4\piup]^{d/2}}
    \frac{\Gammaup\del{2 - \frac d2}}{\Delta^{2-d/2}}
    \sbr{
        - 2 x[1-x] q^\mu q^\nu
        + g^{\mu\nu} \sbr{- \Delta + m^2 + x[1-x] \four q^2}
    }
    \intertext{%
        The $\Delta$ was defined as $-x[1-x] \four q^2 + m^2$. We insert that
        and see that the $m^2$ term cancels. The other term in the square
        bracket doubles. We are left with
    }
    &=
    \underbrace{
        - 8 \iup e^2
        \int_0^1 \dif x \,
        \frac{1}{[4\piup]^{d/2}}
        \frac{\Gammaup\del{2 - \frac d2}}{\Delta^{2-d/2}}
    x [1-x]}_{\iup \Pi_2(\four q^2)} [g^{\mu\nu} \four q^2 - q^\mu q^\nu ] \,.
\end{align*}
The very last square bracket is the part that was not hidden in
$\Pi_2(\four q^2)$. Therefore we can identity the remainder with
exactly this term. This result matches the one from the book, so we probably
are on the right track here.

\paragraph{Separation of singularity}

Our calculation continues with the separation of the singularity into a
manageable expression that hopefully cancels when we compute an observable.
Therefore we expand this around $d = 4$. We introduce $\epsilon := 4 - d$ as
the positive difference of dimensionality. Then we can expand around $\epsilon
= 0$. The $\Gammaup$-function can be expanded around the pole like so:
\parencite[(7.83)]{Peskin/QFT/1995}
\[
    \Gammaup\del{2 - \frac d2} = \Gammaup\del{\frac \epsilon 2}
    = \frac 2\epsilon - \gamma + \mathrm O(\epsilon) \,.
\]
Before we can proceed we should also derive another approximation, the one for
powers:
\[
    x^{\epsilon/2}
    = \exp\del{\ln(x) \frac\epsilon2}
    = 1 + \ln(x) \frac\epsilon2 + \mathrm O(\epsilon^2) \,.
\]
We will need that twice now.

So far we have
\begin{align*}
    \Pi_2(\four q^2)
    &= - 8 e^2
    \int_0^1 \dif x \,
    \frac{[4 \piup]^{\epsilon/2}}{[4\piup]^{2}}
    \frac{\Gammaup\del{\frac \epsilon2}}{\Delta^{\epsilon/2}}
    x [1-x] \,.
    \intertext{%
        Using the approximations we obtain
    }
    &= - 8 e^2
    \int_0^1 \dif x \,
    \frac{1 + \ln(4 \piup) + \mathrm O(\epsilon^2)}{[4\piup]^2}
    \sbr{\frac 2\epsilon - \gamma + \mathrm O(\epsilon)}
    \sbr{1 - \ln(\Delta) + \mathrm O(\epsilon^2)}
    x [1-x] \,.
    \intertext{%
        We can ignore all terms of order $\epsilon$ here as they will vanish in
        the limit $\epsilon\to0$ anyway. Therefore there are only a few
        interesting factors.
    }
    &= - 8 e^2
    \int_0^1 \dif x \,
    \frac{x [1-x]}{16 \piup^2}
    \sbr{\frac2\epsilon + \ln(4 \piup) - \ln(\Delta) - \gamma + \mathrm
    O(\epsilon)}
    \intertext{%
        And now we can cancel some constant factors and introduce the fine
        structure constant $\alpha = \frac{e^2}{4 \piup}$.
    }
    &= - \frac{2 \alpha}{\piup}
    \int_0^1 \dif x \, x [1-x]
    \sbr{\frac2\epsilon + \ln(4 \piup) - \ln(\Delta) - \gamma + \mathrm
    O(\epsilon)}
\end{align*}
This is the expression given by \textcite[(7.90)]{Peskin/QFT/1995} and should
be correct.

\paragraph{Difference expression}

In the book it is explained that one cannot observe this quantity directly;
that is a good thing as it is divergent. Only the scaling with respect to
$\four q^2$ can be observed, therefore we have to take the difference to the
limit of no exchanged momentum ($\four q^2 \to 0$).

\begin{question}
    By the way, isn't $\four q$ the momentum of a photon and should square to
    zero for all physical photons? From the Feynman diagram in
    Figure~\ref{fig:loop} it looks like the photon with momentum~$\four q$ is a
    real one. Why isn't $\four q^2 = 0$ the whole time anyway?
\end{question}

So we start to derive this difference.
\begin{align*}
    \Pi_2(\four q^2) - \Pi_2(0)
    &= - \frac{2 \alpha}{\piup}
    \int_0^1 \dif x \, x [1-x]
    \sbr{- \ln(\Delta(\four q^2)) + \ln(\Delta(0)) + \mathrm O(\epsilon)}
    \intertext{%
        We combine the logarithms and insert the explicit expression for
        $\Delta$.
    }
    &= - \frac{2 \alpha}{\piup}
    \int_0^1 \dif x \, x [1-x]
    \sbr{- \ln\del{\frac{m^2}{-x[1-x] q^2 + m^2}} + \mathrm O(\epsilon)}
    \intertext{%
        And then we are already done. The $\epsilon$ singularity canceled. We
        can perform the limit to drop the higher order terms.
    }
    &\xrightarrow[\epsilon\to0]{}
    \frac{2 \alpha}{\piup}
    \int_0^1 \dif x \, x [1-x]
    \ln\del{\frac{m^2}{-x[1-x] q^2 + m^2}}
\end{align*}
This result is independent and sensible in the $\epsilon = 0$ case. This
expression matches the one given by \textcite[(7.91)]{Peskin/QFT/1995} and the
one given on the problem set as Equation~(9).

\end{document}

% vim: spell spelllang=en tw=79
