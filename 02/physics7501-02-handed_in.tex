\documentclass[11pt, english, fleqn, DIV=15, headinclude]{scrartcl}

\usepackage[bibatend]{../header}

\usepackage{lastpage}
\usepackage{multicol}
\usepackage{simplewick}
\usepackage{slashed}
\usepackage{subcaption}

\newcommand\timeorder{\mathscr T}
\newcommand\normorder{\mathscr N}
\newcommand\eye{\mat 1_4}
\newcommand\fourslash[1]{\slashed{\four{#1}}}

\hypersetup{
    pdftitle=
}

\graphicspath{{build/}}

\newcounter{totalpoints}
\newcommand\punkte[1]{#1\addtocounter{totalpoints}{#1}}

\newcounter{problemset}
\setcounter{problemset}{2}

\subject{physics7501 -- Advanced Quantum Field Theory}
\ihead{physics7501 -- Problem Set \arabic{problemset}}

\title{Problem Set \arabic{problemset}}

\newcommand\thegroup{Tutor: Thorsten Schimmanek}

\publishers{\thegroup}
\ofoot{\thegroup}

\author{
    Martin Ueding \\ \small{\href{mailto:mu@martin-ueding.de}{mu@martin-ueding.de}}
}
\ifoot{Martin Ueding}

\ohead{\rightmark}

\begin{document}

\maketitle

\vspace{3ex}

\begin{center}
    \begin{tabular}{rrr}
        Problem & Achieved points & Possible points \\
        \midrule
        \nameref{homework:1} & & \punkte{15} \\
        \midrule
        Total & & \arabic{totalpoints}
    \end{tabular}
\end{center}

\vspace{3ex}

\begin{center}
    \begin{small}
        This document consists of \pageref{LastPage} pages.
    \end{small}
\end{center}

We will indicate four-vectors with bold sans-serif letters (like
$\tens{abcdef}$) instead of using the underline. The underline looks rather
ugly. There should not be any confusion between four-vectors and higher-rank
tensors. This will still allow to discriminate between three-vectors ($\vec
q$,~$\vec k$) and four-vectors ($\four q$,~$\four k$), although the difference
is subtle now. Please tell if you have a preference.

\section{Vacuum polarization}
\label{homework:1}

\subsection{Form of propagator}

The photon propagator, we will call it $\pi(\four q)$ is corrected in first
order by the diagram in Figure~\ref{fig:loop}. All corrections are contained in
the corrected propagator $\Pi(\four q)$. The notation is similar to the vertex
correction where $\mat\gamma \mapsto \mat\Gamma$.

\begin{figure}[htbp]
    \centering
    \includegraphics{loop}
    \caption{%
        Vacuum polarization in first order
    }
    \label{fig:loop}
\end{figure}

The propagator is a rank-2 tensor and therefore has to transform
Lorentz-covariantly. This means it can only be a linear combination of elements
which are rank-2 tensors themselves. For this we have the metric tensor~$\tens
g$ and the product~$\four q \otimes \four q$. All this can multiply a function
which depends on a Lorentz scalar as this will be invariant. The only scalar
that we have is $\four q^2$ which is Lorentz invariant.

Therefore we can have something like
\[
    \Pi^{\mu\nu}(\four q) = \sbr{ [a \four q^2 + b] g^{\mu\nu} + c q^\mu q^\nu
    } \Pi(\four q^2) \,.
\]
The QED Ward identity then says that replacing the polarization vector of an
external photon with the photon's momentum must give zero when contracted. Here
it means that
\[
    q_\mu \Pi^{\mu\nu} = 0 \,,
\]
the relation with $\nu$ contracted directly follows from the symmetry. We only
have to check one of them.
\begin{align*}
    q_\mu \Pi^{\mu\nu}(\four q)
    &= q_\mu \sbr{ [a \four q^2 + b] g^{\mu\nu} + c q^\mu q^\nu } \Pi(\four
    q^2) \\
    &= \sbr{ [a \four q^2 + b] \, q_\mu g^{\mu\nu} + c q_\mu q^\mu q^\nu }
    \Pi(\four q^2) \\
    &= \sbr{ [a \four q^2 + b] \, q^\nu + c \four q^2 q^\nu } \Pi(\four q^2) \\
    &= [a \four q^2 + b + c \four q^2] \, q^\nu \Pi(\four q^2)
\end{align*}
From here the condition
\[
    [a + c] \four q^2 + b = 0
\]
follows. This can only be fulfilled for any $\four q$ if $a = -c$ and $b = 0$
are given. Therefore the propagator can only have a form as given on the
problem set. The multiple $a$ can be absorbed into the $\Pi(\four q^2)$
function to give Equation~(1) from the problem set.

\subsection{Chains}

In Figure~\ref{fig:split/pokemon} two diagrams are shown which are not simple
chains of the first order diagram. We would say that the full contribution
needs to take those type of diagrams into account.

\begin{figure}[htbp]
    \begin{subfigure}[c]{.5\linewidth}
        \centering
        \includegraphics{split}
        \caption{%
            The “split egg” diagram
        }
        \label{fig:split}
    \end{subfigure}
    \begin{subfigure}[c]{.5\linewidth}
        \centering
        \includegraphics{pokemon}
        \caption{%
            The “Pokéball” diagram
        }
        \label{fig:pokemon}
    \end{subfigure}
    \caption{%
        Correction diagrams which are not chains of the first-order diagram.
    }
    \label{fig:split/pokemon}
\end{figure}

The problem statement asks for \emph{one particle irreducible} diagrams. Then
those can be chained together. This makes sense as this prevent overcounting of
diagrams. The diagrams shown on the problem set are just chains of the first
order diagram. There are more complicated 1-PI diagrams that needs to be
included as well.

When we denote the 1-PI diagrams with a gray-filled blob, the equation would
like the following:
\begin{align*}
    \vcenter{\hbox{\includegraphics{blob}}}
    &=
    \vcenter{\hbox{\includegraphics{blob0}}}
    +
    \vcenter{\hbox{\includegraphics{blob1}}}
    \\&\quad
    +
    \vcenter{\hbox{\includegraphics{blob2}}}
    + \ldots \\
    &= \sum_{n=0}^\infty \sbr{
    \vcenter{\hbox{\includegraphics{blob1}}}
    }^n \,.
\end{align*}

\subsection{General form}

The form given on the problem set will become the usual photon propagator when
$\Pi \to 0$. This suggests that $\Pi$ is a correction to the propagator and not
the \emph{corrected} propagator. \Textcite[219]{Peskin/QFT/1995} show a
similar calculation with the electron self-energy corrections. There the
expression $\Sigma(\four p)$ also does not contain the propagators. The
expression is sandwiched with propagators to give the blob expression.

In the general form we have some sort of power series in $\Pi^{\mu\nu}$.
Between the gray blobs there is a photon propagator which ties the left
outgoing and the right incoming photon together. Let us denote the photon
propagator with $\mat S$. The whole contribution should be
\[
    \vcenter{\hbox{\includegraphics{blob}}}
    = \mat S
    \sum_{n=0}^\infty \sbr{\mat\Pi \mat S}^n \,.
\]
where $\mat \Pi$ is the matrix representation of the propagator. Then the power
is just the power of matrices. The contraction happens by the metric tensor
usually present in the photon propagator.

The solution given on the problem set rather looks like it originated from an
expression like
\[
    \mat S \sum_{n = 0}^\infty \mat\Pi^n \,.
\]
Then the geometric series would give the result
\[
    \mat S \frac{1}{1 - \Pi(\four q^2)}
    =
    \frac{-\iup g_{\mu\nu}}{\four q^2} \frac{1}{1 - \Pi(\four q^2)}
    \,.
\]

The additional photon propagator that we have between the blobs will tie the
$\Pi^{\alpha\beta}$ tensors together. There is an additional $1/\four q^2$
contained in the denominator which does not seem to appear in the desired
result. If it was included in the geometric series, it would lead to a pole at
$\four q^2 \to 0$ of infinite order.

We are not sure how to obtain this exact result.

\subsection{Physical charge}

The first fraction does not change, so we can just leave it be. The fraction in
the parentheses has to be expanded. We start by replacing the bare charge with
the physical charge.
\begin{align*}
    \frac{e_0^2}{1 - \Pi(\four q^2)}
    &= \frac{e^2}{Z_3[1 - \Pi(\four q^2)]} \\
    &= \frac{e^2}{[1 - \Pi(0)][1 - \Pi(\four q^2)]} \\
    &= \frac{e^2}{1 - \Pi(0) - \Pi(\four q^2) + \Pi(0) \Pi(\four q^2)} \\
    \intertext{%
        And then we can perform the approximation.
    }
    &= \frac{e^2}{1 - [\Pi_2(0) + \Pi_2(\four q^2)] + \Pi_2(0) \Pi_2(\four q^2) + \mathrm O(\alpha^2)} \\
    \intertext{%
        The products of $\Pi$ should also be of order $\alpha^2$ as this term
        must be at least linear in $\alpha$.
    }
    &= \frac{e^2}{1 - [\Pi_2(0) + \Pi_2(\four q^2)] + \mathrm O(\alpha^2)}
    \intertext{%
        Except for the sign, this looks like a step into the right direction.
        We should expand the fraction such that the higher order terms are
        added next to the fraction.
    }
    &= \frac{e^2}{1 - [\Pi_2(0) + \Pi_2(\four q^2)]} + \mathrm O(\alpha^2)
\end{align*}

\end{document}

% vim: spell spelllang=en tw=79
