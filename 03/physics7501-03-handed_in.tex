\documentclass[11pt, english, fleqn, DIV=15, headinclude]{scrartcl}

\usepackage[bibatend]{../header}
\usepackage{../my-boxes}

\usepackage{lastpage}
\usepackage{multicol}
\usepackage{simplewick}
\usepackage{slashed}
\usepackage{subcaption}
\usepackage{cancel}

\newcommand\timeorder{\mathscr T}
\newcommand\normorder{\mathscr N}
\newcommand\eye{\mat 1_4}
\newcommand\fourslash[1]{\slashed{\four{#1}}}

\hypersetup{
    pdftitle=
}

\graphicspath{{build/}}

\newcounter{totalpoints}
\newcommand\punkte[1]{#1\addtocounter{totalpoints}{#1}}

\newcounter{problemset}
\setcounter{problemset}{3}

\subject{physics7501 -- Advanced Quantum Field Theory}
\ihead{physics7501 -- Problem Set \arabic{problemset}}

\title{Problem Set \arabic{problemset}}

\newcommand\thegroup{Tutor: Thorsten Schimmanek}

\publishers{\thegroup}
\ofoot{\thegroup}

\author{
    Martin Ueding \\ \small{\href{mailto:mu@martin-ueding.de}{mu@martin-ueding.de}}
}
\ifoot{Martin Ueding}

\ohead{\rightmark}

\begin{document}

\maketitle

\vspace{3ex}

\begin{center}
    \begin{tabular}{rrr}
        Problem & Achieved points & Possible points \\
        \midrule
        \nameref{homework:1} & & \punkte{10} \\
        \nameref{homework:2} & & \punkte{10} \\
        \midrule
        Total & & \arabic{totalpoints}
    \end{tabular}
\end{center}

\vspace{3ex}

\begin{center}
    \begin{small}
        This document consists of \pageref{LastPage} pages.
    \end{small}
\end{center}

\section{The path integral in quantum mechanics}
\label{homework:1}

\subsection{Propagator}

\newcommand\xF{x_\mathrm f}
\newcommand\xI{x_\mathrm i}

I have to show that
\begin{align*}
    G(t, \xF, \xI)
    &= \int \dif x \, G(t - t', \xF, x) \, G(g', x, \xI) \,.
\end{align*}

This is straightforward. First I replace $G$ with the matrix element.
\begin{align*}
    G(t, \xF, \xI)
    &= \bra\xF \exp(- \iup t H) \ket\xI
    \intertext{%
        The exponential can be taken apart. The Hamiltonian commutes with
        itself and therefore the exponentials work like with c-numbers.
    }
    &= \bra\xF \exp(- \iup [t-t'] H) \exp(- \iup t' H) \ket\xI
    \intertext{%
        Then I insert a complete set of position eigenstates in between the
        exponentials.
    }
    &= \int \dif x \, \bra\xF \exp(- \iup [t-t'] H) \ket x \bra x \exp(- \iup t' H) \ket\xI
    \intertext{%
        Those matrix elements are propagators again. I arrive at
    }
    &= \int \dif x \, G(t - t', \xF, x) \, G(g', x, \xI) \,,
\end{align*}
which is the desired expression.

Next I check that adding a Heaviside step function makes it a retarded
Greens-function.
\begin{align*}
    [\iup \partial_0 - H] \, G_\mathrm r(t, \xF, \xI)
    &= [\iup \partial_0 - H] \, \Theta(t) \, G(t, \xF, \xI)
    \intertext{%
        I insert the explicit form of the propagator.
    }
    &= [\iup \partial_0 - H] \, \Theta(t) \, \bra\xF \exp(- \iup t H) \ket\xI
    \intertext{%
        The time derivative will give me two terms.
    }
    &= \iup \, \delta(t) \, \bra\xF \exp(- \iup t H) \ket\xI
    - \Theta(t) \, \bra\xF H \exp(- \iup t H) \ket\xI
    \\&\quad
    - H \Theta(t) \, \bra\xF \exp(- \iup t H) \ket\xI
    \intertext{%
        The first term will only contribute if $t = 0$. Since the exponential
        is smooth in $t$, one can just set $t = 0$ there. The other two terms
        should cancel. Letting the Hamiltonian act on $\bra \xF$ in both cases
        gives the same energy in both cases. The remainder of the first term is
        then the following:
    }
    &= \iup \, \delta(t) \, \braket{\xF|\xI} \,.
    \intertext{%
        And this simplifies to the desired right side of the equation:
    }
    &= \iup \, \delta(t) \, \delta(\xF - \xI) \,.
\end{align*}

\subsection{Split}

\subsection{Split of Hamiltonian}

\subsection{Limit}

\section{Multi-dimensional gaussian integration}
\label{homework:2}

\end{document}

% vim: spell spelllang=en tw=79
