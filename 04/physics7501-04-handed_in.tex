\documentclass[11pt, english, fleqn, DIV=15, headinclude]{scrartcl}

\usepackage[bibatend]{../header}
\usepackage{../my-boxes}

\usepackage{lastpage}
\usepackage{multicol}
\usepackage{simplewick}
\usepackage{slashed}
\usepackage{subcaption}
\usepackage{cancel}

\newcommand\timeorder{\mathscr T}
\newcommand\normorder{\mathscr N}
\newcommand\eye{\mat 1_4}
\newcommand\fourslash[1]{\slashed{\four{#1}}}
\newcommand\T{\mathrm T}

\hypersetup{
    pdftitle=
}

\graphicspath{{build/}}

\newcounter{totalpoints}
\newcommand\punkte[1]{#1\addtocounter{totalpoints}{#1}}

\newcounter{problemset}
\setcounter{problemset}{4}

\subject{physics7501 -- Advanced Quantum Field Theory}
\ihead{physics7501 -- Problem Set \arabic{problemset}}

\title{Problem Set \arabic{problemset}}

\newcommand\thegroup{Tutor: Thorsten Schimmanek}

\publishers{\thegroup}
\ofoot{\thegroup}

\author{
    Martin Ueding \\ \small{\href{mailto:mu@martin-ueding.de}{mu@martin-ueding.de}}
}
\ifoot{Martin Ueding}

\ohead{\rightmark}

\begin{document}

\maketitle

\vspace{3ex}

\begin{center}
    \begin{tabular}{rrr}
        Problem & Achieved points & Possible points \\
        \midrule
        \nameref{homework:1} & & \punkte{15} \\
        \midrule
        Total & & \arabic{totalpoints}
    \end{tabular}
\end{center}

\vspace{3ex}

\begin{center}
    \begin{small}
        This document consists of \pageref{LastPage} pages.
    \end{small}
\end{center}

\section{The optical theorem in $\phi^4$-theory}
\label{homework:1}

\subsection{Validity of analysis}

The three channels in are shown in Figure~\ref{fig:channels}. The loop momentum
is always $\four q$ and an additional four-momentum transfer is needed. The
computation of the diagrams is very parallel as they all have two propagators.
The $\phi^4$-theory diagrams feel rather relaxing after all the fermionic
diagrams!

\begin{figure}
    \begin{subfigure}[c]{0.3\linewidth}
        \centering
        \includegraphics{s-channel}
        \caption{%
            $s$-channel
        }
        \label{fig:s-channel}
    \end{subfigure}
    \hfill
    \begin{subfigure}[c]{0.3\linewidth}
        \centering
        \includegraphics{t-channel}
        \caption{%
            $t$-channel
        }
        \label{fig:t-channel}
    \end{subfigure}
    \hfill
    \begin{subfigure}[c]{0.3\linewidth}
        \centering
        \includegraphics{u-channel}
        \caption{%
            $u$-channel
        }
        \label{fig:u-channel}
    \end{subfigure}
    \caption{%
        The three diagrams in $\lambda^2$ forward scattering. Time direction is
        to the right. $\four \Sigma := \four k_1 + \four k_2$, $\four \Delta =
        \four k_1 - \four k_2$.
    }
    \label{fig:channels}
\end{figure}

\subsubsection{$s$-channel}

The invariant matrix element for the $s$-channel is
\begin{align*}
    \iup \mathscr M
    &= [- \iup \lambda]^2 \int \frac{\dif^4 q}{[2 \piup]^4}
    \frac{\iup}{\sbr{\frac{\four k}2 - \four q}^2 - m^2 + \iup \epsilon}
    \frac{\iup}{\sbr{\frac{\four k}2 + \four q}^2 - m^2 + \iup \epsilon} \,.
    \intertext{%
        Grouping all the minus signs gives us
    }
    &= \lambda^2 \int \frac{\dif^4 q}{[2 \piup]^4}
    \frac{1}{\sbr{\frac{\four k}2 - \four q}^2 - m^2 + \iup \epsilon}
    \frac{1}{\sbr{\frac{\four k}2 + \four q}^2 - m^2 + \iup \epsilon} \,.
\end{align*}
From here we introduce Feynman parameters and look at the denominator only. The
numerator is not very interesting in this case as we have only $\phi^4$-theory.
The denominator with parameters then is
\begin{align*}
    D
    &= x \sbr{\frac{\four k}2 - \four q}^2
    + y \sbr{\frac{\four k}2 + \four q}^2 - m^2 + \iup \epsilon \,.
    \intertext{%
        We can expand the brackets and have
    }
    &= \frac{\four k^2}{4} + \four q^2 - m^2 + \iup \epsilon + [y-x] \four k
    \cdot \four q \,, \\
    &= \four q^2 + \four q \cdot \four k [y-x] + \frac{\four k^2}{4} + \iup
    \epsilon - m^2 \,.
    \intertext{%
        Now we will complete the square to make the shift in the integration
        variable.
    }
    &= \sbr{\four q + \frac{\four k}{2} [y - x]}^2 - \frac{\four k^2}{4}[y-x]^2
    + \frac{\four k^2}{4} - m^2 + \iup \epsilon
\end{align*}
The square bracket will become $\four l$ and the $\Delta$ then is
\[
    \Delta = \frac{\four k^2}{4} \sbr{[y-x]^2 - 1} + m^2 \,.
\]

From the procedure of dimensional regularization we can just jump to the final
result which is
\[
    \iup \mathscr M
    = \lambda^2 \frac{\iup}{[4\piup]^2} \sbr{\frac 2\epsilon - \gamma - 2
    \ln(\Delta) + \mathrm O(\epsilon)} \,.
\]
For $\Delta \in \R^+$ this will give a real number for $\mathscr M$; there is
an imaginary unit on the left and right side of the equation which could be
canceled. There is still the $2/\epsilon$ divergence but we will not look at
that. It should also create some problem but it will probably cancel or
something like that. However, it is a real divergence and will never contribute
any imaginary part.

For $\Delta < 0$ we will have a problem with the logarithm. It will give some
sort of imaginary part depending on where one put the branch cut of the
logarithm itself. For $\Delta < 0$ we have the need to look at the range of the
Feynman parameters. $[y - x]^2$ will always lie in the interval $[0, 1]$.
Subtracting one will give us $[-1, 0]$ as the interval. We take $-1$ to be the
most extreme case and obtain $|\four k| > 2m$ from that. This is exactly the
particle creation threshold that is needed to get both propagators on-shell at
the same time.

\subsubsection{$t$-channel}

The $t$-channel is shown in Figure~\ref{fig:t-channel} and contains no net
momentum transfer in the loop. Therefore the propagator part of the invariant
matrix element is just
\[
    \frac{1}{[\four q^2 - m^2 + \iup \epsilon]^2} \,.
\]
We can write the loop momentum as $\four q^2 = E_q^2 - \vec q^2$. Then the
denominator is
\[
    \frac{1}{[E_q^2 - \vec q^2 - m^2 + \iup \epsilon]^2} \,.
\]
The whole denominator has to be zero to give a pole. The integration over $E_q$
is the one which shall be the one going over the pole. The pole will be at
\[
    E_q = \pm \sqrt{m^2 + \vec q^2 - \iup \epsilon}
    \simeq \pm \sqrt{m^2 + \vec q^2} \mp \iup \epsilon \,.
\]
The residual of the denominator (with the square) for $E_q > 0$ is
\[
    - \frac{1}{4 \sqrt{\vec q^2 + m^2}^3}
\]
whereas the other side is positive. If take the first one, the pole is slightly
below the real axis. We close the contour downwards (anti clockwise) and incur
a factor $- 2 \piup \iup$ from the residue theorem and the negative counting
winding number. Therefore the whole expression becomes
\[
    \frac{\piup \iup}{2 \sqrt{\vec q^2 + m^2}^3} \,.
\]
It is completely imaginary. This is proportional (with a real factor) to $\iup
\mathscr M$ as there are only momentum integrations in the three-space as well
as constant factors left. Therefore $\mathscr M \in \R$ and it does not
contribute to the optical theorem.

This can also be seen when using the Feynman parameters. They are not really
needed here as the integrand already has the desired form with $\Delta = m^2$
here. Since $\Delta$ does not depend on $\four k$ here and is from $\R^+$,
there is no way to create some branch cut with this.

\subsubsection{$u$-channel}

The Feynman diagram for the $u$-channel is shown in Figure~\ref{fig:u-channel}.
The two propagators are
\[
    \frac{1}{\sbr{\four q + \frac{\four\Delta}2}^2 - m^2 + \iup\epsilon}
    \frac{1}{\sbr{\four q - \frac{\four\Delta}2}^2 - m^2 + \iup\epsilon} \,.
\]
The momentum transfer $\four \Delta$ is defined as $\four k_1 - \four k_2$. In
the center of mass frame for both particles (which we assume we are identical)
the transfer is given as $\four \Delta = (0, 2\vec k_1)$. Then we can write the
denominators as
\[
    \sbr{\four q \pm \frac{\four\Delta}2}^2 - m^2 + \iup\epsilon
    = E_q^2 - \vec q \pm 2 \vec q \cdot \vec k_1 - \vec k_1^2 - m^2 +
    \iup\epsilon \,.
\]
Although this is different from the $t$-channel, the position of the pole is
just shifted, the structure of the poles should still be the same. Therefore
the residue is real and the invariant matrix element is purely imaginary. This
diagram does not contribute either.

\subsubsection{Verification of the optical theorem}

We are asked to check the validity of the optical theorem. This requires the
comparison with the other side. So far we have computed the imaginary part of
all $\lambda^2$ diagrams shown in Figure~\ref{fig:channels}.

Should we just add the real and imaginary parts of all the diagrams up, take
the modulus squared and see that it matches (up to a couple factors) to the
imaginary part of the $s$-channel?

\subsection{Identity}

Once you know the trick, this one is actually quite simple. One uses an
integration contour like shown in Figure~\ref{fig:contour}.

\begin{figure}
    \centering
    \includegraphics{contour}
    \caption{%
        Integration contour for distribution identity.
    }
    \label{fig:contour}
\end{figure}

Then the integral with a test function $f$ can be written as where we split the
integral into three sections
\begin{align*}
    \int \dif x \, \frac{f(x)}{x + \iup \epsilon}
    &= \sbr{\int_{-\infty}^{-\delta} + \int_{-\delta}^{\delta} +
    \int_\delta^\infty} \dif x \,
    \frac{f(x)}{x + \iup \epsilon}
    \,.
    \intertext{%
        In the first and last integral, nothing interesting happens when
        $\epsilon \to 0$. Therefore we can directly write this shorter as
    }
    &= \int_{-\delta}^{\delta} \dif x \,
    \frac{f(x)}{x + \iup \epsilon} + P\del{\frac {f(x)}x} \,.
    \intertext{%
        The first integral has to be evaluated explicitly. We parametrize this
        infinitesimal section as the contour shown in Figure~\ref{fig:contour}.
        The change of variables $x$ to $\delta \eup^{\iup \theta}$ will give us
        an additional factor of $\iup \delta \eup^{\iup \theta}$ for $\dif
        \theta$ in the integration.
    }
    &= \int_{\piup}^{0} \dif \theta \, \iup \delta \eup^{\iup \theta}
    \frac{f(\delta \eup^{\iup \theta})}{\delta \eup^{\iup \theta} + \iup
    \epsilon} + P\del{\frac {f(x)}x}
    \intertext{%
        Then we can let $\epsilon \to 0$ and cancel the remaining divisor.
    }
    &= \int_{\piup}^{0} \dif \theta \, \iup 
    f(\delta \eup^{\iup \theta})
    + P\del{\frac {f(x)}x}
    \intertext{%
        We can also let $\delta \to 0$ now safely and then the integral will
        just give us a factor of $- \piup$.
    }
    &= - \iup \piup f(0) + P\del{\frac {f(x)}x}
\end{align*}
Without the test function $f$ the identity assumes the desired form of
\[
    \lim_{\epsilon\to 0} \frac{1}{x + \iup \epsilon}
    = - \iup \piup \delta(x) + P\del{\frac 1x} \,.
\]

\end{document}

% vim: spell spelllang=en tw=79
