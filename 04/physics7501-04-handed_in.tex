\documentclass[11pt, english, fleqn, DIV=15, headinclude]{scrartcl}

\usepackage[bibatend]{../header}
\usepackage{../my-boxes}

\usepackage{lastpage}
\usepackage{multicol}
\usepackage{simplewick}
\usepackage{slashed}
\usepackage{subcaption}
\usepackage{cancel}

\newcommand\timeorder{\mathscr T}
\newcommand\normorder{\mathscr N}
\newcommand\eye{\mat 1_4}
\newcommand\fourslash[1]{\slashed{\four{#1}}}
\newcommand\T{\mathrm T}

\hypersetup{
    pdftitle=
}

\graphicspath{{build/}}

\newcounter{totalpoints}
\newcommand\punkte[1]{#1\addtocounter{totalpoints}{#1}}

\newcounter{problemset}
\setcounter{problemset}{4}

\subject{physics7501 -- Advanced Quantum Field Theory}
\ihead{physics7501 -- Problem Set \arabic{problemset}}

\title{Problem Set \arabic{problemset}}

\newcommand\thegroup{Tutor: Thorsten Schimmanek}

\publishers{\thegroup}
\ofoot{\thegroup}

\author{
    Martin Ueding \\ \small{\href{mailto:mu@martin-ueding.de}{mu@martin-ueding.de}}
}
\ifoot{Martin Ueding}

\ohead{\rightmark}

\begin{document}

\maketitle

\vspace{3ex}

\begin{center}
    \begin{tabular}{rrr}
        Problem & Achieved points & Possible points \\
        \midrule
        \nameref{homework:1} & & \punkte{15} \\
        \midrule
        Total & & \arabic{totalpoints}
    \end{tabular}
\end{center}

\vspace{3ex}

\begin{center}
    \begin{small}
        This document consists of \pageref{LastPage} pages.
    \end{small}
\end{center}

\section{The optical theorem in $\phi^4$-theory}
\label{homework:1}

\subsection{Validity of analysis}

The three channels in are shown in Figure~\ref{fig:channels}. The loop momentum
is always $\four q$ and an additional four-momentum transfer is needed. The
computation of the diagrams is very parallel as they all have two propagators.
The $\phi^4$-theory diagrams feel rather relaxing after all the fermionic
diagrams!

\begin{figure}
    \begin{subfigure}[c]{0.3\linewidth}
        \centering
        \includegraphics{s-channel}
        \caption{%
            $s$-channel
        }
        \label{fig:s-channel}
    \end{subfigure}
    \hfill
    \begin{subfigure}[c]{0.3\linewidth}
        \centering
        \includegraphics{t-channel}
        \caption{%
            $t$-channel
        }
        \label{fig:t-channel}
    \end{subfigure}
    \hfill
    \begin{subfigure}[c]{0.3\linewidth}
        \centering
        \includegraphics{u-channel}
        \caption{%
            $u$-channel
        }
        \label{fig:u-channel}
    \end{subfigure}
    \caption{%
        The three diagrams in $\lambda^2$ forward scattering. Time direction is
        to the right. $\four \Sigma := \four k_1 + \four k_2$, $\four \Delta =
        \four k_1 - \four k_2$.
    }
    \label{fig:channels}
\end{figure}

\subsubsection{$t$-channel}

The $t$-channel is shown in Figure~\ref{fig:t-channel} and contains no net
momentum transfer in the loop. Therefore the propagator part of the invariant
matrix element is just
\[
    \frac{1}{[\four q^2 - m^2 + \iup \epsilon]^2} \,.
\]
We can write the loop momentum as $\four q^2 = E_q^2 - \vec q^3$. Then the
denominator is
\[
    \frac{1}{[E_q^2 - \vec q^2 - m^2 + \iup \epsilon]^2} \,.
\]
The whole denominator has to be zero to give a pole. The integration over $E_q$
is the one which shall be the one going over the pole. The pole will be at
\[
    E_q = \pm \sqrt{m^2 + \vec q^2 - \iup \epsilon}
    \simeq \pm \sqrt{m^2 + \vec q^2} \mp \iup \epsilon \,.
\]
The residual of the denominator (with the square) for $E_q > 0$ is
\[
    - \frac{1}{4 \sqrt{\vec q^2 + m^2}^3}
\]
whereas the other side is positive. If take the first one, the pole is slightly
below the real axis. We close the contour downwards (anti clockwise) and incur
a factor $- 2 \piup \iup$ from the residue theorem and the negative counting
winding number. Therefore the whole expression becomes
\[
    \frac{\piup \iup}{2 \sqrt{\vec q^2 + m^2}^3} \,.
\]
It is completely imaginary. This is proportional (with a real factor) to $\iup
\mathscr M$ as there are only momentum integrations in the three-space as well
as constant factors left. Therefore $\mathscr M \in \R$ and it does not
contribute to the optical theorem.

\subsubsection{$u$-channel}

The Feynman diagram for the $u$-channel is shown in Figure~\ref{fig:u-channel}.
The two propagators are
\[
    \frac{1}{\sbr{\four q + \frac{\four\Delta}2}^2 - m^2 + \iup\epsilon}
    \frac{1}{\sbr{\four q - \frac{\four\Delta}2}^2 - m^2 + \iup\epsilon} \,.
\]
The momentum transfer $\four \Delta$ is defined as $\four k_1 - \four k_2$. In
the center of mass frame for both particles (which we assume we are identical)
the transfer is given as $\four \Delta = (0, 2\vec k_1)$. Then we can write the
denominators as
\[
    \sbr{\four q \pm \frac{\four\Delta}2}^2 - m^2 + \iup\epsilon
    = E_q^2 - \vec q \pm 2 \vec q \cdot \vec k_1 - \vec k_1^2 - m^2 +
    \iup\epsilon \,.
\]
Although this is different from the $t$-channel, the position of the pole is
just shifted, the structure of the poles should still be the same. Therefore
the residue is real and the invariant matrix element is purely imaginary. This
diagram does not contribute either.

\the\linewidth

\end{document}

% vim: spell spelllang=en tw=79
