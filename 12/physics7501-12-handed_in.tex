\documentclass[11pt, english, fleqn, DIV=15, headinclude]{scrartcl}

\usepackage[bibatend]{../header}
\usepackage{../my-boxes}

\usepackage{lastpage}
\usepackage{multicol}
\usepackage{simplewick}
\usepackage{multicol}
\usepackage{adjustbox}
\usepackage{slashed}
\usepackage{subcaption}
\usepackage{cancel}
\usepackage{tikzsymbols}
\usepackage{placeins}

\newcommand\timeorder{\mathscr T}
\newcommand\normorder{\mathscr N}
\newcommand\eye{\mat 1_4}
\newcommand\fourslash[1]{\slashed{\four{#1}}}
\newcommand\T{\mathrm T}

\hypersetup{
    pdftitle=
}

\graphicspath{{build/}}

\newcounter{totalpoints}
\newcommand\punkte[1]{#1\addtocounter{totalpoints}{#1}}

\newcounter{problemset}
\setcounter{problemset}{12}

\subject{physics7501 -- Advanced Quantum Field Theory}
\ihead{physics7501 -- Problem Set \arabic{problemset}}

\title{Problem Set \arabic{problemset}}

\newcommand\thegroup{Tutor: Thorsten Schimannek}

\publishers{\thegroup}
\ofoot{\thegroup}

\author{
    Martin Ueding \\ \small{\href{mailto:mu@martin-ueding.de}{mu@martin-ueding.de}}
}
\ifoot{Martin Ueding}

\ohead{\rightmark}

\begin{document}

\maketitle

\vspace{3ex}

\begin{center}
    \begin{tabular}{rrr}
        \toprule
        Problem & Achieved points & Possible points \\
        \midrule
        \nameref{homework:1} & & \punkte{12} \\
        \midrule
        Total & & \arabic{totalpoints} \\
        \bottomrule
    \end{tabular}
\end{center}

\vspace{3ex}

\begin{center}
    \begin{small}
        This document consists of \pageref{LastPage} pages.
    \end{small}
\end{center}

\section{Chiral anomalies à la Fujikava}
\label{homework:1}

\subsection{Invariance}

Giving names makes them less scary\footnote{Just as they called the hedge
\enquote{Hugo} in the German version of \enquote{Over the Hedge} to make it
less scary.}, so we introduce
\[
    C := \exp\del{\iup \frac\alpha2 \mat\gamma^5}
\]
to be the chiral transformation. Since $\mat\gamma^5$ is hermitian, we adjoint
transformation is just
\[
    C^\dagger = \exp\del{- \iup \frac\alpha2 \mat\gamma^5} \,.
\]
The sandwich between $\mat\gamma^0$ of the chiral transformation converts it
into its hermitian conjugate. We start with just expanding $C$ and have
\begin{align*}
    \mat\gamma^0 C \mat\gamma^0
    &= \mat\gamma^0 \exp\del{\iup \frac\alpha2 \mat\gamma^5} \mat\gamma^0 \,.
    \intertext{%
        Then we expand the exponential as a series and have
    }
    &= \mat\gamma^0
    \sum_{n = 0}^\infty
    \frac{1}{n!} \sbr{\iup \frac\alpha2 \mat\gamma^5}^n
    \mat\gamma^0 \,.
    \intertext{%
        With $\mat\gamma^0 \mat\gamma^0 = \mat 1_4$ we can wrap every single
        $\mat\gamma^5$ into $\mat\gamma^0$.
    }
    &=
    \sum_{n = 0}^\infty
    \frac{1}{n!} \sbr{\iup \frac\alpha2 \mat\gamma^0 \mat\gamma^5 \mat\gamma^0}^n
    \,.
    \intertext{%
        Either using the anticommutation or the explicit $\mat\gamma^5 = \iup
        \mat\gamma^0 \mat\gamma^1 \mat\gamma^2 \mat\gamma^3$ one can show that
        this flips the sign.
    }
    &= \sum_{n = 0}^\infty
    \frac{1}{n!} \sbr{- \iup \frac\alpha2 \mat\gamma^5}^n
    \intertext{%
        This then gives us
    }
    &= C^\dagger \,.
\end{align*}

The quark term in the Lagrangian then transforms as
\begin{align*}
    \iup \bar q \slashed \Dif q
    &= \iup q^\dagger \mat\gamma^0 \slashed \Dif q \\
    &\to \iup q^\dagger C^\dagger \mat\gamma^0 \slashed \Dif C q \,.
    \intertext{%
        We insert pairs of $\mat\gamma^0$ and add some suggestive spacing. We
        have
    }
    &= \iup q^\dagger \mat\gamma^0 \; \mat\gamma^0 C^\dagger \mat\gamma^0 \;
    \slashed \Dif C q \,.
    \intertext{%
        Then we can get back the \enquote{bar} and also remove the dagger on
        the $C$ to give
    }
    &= \iup \bar q C \slashed \Dif C q \,.
    \intertext{%
        The derivative does not act on the transformation as it is a global
        transformation. The \enquote{slashed} Dirac matrices however
        anticommute with the ones in the exponential. Therefore the sign in the
        exponential flips again and we have
    }
    &= \iup \bar q \slashed \Dif C^\dagger C q \,.
    \intertext{%
        And then we can cancel the two chiral transformation to yield a unit
        matrix and we are back to the original Lagrangian, 
    }
    &= \iup \bar q \slashed \Dif q \,.
\end{align*}
Therefore the chiral transformation is a global symmetry of the Lagrangian. The
gluon field is not affected by this transformation acting on the Dirac
structure only as the gluons only have Lorentz and gauge group structure.

\subsection{Local symmetry}

The chiral transformation is now promoted to a local symmetry. This means that
it does not commute with the partial derivative any more. From the previous
part we know that the transformed Lagrangian has the form 
$\bar q C \slashed D C q$ which we can then expand as
\begin{align*}
    \bar q C \slashed D C q
    &= \bar q C \slashed \partial C q + \iup g \bar q C \slashed A C q \,.
    \intertext{%
        In the first term, we need to apply the product rule. The $C$ will be
        hermitian conjugated when we move it through the Dirac matrix from the
        \enquote{slashed} notation. In the second we can just commute $C$ and
        $\vec A$ to give $C C^\dagger$ and get that back to the untransformed
        way. As that term was included before, we drop that. We we are left
        with
    }
    &\simeq \bar q C (\slashed \partial C) q
    + \bar q C C^\dagger \slashed \partial q \,.
    \intertext{%
        In the first term we actually need to compute the derivative. The
        second term does not feature an explicit transformation any more, so we
        can just drop that as well. We have
    }
    &\simeq \frac \iup 2 \bar q C \mat\gamma^\mu \alpha_{,\mu}(\four x) \mat\gamma_5 C q \,,
    \intertext{%
        where we can now anticommute the $C$ across the $\mat\gamma^\mu$ and
        obtain $C^\dagger$. That will cancel the other $C$. Then we pull out
        the $\alpha$ and we have
    }
    &= \frac \iup 2 \alpha_{,\mu}(\four x) \bar q \mat\gamma^\mu \mat\gamma_5 q \,.
    \intertext{%
        Using $j_5^\mu := \bar q \mat\gamma^\mu \mat\gamma_5 q$ we can write this
        as
    }
    &= \frac \iup 2 \alpha_{,\mu}(\four x) j_5^\mu \,.
\end{align*}
The difference in the Lagrangian is almost Equation~(3) from the problem set.
We got a factor $\iup/2$ more. The desired result contains three indices $\mu$
which does not seem right. This here is probably not far off from the actual
result.

\end{document}

% vim: spell spelllang=en tw=79
