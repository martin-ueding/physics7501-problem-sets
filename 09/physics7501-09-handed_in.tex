\documentclass[11pt, english, fleqn, DIV=15, headinclude]{scrartcl}

\usepackage[bibatend]{../header}
\usepackage{../my-boxes}

\usepackage{lastpage}
\usepackage{multicol}
\usepackage{simplewick}
\usepackage{multicol}
\usepackage{slashed}
\usepackage{subcaption}
\usepackage{cancel}
\usepackage{tikzsymbols}
\usepackage{placeins}

\newcommand\timeorder{\mathscr T}
\newcommand\normorder{\mathscr N}
\newcommand\eye{\mat 1_4}
\newcommand\fourslash[1]{\slashed{\four{#1}}}
\newcommand\T{\mathrm T}

\hypersetup{
    pdftitle=
}

\graphicspath{{build/}}

\newcounter{totalpoints}
\newcommand\punkte[1]{#1\addtocounter{totalpoints}{#1}}

\newcounter{problemset}
\setcounter{problemset}{9}

\subject{physics7501 -- Advanced Quantum Field Theory}
\ihead{physics7501 -- Problem Set \arabic{problemset}}

\title{Problem Set \arabic{problemset}}

\newcommand\thegroup{Tutor: Thorsten Schimannek}

\publishers{\thegroup}
\ofoot{\thegroup}

\author{
    Martin Ueding \\ \small{\href{mailto:mu@martin-ueding.de}{mu@martin-ueding.de}}
}
\ifoot{Martin Ueding}

\ohead{\rightmark}

\begin{document}

\maketitle

\vspace{3ex}

\begin{center}
    \begin{tabular}{rrr}
        Problem & Achieved points & Possible points \\
        \midrule
        \nameref{homework:1} & & \punkte{20} \\
        \midrule
        Total & & \arabic{totalpoints}
    \end{tabular}
\end{center}

\vspace{3ex}

\begin{center}
    \begin{small}
        This document consists of \pageref{LastPage} pages.
    \end{small}
\end{center}

\section{Fixing the non-Abelian gauge (with ghosts)}
\label{homework:1}

\subsection{Functional determinant and propagator}

\paragraph{Functional determinant}

In the functional determinant there is the term
\[
    \frac{\deltaup G([\hat A^\alpha]^a)}{\deltaup \alpha} \,.
\]
We have to evaluate this expression.

$G$ is the gauge fixing function where the condition simply is $G(\hat A) \equiv 0$.
The condition used is the Lorenz gauge $G(\hat A) = \partial^\mu \hat A_\mu$ also
used by \textcite[512]{Peskin/QFT/1995}. The transformed $\hat A^\alpha$ is
given by
\[
    [\hat A^\alpha]^a_\mu t^a = \exp(\iup \alpha^a t^a) \sbr{A_\mu^b t^b +
    \frac \iup g \partial_\mu} \exp(\iup \alpha^c t^c) \,.
\]
The infinitesimal form, which is more useful here, is given by
\[
    [\hat A^\alpha]^a_\mu t^a = \sbr{A_\mu^a + \frac 1g \partial_\mu \alpha^a +
    f^{abc} A^b_\mu \alpha^c} t^a + \mathrm O(\alpha^2) \,.
\]
We can now take the partial derivative of this and obtain the gauge functional
\[
    G(\hat A^\alpha) = \partial^\mu \hat A_\mu + \frac 1g \dalambert \hat
    \alpha
    +
    f^{abc} \partial^\mu A^b_\mu \alpha^c t^a + \mathrm O(\alpha^2) \,.
\]
A functional derivative with respect to $\alpha^c$ gives us
\[
    \frac{\deltaup G([\hat A^\alpha]^a)}{\deltaup \alpha}
    = \frac 1g \dalambert \delta^{ac} + f^{abc} \partial^\mu A_\mu^b \,.
\]
That is the expression to derive.

\paragraph{Propagator}

For the propagator of the gluons we have to expand the gauge kinetic term. This
is
\begin{align*}
    \mathcal L_\text{gauge-kinetic}
    &= - \frac14 F_{\mu\nu}^a F^{\mu\nu}_a - \frac{1}{2\xi} \sbr{\partial^\mu
    A_\mu^a}^2 \,.
    \intertext{%
        Now we will have to expand this term. The gauge field strength tensor
        expanded gives us
    }
    &= - \frac14 \sbr{\partial_\mu A_\nu^a - \partial_\nu A_\mu^a + g f^{abc}
    A_\mu^b A_\nu^c}^2
    - \frac{1}{2\xi} \sbr{\partial^\mu A_\mu^a}^2 \,.
    \intertext{%
        The partial derivatives are not that handy, therefore we convert them
        to $\iup k_\mu$ and regard the expression being in momentum space. Then
        we have
    }
    &= - \frac14 \sbr{\iup k_\mu A_\nu^a - \iup k_\nu A_\mu^a + g f^{abc}
    A_\mu^b A_\nu^c}^2
    - \frac{1}{2\xi} \sbr{\iup k^\mu A_\mu^a}^2 \,.
    \intertext{%
        The square in the first expression will give nine terms, some of them
        are redundant directly or after renaming the internal indices. We then
        have
    }
    &= - \frac14 \sbr{-2 k^2 A_\nu^a A^\nu_a + 2 k^\mu k_\nu A_\mu^a A_\nu^a +
    2 \iup g f^{abc} \sbr{k^\mu A^\nu_a - k^\nu A_\mu^a} A_\mu^b A_\nu^c + g^2
    f^{abc} f^{ade} A_\mu^b A_\nu^c A^\mu_d A^\nu_e}
    \\&\qquad
    - \frac{1}{2\xi} \sbr{\iup k^\mu A_\mu^a}^2 \,.
\end{align*}

This is the part I am not sure about. There are a lot of terms in this
expression. They contain two, three or four terms of $A$. Can I now just say
that the propagator is the quadratic term in $A$ and only look at that?
Otherwise the propagator will not turn out to be similar to the photon
propagator. When only the quadratic terms in $A$ are taken into account, no
terms with structure constants remain. The whole non-abelian part of the theory
is not apparent in the quadratic term which looks exactly like the photon part.
Therefore one can conclude from here that gluons propagate like photons. Gluon
propagation does not change the color-adj., so one needs an additional
$\delta^{ab}$ factor.

The kinetic operator can then be written suggestively as
\[
    A_\mu^a \mathcal K^{\mu\nu}_{ab} A_\nu^b := \frac 12 A_\mu^a \sbr{k^2
    g^{\mu\nu} - \sbr{1 - \frac 1\xi} k^\mu k^\nu} \delta^{ab} A_\nu^b \,.
\]
A properly defined inverse of this $\mathcal K$ is the propagator. The defining
equation is
\[
    \mathcal K_{\mu\nu}^{ab}
    \tilde D^{\nu\rho}_{bc} = \iup \delta^\rho_\mu \delta^{ac} \,.
\]
One then goes by Lorentz symmetry and finds that the only terms that may
contribute to $\tilde D$ are $g^{\nu\rho}$ and $k^\nu k^\rho$ as well as the
scalar $k^2$. Setting up an ansatz
\[
    \tilde D^{\nu\rho}_{bc} = \sbr{a g^{\nu\rho} + b k^\nu k^\rho} \delta_{bc}
\]
and plugging that into the equation will give us
\[
    a = \frac{\iup}{k^2}
    \eqnsep
    b = \frac{\iup}{k^3} [\xi - 1] \,.
\]
From this we obtain the propagator
\[
    \tilde D^{\mu\nu} = \frac{\iup}{k^2} \sbr{g^{\mu\nu} + [1 - \xi]
    \frac{k^\mu k^\nu}{k^2}} \,,
\]
where we can flip the minus sign to give the usual form of
\[
    \tilde D^{\mu\nu} = \frac{\iup}{k^2} \sbr{g^{\mu\nu} - [\xi - 1]
    \frac{k^\mu k^\nu}{k^2}} \,.
\]

\FloatBarrier
\subsection{Feynman rules for ghosts}

In the equation~(3) given on the problem set the ghost fields do not explicitly
contain color-adjoint indices. The expression should be
\[
    \bar c^a \sbr{\frac1g \dalambert \delta^{ac} + f^{abc} \partial_\mu
    A_\mu^b} c^c \,.
\]
The $c^c$ looks rather weird as both $c$s are not members of the same field.
The normal one is a Grassmann field and the superscript is a natural number, a
color-adjoint index.

\paragraph{Naïve extraction}

The propagator seems easy. The quadratic part in the ghost field is just a
massless Klein-Gordon equation which also preserves color-adjoint. In the
momentum space this is
\[
    \frac 1g \bar c^a \four k^2 \delta^{ab} c^b \,.
\]
The kinetic operator is just $\four k^2 \delta^{ab} /g$ and therefore the
inverse is
\[
    \frac 1g \frac{\iup \delta^{ab}}{\four k^2} \,.
\]
The imaginary unit comes from the definition of the inverse as used in the
previous problem.

Just looking at the Lagrangian density in the given path integral one read off
the vertex to some extent. The partial derivative has to be converted into a
momentum. For some reason that is the momentum of the outgoing ghost.
\Textcite[514]{Peskin/QFT/1995} write:

\begin{quote}
    \enquote{%
        In the interaction term of (16.32), the derivative stands to the left
        of the gauge field; this implies that this derivative is evaluated with
        the momentum coming out of the vertex along the ghost line.%
    }
\end{quote}

\begin{table}
    \begin{question}
        Why does the partial derivative become the momentum of $\bar c$?
    \end{question}
\end{table}

If one assumes this, then the vertex is easy. The term interaction term in the
Lagrangian is just
\[
    - \bar c^a f^{abc} \partial^\mu A^b_\mu c^c \,.
\]
Now replace the partial derivative with the momentum of $\bar c$ and obtain
\[
    - \iup k^\mu \bar c^a f^{abc} A^b_\mu c^c \,.
\]
Taking the functional derivative with respect to all the terms we have
\[
    - \iup k^\mu f^{abc} \,.
\]
Another $\iup$ comes in for the vertex from some $\iup$ in the path integral.
So all in all we have
\[
    k^\mu f^{abc}
\]
for the vertex. This is not correct, \textcite[Figure~16.5]{Peskin/QFT/1995}
gives the same rule with a minus sign. Therefore we made a sign mistake by
ignoring the anticommuting property of the Grassmann field~$c$. The factor $g$
that is missing here seems to be absorbed or extracted from the ghost field.
Therefore it does not directly appear in the vertex expression.

\begin{table}
    \begin{question}
        Where does the imaginary unit come from that we have to add to every
        vertex? I remember it being from the path integral but I am not sure
        which $\iup$ it was exactly.
    \end{question}
\end{table}

\paragraph{Generating functional}

\FloatBarrier

\subsection{Yang-Mills Lagrangian}

The Yang-Mills Lagrangian coupling to quarks is given by
\[
    \mathcal L_\text{YM}
    = \underbrace{\bar\psi [\iup \slashed\Dif - m] \psi}_\text{coupled fermion}
    - \underbrace{\frac 14 \trcolorf\del{\hat F_{\mu\nu} \hat F^{\mu\nu}}}_\text{gluons}
    - \underbrace{\frac{1}{2\xi} \partial^\mu \hat A_\mu}_\text{gauge fixing}
    + \underbrace{\bar c^a \dalambert c^a}_\text{free ghost}
    + \underbrace{\bar c^a g f^{abc} \partial^\mu A_\mu^b c^c}_\text{ghost--gluon
    interaction} \,.
\]
This is a rather compact notation where complexity is hidden in the
gauge-covariant derivative and the gluon field-strength tensor. The trace is
meant to go over the fundamental indices as the gluon field-strength tensor
$\tens{\hat F}$ is a shorthand for $\tens F^a \mat t^a$ and the generators
$\mat t^a$ are given in the fundamental representation.

The gluon term must be expanded to give an expression to work with and derive
Feynman rules from.
\begin{align*}
    - \frac14 \trcolorf\del{\hat F_{\mu\nu} \hat F^{\mu\nu}}
    &= - \frac14 F_{\mu\nu}^a F^{\mu\nu}_b \trcolorf(t^a t^b) \\
    &= - \frac18 F_{\mu\nu}^a F^{\mu\nu}_b \delta^{ab} \\
    &= - \frac18 F_{\mu\nu}^a F^{\mu\nu}_a \\
    &= - \frac18 \sbr{\partial_\mu A_\nu^a - \partial_\nu A_\mu^a + g f^{abc} A_\mu^b
    A_\nu^c}^2 \\
    &= - \frac18 \sbr{\iup k_\mu A_\nu^a - \iup k_\nu A_\mu^a + g f^{abc} A_\mu^b
    A_\nu^c}^2 \\
    &= - \frac18 \bigg[
        - 2 k^2 (A^a)^2
        + 2 k^\mu k^\nu A^a_\mu A^a_\nu
        + 2 \iup g f^{abc} \sbr{k^\mu a^\nu_a - k^\nu A^\mu_a} A_\mu^b A_\nu^c
        \\
        &\qquad
        + g^2 f^{abc} f^{ade} A_\mu^b A_\nu^c A^\mu_d A^\nu_e
    \bigg] \\
    &= \frac14 k^2 (A^a)^2
    - \frac14 k^\mu k^\nu A^a_\mu A^a_\nu
    - \frac12 \iup g f^{abc} k^\mu a^\nu_a A_\mu^b A_\nu^c
    - \frac18 g^2 f^{abc} f^{ade} A_\mu^b A_\nu^c A^\mu_d A^\nu_e
\end{align*}

The expanded Lagrangian can be written as
\begin{align*}
    \mathcal L_\text{YM, Quarks} &= \mathcal L_\text{free} + \mathcal L_\text{int}
    \intertext{with}
    \mathcal L_\text{free}
    &=
    \underbrace{\bar\psi [\iup \slashed\partial - m] \psi}_\text{free fermion}
    + \underbrace{\frac14 A^a_\mu \sbr{\dalambert g^{\mu\nu} -
    \partial^\mu \partial^\nu} A^a_\nu}_\text{free gluon}
    + \underbrace{\bar c^a \dalambert c^a}_\text{free ghost}
    - \underbrace{\frac{1}{2\xi} \partial^\mu \hat A_\mu}_\text{gauge fixing}
    \\
    \mathcal L_\text{int}
    &=
    \underbrace{g \bar\psi \hat{\fourslash A} \psi}_\text{fermion--gluon}
    + \underbrace{\bar c^a g f^{abc} [\partial^\mu A_\mu^b]
    c^c}_\text{ghost--gluon}
    - \underbrace{\frac12 \iup g f^{abc} [\partial^\mu A^\nu_a] A_\mu^b
    A_\nu^c}_\text{triple gluon interaction}
    - \underbrace{\frac18 g^2 f^{abc} f^{ade} A_\mu^b A_\nu^c A^\mu_d
    A^\nu_e}_\text{quartic gluon interaction}
\end{align*}

One can see in the \enquote{ghost--gluon} term that the ghosts decouple in an
abelian theory ($f^{abc} \equiv 0$) and also when the Lorentz gauge is chosen.

\subsection{Feynman rules for non-ghosts}

Here we will derive the QCD vertices. At first I thought \enquote{Didn't we
derive those already? They look so familiar!} but that was only in Theoretical
Particle/Hadron Physics and the derivation was rather quick. Here we will
reproduce it in all its \cancel{glory} gory.

\paragraph{Gluon--fermion vertex}

\begin{figure}
    \centering
    \includegraphics{one}
    \caption{%
        Single gluon vertex.
    }
    \label{fig:one}
\end{figure}

The interaction between the gluon and the fermion comes from the 
\(
    g \bar\psi \hat{\fourslash A} \psi
\)
term in the Lagrangian. The vertex is shown in Figure~\ref{fig:one} Since all
the terms are unique and there are no derivatives the Feynman rule is just
\(
    \iup g \mat \gamma^\mu \mat t^a
\)
as we have the $\gamma^\mu$ in the \enquote{slash} notation and the generator
$\mat t^a$ in the \enquote{hat} notation.

\paragraph{Three gluon vertex}

The quartic gluon interaction vertex is shown in Figure~\ref{fig:three}. The
naming of the incoming polarizations and colors (adj.) follows
\textcite[Figure~16.1]{Peskin/QFT/1995} to allow easy comparison of the result.

\begin{figure}
    \centering
    \includegraphics{three}
    \caption{%
        Triple gluon interaction vertex.
    }
    \label{fig:three}
\end{figure}

From the Lagrangian the term we need to use is
\[
    - g f^{abc} [\partial_\kappa A^a_\lambda] A^\kappa_b A^\lambda_c \,.
\]

The Feynman rule for the vertex is derived using functional derivatives with
respect to the fields. The number of indices is a notational hurdle in itself
made worse by upper and lower indices. The metric tensor becomes a
Kronecker-symbol when the indices are mixed. It can be done rigorously with
upper and lower indices, but that is not important here. Therefore we will just
move all the Lorentz indices down and color-adjoint indices up. Indices
occurring twice are still summed over by introduction of the appropriate metric
tensor. This will make it a bit harder to read at the expense of rigor.

Since we want to use the more common indices for the end result, we need to
rename the indices in that expression. Then we have
\[
    - g f^{def} \sbr{\partial^\kappa A_d^\lambda} A_\kappa^e A_\lambda^f \,.
\]
In order to get rid of the partial derivative we need to look at this in
momentum space.
\[
    - \iup g f^{def} k^\kappa A_d^\lambda A_\kappa^e A_\lambda^f
\]

Now we need to apply functional derivatives to this expression in order to
retrieve the Feynman rules. The derivatives and the momentum $k^\kappa$ in
there are a bit tricky. The way I see it that we are now doing the functional
differentiation in momentum space. This derivative will then give a
$\delta$-distribution with two momenta, one of them being the $k^\kappa$ that
is in the term. Performing the Fourier integral will then set this $k^\kappa$
to the momentum of the incoming particle that corresponds to that particular
functional derivative.

Another thing that we will use is the $\Xi_{\mu\nu}^{ab}$ symbol that is just
$g_{\mu\nu} \delta^{ab}$. If some indices are missing, it is just the
corresponding part. This will save a lot of fudging with indices as a
functional derivative will just replace $A$ with $\Xi$ and add the indices from
the functional derivative to it.

With those two things in place we can tackle the three gluon vertex
$V^{abc}_{\mu\nu\rho}$.
\begin{align*}
    V^{abc}_{\mu\nu\rho}
    &= g f^{def}
    \frac{\deltaup}{\deltaup A^a_\mu}
    \frac{\deltaup}{\deltaup A^b_\nu}
    \frac{\deltaup}{\deltaup A^c_\rho}
    \Big[
        k^\kappa A_{\lambda}^{d} A_{\kappa}^{e} A_{\lambda}^{f}
    \Big]
    \intertext{%
        We perform the first derivative.
    }
    &= g f^{def}
    \frac{\deltaup}{\deltaup A^a_\mu}
    \frac{\deltaup}{\deltaup A^b_\nu}
    \Big[
        k_3^\kappa \Xi_{\rho\lambda}^{cd} A_{\kappa}^{e} A_{\lambda}^{f}
        + k^\kappa A_{\lambda}^{d} \Xi_{\rho\kappa}^{ce} A_{\lambda}^{f}
        + k^\kappa A_{\lambda}^{d} A_{\kappa}^{e} \Xi_{\rho\lambda}^{cf}
    \Big]
    \intertext{%
        We simplify the Lorentz structure a bit.
    }
    &= g f^{def}
    \frac{\deltaup}{\deltaup A^a_\mu}
    \frac{\deltaup}{\deltaup A^b_\nu}
    \Big[
        k_3^\kappa \Xi^{cd} A_{\rho}^{f} A_{\kappa}^{e}
        + k^\kappa \Xi_{\rho\kappa}^{ce} A_{\lambda}^{d} A_{\lambda}^{f}
        + k^\kappa \Xi^{cf} A_{\rho}^{d} A_{\kappa}^{e}
    \Big]
    \intertext{%
        Then we apply the second derivative.
    }
    &= g f^{def}
    \frac{\deltaup}{\deltaup A^a_\mu}
    \Big[
        k_3^\kappa \Xi^{cd} \Xi_{\nu\rho}^{bf} A_{\kappa}^{e}
        + k_3^\kappa \Xi^{cd} A_{\rho}^{f} \Xi_{\nu\kappa}^{be}
        + k_2^\kappa \Xi_{\rho\kappa}^{ce} \Xi_{\nu\lambda}^{bd} A_{\lambda}^{f}
        + k_2^\kappa \Xi_{\rho\kappa}^{ce} A_{\lambda}^{d} \Xi_{\nu\lambda}^{bf}
        \\&\qquad
        + k^\kappa \Xi^{cf} \Xi_{\nu\rho}^{bd} A_{\kappa}^{e}
        + k^\kappa \Xi^{cf} A_{\rho}^{d} \Xi_{\nu\kappa}^{be}
    \Big]
    \intertext{%
        Again we can simplify a bit by applying the metric tensors in the $\Xi$
        symbol.
    }
    &= g f^{def}
    \frac{\deltaup}{\deltaup A^a_\mu}
    \Big[
        k_3^\kappa \Xi^{cd} \Xi_{\nu\rho}^{bf} A_{\kappa}^{e}
        + k_3^\kappa \Xi^{cd} A_{\rho}^{f} \Xi_{\nu\kappa}^{be}
        + k_2^\kappa \Xi_{\rho\kappa}^{ce} \Xi^{bd} A_{\nu}^{f}
        + k_2^\kappa \Xi_{\rho\kappa}^{ce} A_{\nu}^{d} \Xi^{bf}
        \\&\qquad
        + k^\kappa \Xi^{cf} \Xi_{\nu\rho}^{bd} A_{\kappa}^{e}
        + k^\kappa \Xi^{cf} A_{\rho}^{d} \Xi_{\nu\kappa}^{be}
    \Big]
    \intertext{%
        Then we apply the last derivative.
    }
    &= g f^{def}
    \Big[
        k_3^\kappa \Xi^{cd} \Xi_{\nu\rho}^{bf} \Xi_{\mu\kappa}^{ae}
        + k_3^\kappa \Xi^{cd} \Xi_{\mu\rho}^{af} \Xi_{\nu\kappa}^{be}
        + k_2^\kappa \Xi_{\rho\kappa}^{ce} \Xi^{bd} \Xi_{\mu\nu}^{af}
        + k_2^\kappa \Xi_{\rho\kappa}^{ce} \Xi_{\mu\nu}^{ad} \Xi^{bf}
        \\&\qquad
        + k_1^\kappa \Xi^{cf} \Xi_{\nu\rho}^{bd} \Xi_{\mu\kappa}^{ae}
        + k_1^\kappa \Xi^{cf} \Xi_{\mu\rho}^{ad} \Xi_{\nu\kappa}^{be}
    \Big]
    \intertext{%
        Now we simplify the Lorentz structure.
    }
    &= g f^{def}
    \Big[
        k_3^\mu \Xi^{cd} \Xi_{\nu\rho}^{bf} \Xi^{ae}
        + k_3^\nu \Xi^{cd} \Xi_{\mu\rho}^{af} \Xi^{be}
        + k_2^\rho \Xi^{ce} \Xi^{bd} \Xi_{\mu\nu}^{af}
        + k_2^\rho \Xi^{ce} \Xi_{\mu\nu}^{ad} \Xi^{bf}
        \\&\qquad
        + k_1^\mu \Xi^{cf} \Xi_{\nu\rho}^{bd} \Xi^{ae}
        + k_1^\nu \Xi^{cf} \Xi_{\mu\rho}^{ad} \Xi^{be}
    \Big]
    \intertext{%
        The color-adjoint indices can be simplified. The terms contain a
        permutation of $(abc)$ which is either even or odd. Those are: $(cba)$
        odd, $(cab)$ even, $(cba)$ odd, $(cab)$ even, $(cba)$ odd, $(cab)$
        even. Therefore we can just flip the signs and let all the $\Xi$ with
        color-adjoint indices go away.
    }
    &= g f^{def}
    \Big[
        k_3^\nu g_{\mu\rho}
        + k_1^\nu g_{\mu\rho}
        - k_2^\rho g_{\mu\nu}
        + k_2^\rho g_{\mu\nu}
        - k_3^\mu g_{\nu\rho}
        - k_1^\mu g_{\nu\rho}
    \Big]
\end{align*}
This is somewhat close to the desired result but wrong.

\paragraph{Four gluon vertex}

The quartic gluon interaction vertex is shown in Figure~\ref{fig:four}. The
naming of the incoming polarizations and colors (adj.) follows
\textcite[Figure~16.1]{Peskin/QFT/1995} to allow easy comparison of the result.

\begin{figure}
    \centering
    \includegraphics{four}
    \caption{%
        Quartic gluon interaction vertex.
    }
    \label{fig:four}
\end{figure}

The term in the Lagrangian responsible for this vertex is
\[
    - \frac14 g^2 f^{abc} f^{ade} A_\mu^b A_\nu^c A^\mu_d A^\nu_e \,.
\]
Since we want to use the more common indices for the end result, we need to
rename the indices in that expression. Then we have
\[
    - \frac14 g^2 f^{fgh} f^{fij} A_\kappa^g A_\lambda^h A^\kappa_i A^\lambda_j \,.
\]
Although the indices $i$ and $j$ look like color-fundamental indices they are
supposed to be color-adjoint indices. I'm sorry, there are just not enough
letters from \enquote a to \enquote i in the alphabet \Winkey.

So our vertex $V^{bcde}$ is
\begin{align*}
    V^{bcde}
    &= - \frac\iup4 g^2 f^{fgh} f^{fij}
    \frac{\deltaup}{\deltaup A^a_\mu}
    \frac{\deltaup}{\deltaup A^b_\nu}
    \frac{\deltaup}{\deltaup A^c_\rho}
    \frac{\deltaup}{\deltaup A^d_\sigma}
    A_\kappa^g A_\lambda^h A_\kappa^i A_\lambda^j \,.
    \intertext{%
        Naïve computation would give 24 terms directly, that is not really
        nice. Therefore one should simplify after every step. The first
        differentiation will give us
    }
    &= - \frac\iup4 g^2 f^{fgh} f^{fij}
    \frac{\deltaup}{\deltaup A^a_\mu}
    \frac{\deltaup}{\deltaup A^b_\nu}
    \frac{\deltaup}{\deltaup A^c_\rho}
    \Big[
        \Xi_{\sigma\kappa}^{dg} A_{\lambda}^{h} A_{\kappa}^{i} A_{\lambda}^{j}
        + A_{\kappa}^{g} \Xi_{\sigma\lambda}^{dh} A_{\kappa}^{i} A_{\lambda}^{j}
        + A_{\kappa}^{g} A_{\lambda}^{h} \Xi_{\sigma\kappa}^{di} A_{\lambda}^{j}
        + A_{\kappa}^{g} A_{\lambda}^{h} A_{\kappa}^{i} \Xi_{\sigma\lambda}^{dj}
    \Big]
    \intertext{%
        We reduce the number of Lorentz indices and move the $\Xi$ up front.
        Also we exchange the terms such that the order of the Lorentz indices
        is always the same.
    }
    &= - \frac\iup4 g^2 f^{fgh} f^{fij}
    \frac{\deltaup}{\deltaup A^a_\mu}
    \frac{\deltaup}{\deltaup A^b_\nu}
    \frac{\deltaup}{\deltaup A^c_\rho}
    \Big[
        \Xi^{dg} A_{\lambda}^{h} A_{\lambda}^{j}A_{\sigma}^{i} 
        + \Xi^{dh} A_{\lambda}^{g} A_{\lambda}^{i} A_{\sigma}^{j}
        + \Xi^{di} A_{\lambda}^{h} A_{\lambda}^{j} A_{\sigma}^{g}
        + \Xi^{dj} A_{\lambda}^{g} A_{\lambda}^{i} A_{\sigma}^{h}
    \Big]
    \intertext{%
        The structure constants are antisymmetric under the exchange $g
        \leftrightarrow h$ and $i \leftrightarrow j$. They are even when done
        both at the same time, of course. The two structure constants are even
        under the exchange $g, h \leftrightarrow i, j$ as this can be untangled
        by just commuting the structure constants. The first two summands are
        equal with the two pairwise exchanges done. In the second one we have
        $(ihjg)$ and $(jgih)$. We perform the two transforms as well and see
        that the third and fourth term are equal as well. Therefore the
        expression is reduced to
    }
    &= - \frac\iup2 g^2 f^{fgh} f^{fij}
    \frac{\deltaup}{\deltaup A^a_\mu}
    \frac{\deltaup}{\deltaup A^b_\nu}
    \frac{\deltaup}{\deltaup A^c_\rho}
    \Big[
        \Xi^{dg} A_{\lambda}^{h} A_{\lambda}^{j}A_{\sigma}^{i} 
        + \Xi^{dh} A_{\lambda}^{g} A_{\lambda}^{i} A_{\sigma}^{j}
    \Big] \,.
    \intertext{%
        As the telemarketers say: \enquote{But why stop here?}. The remaining two
        terms now have the indices $(ghji)$ and $(hgij)$. A quick exchange of
        $i$ with $j$ as well as $h$ with $g$ will give the first term.
        Therefore we are left with a single term only! The $\Xi$ can be removed
        by renaming the index in the structure constant directly. Our current
        expression is the eye in the storm and just has one summand:
    }
    &= - \iup g^2 f^{fdh} f^{fij}
    \frac{\deltaup}{\deltaup A^a_\mu}
    \frac{\deltaup}{\deltaup A^b_\nu}
    \frac{\deltaup}{\deltaup A^c_\rho}
    \Big[
        A_{\lambda}^{h} A_{\lambda}^{j} A_{\sigma}^{i} 
    \Big] \,.
    \intertext{%
        The next derivative will give us three terms again. Those are
    }
    &= - \iup g^2 f^{fdh} f^{fij}
    \frac{\deltaup}{\deltaup A^a_\mu}
    \frac{\deltaup}{\deltaup A^b_\nu}
    \Big[
        \Xi_{\rho\lambda}^{ch} A_{\lambda}^{j} A_{\sigma}^{i} 
        + A_{\lambda}^{h} \Xi_{\rho\lambda}^{cj} A_{\sigma}^{i} 
        + A_{\lambda}^{h} A_{\lambda}^{j} \Xi_{\rho\sigma}^{ci} 
    \Big] \,.
    \intertext{%
        Again we reduce the number of indices as far as we can. This leaves us
    }
    &= - \iup g^2 f^{fdh} f^{fij}
    \frac{\deltaup}{\deltaup A^a_\mu}
    \frac{\deltaup}{\deltaup A^b_\nu}
    \Big[
        \Xi^{ch} A_{\rho}^{j} A_{\sigma}^{i} 
        + \Xi^{cj} A_{\rho}^{h} A_{\sigma}^{i} 
        + \Xi_{\rho\sigma}^{ci}  A_{\lambda}^{h} A_{\lambda}^{j}
    \Big] \,.
    \intertext{%
        At this point we expand the structure constants and just rename the
        indices to get rid of the $\Xi$. The only remaining $\Xi$ with just
        Lorentz indices is transformed into a metric tensor (in Lorentz space).
    }
    &= - \iup g^2
    \frac{\deltaup}{\deltaup A^a_\mu}
    \frac{\deltaup}{\deltaup A^b_\nu}
    \Big[
        f^{fdc} f^{fij} A_{\rho}^{j} A_{\sigma}^{i} 
        + f^{fdh} f^{fic} A_{\rho}^{h} A_{\sigma}^{i} 
        + f^{fdh} f^{fcj} g_{\rho\sigma} A_{\lambda}^{h} A_{\lambda}^{j}
    \Big]
    \intertext{%
        Clearing of smoke means that more smoke has to be created. The next
        derivative gives us just two terms per summand.
    }
    &= - \iup g^2
    \frac{\deltaup}{\deltaup A^a_\mu}
    \Big[
        f^{fdc} f^{fij} \Xi_{\nu\rho}^{bj} A_{\sigma}^{i} 
        + f^{fdc} f^{fij} A_{\rho}^{j} \Xi_{\nu\sigma}^{bi} 
        + f^{fdh} f^{fic} \Xi_{\nu\rho}^{bh} A_{\sigma}^{i} 
        + f^{fdh} f^{fic} A_{\rho}^{h} \Xi_{\nu\sigma}^{bi} 
        \\&\qquad
        + f^{fdh} f^{fcj} g_{\rho\sigma} \Xi_{\nu\lambda}^{bh} A_{\lambda}^{j}
        + f^{fdh} f^{fcj} g_{\rho\sigma} A_{\lambda}^{h} \Xi_{\nu\lambda}^{bj}
    \Big]
    \intertext{%
        Simplifying is easier here. We rename the indices in the structure
        constants and introduce metric tensors.
    }
    &= - \iup g^2
    \frac{\deltaup}{\deltaup A^a_\mu}
    \Big[
        f^{fdc} f^{fib} g_{\nu\rho} A_{\sigma}^{i} 
        + f^{fdc} f^{fbj} g_{\nu\sigma} A_{\rho}^{j}
        + f^{fdb} f^{fic} g_{\nu\rho} A_{\sigma}^{i} 
        + f^{fdh} f^{fbc} g_{\nu\sigma} A_{\rho}^{h}
        \\&\qquad
        + f^{fdb} f^{fcj} g_{\rho\sigma} g_{\nu\lambda} A_{\lambda}^{j}
        + f^{fdh} f^{fcb} g_{\rho\sigma} g_{\nu\lambda} A_{\lambda}^{h}
    \Big]
    \intertext{%
        The last derivative will just replace every $A$ with $\Xi$.
    }
    &= - \iup g^2
    \Big[
        f^{fdc} f^{fib} g_{\nu\rho} \Xi_{\mu\sigma}^{ai} 
        + f^{fdc} f^{fbj} g_{\nu\sigma} \Xi_{\mu\rho}^{aj}
        + f^{fdb} f^{fic} g_{\nu\rho} \Xi_{\mu\sigma}^{ai} 
        + f^{fdh} f^{fbc} g_{\nu\sigma} \Xi_{\mu\rho}^{ah}
        \\&\qquad
        + f^{fdb} f^{fcj} g_{\rho\sigma} g_{\nu\lambda} \Xi_{\mu\lambda}^{aj}
        + f^{fdh} f^{fcb} g_{\rho\sigma} g_{\nu\lambda} \Xi_{\mu\lambda}^{ah}
    \Big]
    \intertext{%
        Then we simplify again.
    }
    &= - \iup g^2
    \Big[
        f^{fdc} f^{fab}  g_{\mu\sigma} g_{\nu\rho}
        + f^{fdc} f^{fba} g_{\mu\rho} g_{\nu\sigma}
        + f^{fdb} f^{fac} g_{\mu\sigma}  g_{\nu\rho}
        + f^{fda} f^{fbc} g_{\mu\rho} g_{\nu\sigma}
        \\&\qquad
        + f^{fdb} f^{fca} g_{\mu\nu} g_{\rho\sigma}
        + f^{fda} f^{fcb} g_{\mu\nu} g_{\rho\sigma}
    \Big]
    \intertext{%
        We sort the indices in the structure constants as far as possible and
        introduce some minus signs doing so.
    }
    &= - \iup g^2
    \Big[
        f^{fab} f^{fcd} g_{\mu\rho} g_{\nu\sigma}
        - f^{fab} f^{fcd} g_{\mu\sigma} g_{\nu\rho}
        + f^{fac} f^{fbd} g_{\mu\nu} g_{\rho\sigma}
        - f^{fac} f^{fbd} g_{\mu\sigma}  g_{\nu\rho}
        \\&\qquad
        + f^{fad} f^{fbc} g_{\mu\nu} g_{\rho\sigma}
        - f^{fad} f^{fbc} g_{\mu\rho} g_{\nu\sigma}
    \Big]
    \intertext{%
        We can then factor out the common terms and have the final result
    }
    &= - \iup g^2
    \Big[
        f^{fab} f^{fcd} [g_{\mu\rho} g_{\nu\sigma}
        - g_{\mu\sigma} g_{\nu\rho}]
        + f^{fac} f^{fbd} [g_{\mu\nu} g_{\rho\sigma}
        - g_{\mu\sigma}  g_{\nu\rho}]
        \\&\qquad
        + f^{fad} f^{fbc} [g_{\mu\nu} g_{\rho\sigma}
        - g_{\mu\rho} g_{\nu\sigma}]
    \Big] \,.
\end{align*}
This matches the result given by \textcite[Figure~16.1]{Peskin/QFT/1995} so
although the handling of the Lorentz indices was a little rough along the way
the result is correct.

\paragraph{Propagators}

The propagator for the quarks is just as with the leptons in QED. There is an
additional factor of $\delta^{ij}$ since propagation does not change the
color-fun.\ of a fermion.

Also the gluons behave like photons in the propagation, the propagator is just
the photon propagator with an additional $\delta^{ab}$ as the propagation does
not change the color-adj.\ of the gluon.

\end{document}

% vim: spell spelllang=en tw=79
