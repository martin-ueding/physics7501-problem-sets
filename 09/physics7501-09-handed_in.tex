\documentclass[11pt, english, fleqn, DIV=15, headinclude]{scrartcl}

\usepackage[bibatend]{../header}
\usepackage{../my-boxes}

\usepackage{lastpage}
\usepackage{multicol}
\usepackage{simplewick}
\usepackage{multicol}
\usepackage{slashed}
\usepackage{subcaption}
\usepackage{cancel}
\usepackage{tikzsymbols}

\newcommand\timeorder{\mathscr T}
\newcommand\normorder{\mathscr N}
\newcommand\eye{\mat 1_4}
\newcommand\fourslash[1]{\slashed{\four{#1}}}
\newcommand\T{\mathrm T}

\hypersetup{
    pdftitle=
}

\graphicspath{{build/}}

\newcounter{totalpoints}
\newcommand\punkte[1]{#1\addtocounter{totalpoints}{#1}}

\newcounter{problemset}
\setcounter{problemset}{9}

\subject{physics7501 -- Advanced Quantum Field Theory}
\ihead{physics7501 -- Problem Set \arabic{problemset}}

\title{Problem Set \arabic{problemset}}

\newcommand\thegroup{Tutor: Thorsten Schimannek}

\publishers{\thegroup}
\ofoot{\thegroup}

\author{
    Martin Ueding \\ \small{\href{mailto:mu@martin-ueding.de}{mu@martin-ueding.de}}
}
\ifoot{Martin Ueding}

\ohead{\rightmark}

\begin{document}

\maketitle

\vspace{3ex}

\begin{center}
    \begin{tabular}{rrr}
        Problem & Achieved points & Possible points \\
        \midrule
        \nameref{homework:1} & & \punkte{20} \\
        \midrule
        Total & & \arabic{totalpoints}
    \end{tabular}
\end{center}

\vspace{3ex}

\begin{center}
    \begin{small}
        This document consists of \pageref{LastPage} pages.
    \end{small}
\end{center}

\section{Fixing the non-Abelian gauge (with ghosts)}
\label{homework:1}

\subsection{Functional determinant and propagator}

\paragraph{Functional determinant}

\paragraph{Propagator}

\subsection{Feynman rules for ghosts}

\subsection{Yang-Mills Lagrangian}

The Yang-Mills Lagrangian coupling to quarks is given by
\[
    \mathcal L_\text{YM}
    = \underbrace{\bar\psi [\iup \slashed\Dif - m] \psi}_\text{coupled fermion}
    + \underbrace{\trcolorf\del{\hat F_{\mu\nu} \hat F^{\mu\nu}}}_\text{gluons}
    - \underbrace{\frac{1}{2\xi} \partial^\mu \hat A_\mu}_\text{gauge fixing}
    + \underbrace{\bar c^a \dalambert c^a}_\text{free ghost}
    + \underbrace{\bar c^a g f^{abc} \partial^\mu A_\mu^b c^c}_\text{ghost-gluon
    interaction} \,.
\]
This is a rather compact notation where complexity is hidden in the
gauge-covariant derivative and the gluon field-strength tensor. The trace is
meant to go over the fundamental indices as the gluon field-strength tensor
$\tens{\hat F}$ is a shorthand for $\tens F^a \mat t^a$ and the generators
$\mat t^a$ are given in the fundamental representation.

\subsection{Feynman rules for non-ghosts}

\paragraph{Three-gluon-vertex}

\paragraph{Four-gluon-vertex}

\end{document}

% vim: spell spelllang=en tw=79
