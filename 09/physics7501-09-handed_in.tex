\documentclass[11pt, english, fleqn, DIV=15, headinclude]{scrartcl}

\usepackage[bibatend]{../header}
\usepackage{../my-boxes}

\usepackage{lastpage}
\usepackage{multicol}
\usepackage{simplewick}
\usepackage{multicol}
\usepackage{slashed}
\usepackage{subcaption}
\usepackage{cancel}
\usepackage{tikzsymbols}

\newcommand\timeorder{\mathscr T}
\newcommand\normorder{\mathscr N}
\newcommand\eye{\mat 1_4}
\newcommand\fourslash[1]{\slashed{\four{#1}}}
\newcommand\T{\mathrm T}

\hypersetup{
    pdftitle=
}

\graphicspath{{build/}}

\newcounter{totalpoints}
\newcommand\punkte[1]{#1\addtocounter{totalpoints}{#1}}

\newcounter{problemset}
\setcounter{problemset}{9}

\subject{physics7501 -- Advanced Quantum Field Theory}
\ihead{physics7501 -- Problem Set \arabic{problemset}}

\title{Problem Set \arabic{problemset}}

\newcommand\thegroup{Tutor: Thorsten Schimannek}

\publishers{\thegroup}
\ofoot{\thegroup}

\author{
    Martin Ueding \\ \small{\href{mailto:mu@martin-ueding.de}{mu@martin-ueding.de}}
}
\ifoot{Martin Ueding}

\ohead{\rightmark}

\begin{document}

\maketitle

\vspace{3ex}

\begin{center}
    \begin{tabular}{rrr}
        Problem & Achieved points & Possible points \\
        \midrule
        \nameref{homework:1} & & \punkte{20} \\
        \midrule
        Total & & \arabic{totalpoints}
    \end{tabular}
\end{center}

\vspace{3ex}

\begin{center}
    \begin{small}
        This document consists of \pageref{LastPage} pages.
    \end{small}
\end{center}

\section{Fixing the non-Abelian gauge (with ghosts)}
\label{homework:1}

\subsection{Functional determinant and propagator}

\paragraph{Functional determinant}

\paragraph{Propagator}

\subsection{Feynman rules for ghosts}

\subsection{Yang-Mills Lagrangian}

The Yang-Mills Lagrangian coupling to quarks is given by
\[
    \mathcal L_\text{YM}
    = \underbrace{\bar\psi [\iup \slashed\Dif - m] \psi}_\text{coupled fermion}
    - \underbrace{\frac 14 \trcolorf\del{\hat F_{\mu\nu} \hat F^{\mu\nu}}}_\text{gluons}
    - \underbrace{\frac{1}{2\xi} \partial^\mu \hat A_\mu}_\text{gauge fixing}
    + \underbrace{\bar c^a \dalambert c^a}_\text{free ghost}
    + \underbrace{\bar c^a g f^{abc} \partial^\mu A_\mu^b c^c}_\text{ghost--gluon
    interaction} \,.
\]
This is a rather compact notation where complexity is hidden in the
gauge-covariant derivative and the gluon field-strength tensor. The trace is
meant to go over the fundamental indices as the gluon field-strength tensor
$\tens{\hat F}$ is a shorthand for $\tens F^a \mat t^a$ and the generators
$\mat t^a$ are given in the fundamental representation.

The gluon term must be expanded to give an expression to work with and derive
Feynman rules from.
\begin{align*}
    - \frac14 \trcolorf\del{\hat F_{\mu\nu} \hat F^{\mu\nu}}
    &= - \frac14 F_{\mu\nu}^a F^{\mu\nu}_b \trcolorf(t^a t^b) \\
    &= - \frac18 F_{\mu\nu}^a F^{\mu\nu}_b \delta^{ab} \\
    &= - \frac18 F_{\mu\nu}^a F^{\mu\nu}_a \\
    &= - \frac18 \sbr{\partial_\mu A_\nu^a - \partial_\nu A_\mu^a + g f^{abc} A_\mu^b
    A_\nu^c}^2 \\
    &= - \frac18 \sbr{\iup k_\mu A_\nu^a - \iup k_\nu A_\mu^a + g f^{abc} A_\mu^b
    A_\nu^c}^2 \\
    &= - \frac18 \bigg[
        - 2 k^2 (A^a)^2
        + 2 k^\mu k^\nu A^a_\mu A^a_\nu
        + 2 \iup g f^{abc} \sbr{k^\mu a^\nu_a - k^\nu A^\mu_a} A_\mu^b A_\nu^c
        \\
        &\qquad
        + g^2 f^{abc} f^{ade} A_\mu^b A_\nu^c A^\mu_d A^\nu_e
    \bigg] \\
    &= \frac14 k^2 (A^a)^2
    - \frac14 k^\mu k^\nu A^a_\mu A^a_\nu
    - \frac12 \iup g f^{abc} k^\mu a^\nu_a A_\mu^b A_\nu^c
    - \frac18 g^2 f^{abc} f^{ade} A_\mu^b A_\nu^c A^\mu_d A^\nu_e
\end{align*}

The expanded Lagrangian can be written as
\begin{align*}
    \mathcal L_\text{YM, Quarks} &= \mathcal L_\text{free} + \mathcal L_\text{int}
    \intertext{with}
    \mathcal L_\text{free}
    &=
    \underbrace{\bar\psi [\iup \slashed\partial - m] \psi}_\text{free fermion}
    + \underbrace{\frac14 A^a_\mu \sbr{\dalambert g^{\mu\nu} -
    \partial^\mu \partial^\nu} A^a_\nu}_\text{free gluon}
    + \underbrace{\bar c^a \dalambert c^a}_\text{free ghost}
    - \underbrace{\frac{1}{2\xi} \partial^\mu \hat A_\mu}_\text{gauge fixing}
    \\
    \mathcal L_\text{int}
    &=
    \underbrace{g \bar\psi \hat{\fourslash A} \psi}_\text{fermion--gluon}
    + \underbrace{\bar c^a g f^{abc} [\partial^\mu A_\mu^b]
    c^c}_\text{ghost--gluon}
    - \underbrace{\frac12 \iup g f^{abc} [\partial^\mu A^\nu_a] A_\mu^b
    A_\nu^c}_\text{triple gluon interaction}
    - \underbrace{\frac18 g^2 f^{abc} f^{ade} A_\mu^b A_\nu^c A^\mu_d
    A^\nu_e}_\text{quartic gluon interaction}
\end{align*}

One can see in the \enquote{ghost--gluon} term that the ghosts decouple in an
abelian theory ($f^{abc} \equiv 0$) and also when the Lorentz gauge is chosen.

\subsection{Feynman rules for non-ghosts}

Here we will derive the QCD vertices. At first I thought \enquote{Didn't we
derive those already? They look so familiar!} but that was only in Theoretical
Particle/Hadron Physics and the derivation was rather quick. Here we will
reproduce it in all its \cancel{glory} gory.

\paragraph{Three-gluon-vertex}

\paragraph{Four-gluon-vertex}

The quartic gluon interaction vertex is shown in Figure~\ref{fig:four}. The
naming of the incoming polarizations and colors (adj.) follows
\textcite[Figure~16.1]{Peskin/QFT/1995} to allow easy comparison of the result.

\begin{figure}
    \centering
    \includegraphics{four}
    \caption{%
        Quartic gluon interaction vertex.
    }
    \label{fig:four}
\end{figure}

The term in the Lagrangian responsible for this vertex is
\[
    - \frac14 g^2 f^{abc} f^{ade} A_\mu^b A_\nu^c A^\mu_d A^\nu_e \,.
\]
Since we want to use the more common indices for the end result, we need to
rename the indices in that expression. Then we have
\[
    - \frac14 g^2 f^{fgh} f^{fij} A_\kappa^g A_\lambda^h A^\kappa_i A^\lambda_j \,.
\]
Although the indices $i$ and $j$ look like color-fundamental indices they are
supposed to be color-adjoint indices. I'm sorry, there are just not enough
letters from \enquote a to \enquote i in the alphabet \Winkey. The Feynman
rule for the vertex is derived using functional derivatives with respect to the
fields. The number of indices is a notational hurdle in itself made worse by
upper and lower indices. The metric tensor becomes a Kronecker-symbol when the
indices are mixed. It can be done rigorously with upper and lower indices, but
that is not important here. Therefore we will just move all the Lorentz indices
down and color-adjoint indices up. Indices occurring twice are still summed
over by introduction of the appropriate metric tensor. This will make it a bit
harder to read at the expense of rigor. So our vertex $V^{bcde}$ is
\begin{align*}
    V^{bcde}
    &= - \frac\iup4 g^2 f^{fgh} f^{fij}
    \frac{\deltaup}{\deltaup A^a_\mu}
    \frac{\deltaup}{\deltaup A^b_\nu}
    \frac{\deltaup}{\deltaup A^c_\rho}
    \frac{\deltaup}{\deltaup A^d_\sigma}
    A_\kappa^g A_\lambda^h A_\kappa^i A_\lambda^j \,.
    \intertext{%
        Naïve computation would give 24 terms directly, that is not really
        nice. Therefore one should simplify after every step. The first
        differentiation will give us
    }
    &= - \frac\iup4 g^2 f^{fgh} f^{fij}
    \frac{\deltaup}{\deltaup A^a_\mu}
    \frac{\deltaup}{\deltaup A^b_\nu}
    \frac{\deltaup}{\deltaup A^c_\rho}
    \Big[
        \Xi_{\sigma\kappa}^{dg} A_{\lambda}^{h} A_{\kappa}^{i} A_{\lambda}^{j}
        + A_{\kappa}^{g} \Xi_{\sigma\lambda}^{dh} A_{\kappa}^{i} A_{\lambda}^{j}
        + A_{\kappa}^{g} A_{\lambda}^{h} \Xi_{\sigma\kappa}^{di} A_{\lambda}^{j}
        + A_{\kappa}^{g} A_{\lambda}^{h} A_{\kappa}^{i} \Xi_{\sigma\lambda}^{dj}
    \Big]
    \intertext{%
        We reduce the number of Lorentz indices and move the $\Xi$ up front.
        Also we exchange the terms such that the order of the Lorentz indices
        is always the same.
    }
    &= - \frac\iup4 g^2 f^{fgh} f^{fij}
    \frac{\deltaup}{\deltaup A^a_\mu}
    \frac{\deltaup}{\deltaup A^b_\nu}
    \frac{\deltaup}{\deltaup A^c_\rho}
    \Big[
        \Xi^{dg} A_{\lambda}^{h} A_{\lambda}^{j}A_{\sigma}^{i} 
        + \Xi^{dh} A_{\lambda}^{g} A_{\lambda}^{i} A_{\sigma}^{j}
        + \Xi^{di} A_{\lambda}^{h} A_{\lambda}^{j} A_{\sigma}^{g}
        + \Xi^{dj} A_{\lambda}^{g} A_{\lambda}^{i} A_{\sigma}^{h}
    \Big]
    \intertext{%
        The structure constants are antisymmetric under the exchange $g
        \leftrightarrow h$ and $i \leftrightarrow j$. They are even when done
        both at the same time, of course. The two structure constants are even
        under the exchange $g, h \leftrightarrow i, j$ as this can be untangled
        by just commuting the structure constants. The first two summands are
        equal with the two pairwise exchanges done. In the second one we have
        $(ihjg)$ and $(jgih)$. We perform the two transforms as well and see
        that the third and fourth term are equal as well. Therefore the
        expression is reduced to
    }
    &= - \frac\iup2 g^2 f^{fgh} f^{fij}
    \frac{\deltaup}{\deltaup A^a_\mu}
    \frac{\deltaup}{\deltaup A^b_\nu}
    \frac{\deltaup}{\deltaup A^c_\rho}
    \Big[
        \Xi^{dg} A_{\lambda}^{h} A_{\lambda}^{j}A_{\sigma}^{i} 
        + \Xi^{dh} A_{\lambda}^{g} A_{\lambda}^{i} A_{\sigma}^{j}
    \Big] \,.
    \intertext{%
        As the telemarketers say: \enquote{But why stop here?}. The remaining two
        terms now have the indices $(ghji)$ and $(hgij)$. A quick exchange of
        $i$ with $j$ as well as $h$ with $g$ will give the first term.
        Therefore we are left with a single term only! The $\Xi$ can be removed
        by renaming the index in the structure constant directly. Our current
        expression is the eye in the storm and just has one summand:
    }
    &= - \iup g^2 f^{fdh} f^{fij}
    \frac{\deltaup}{\deltaup A^a_\mu}
    \frac{\deltaup}{\deltaup A^b_\nu}
    \frac{\deltaup}{\deltaup A^c_\rho}
    \Big[
        A_{\lambda}^{h} A_{\lambda}^{j} A_{\sigma}^{i} 
    \Big] \,.
    \intertext{%
        The next derivative will give us three terms again. Those are
    }
    &= - \iup g^2 f^{fdh} f^{fij}
    \frac{\deltaup}{\deltaup A^a_\mu}
    \frac{\deltaup}{\deltaup A^b_\nu}
    \frac{\deltaup}{\deltaup A^c_\rho}
    \Big[
        \Xi_{\rho\lambda}^{ch} A_{\lambda}^{j} A_{\sigma}^{i} 
        + A_{\lambda}^{h} \Xi_{\rho\lambda}^{cj} A_{\sigma}^{i} 
        + A_{\lambda}^{h} A_{\lambda}^{j} \Xi_{\rho\sigma}^{ci} 
    \Big] \,.
    \intertext{%
        Again we reduce the number of indices as far as we can. This leaves us
    }
    &= - \iup g^2 f^{fdh} f^{fij}
    \frac{\deltaup}{\deltaup A^a_\mu}
    \frac{\deltaup}{\deltaup A^b_\nu}
    \frac{\deltaup}{\deltaup A^c_\rho}
    \Big[
        \Xi^{ch} A_{\rho}^{j} A_{\sigma}^{i} 
        + \Xi^{cj} A_{\rho}^{h} A_{\sigma}^{i} 
        + \Xi_{\rho\sigma}^{ci}  A_{\lambda}^{h} A_{\lambda}^{j}
    \Big] \,.
    \intertext{%
        At this point we expand the structure constants and just rename the
        indices to get rid of the $\Xi$. The only remaining $\Xi$ with just
        Lorentz indices is transformed into a metric tensor (in Lorentz space).
    }
    &= - \iup g^2
    \frac{\deltaup}{\deltaup A^a_\mu}
    \frac{\deltaup}{\deltaup A^b_\nu}
    \frac{\deltaup}{\deltaup A^c_\rho}
    \Big[
        f^{fdc} f^{fij} A_{\rho}^{j} A_{\sigma}^{i} 
        + f^{fdh} f^{fic} A_{\rho}^{h} A_{\sigma}^{i} 
        + f^{fdh} f^{fcj} g_{\rho\sigma} A_{\lambda}^{h} A_{\lambda}^{j}
    \Big] \,.
\end{align*}

\end{document}

% vim: spell spelllang=en tw=79
