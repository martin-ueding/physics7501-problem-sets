\documentclass[11pt, english, fleqn, DIV=15, headinclude]{scrartcl}

\usepackage[bibatend]{../header}
\usepackage{../my-boxes}

\usepackage{lastpage}
\usepackage{multicol}
\usepackage{simplewick}
\usepackage{multicol}
\usepackage{slashed}
\usepackage{subcaption}
\usepackage{cancel}
\usepackage{tikzsymbols}

\newcommand\timeorder{\mathscr T}
\newcommand\normorder{\mathscr N}
\newcommand\eye{\mat 1_4}
\newcommand\fourslash[1]{\slashed{\four{#1}}}
\newcommand\T{\mathrm T}

\hypersetup{
    pdftitle=
}

\graphicspath{{build/}}

\newcounter{totalpoints}
\newcommand\punkte[1]{#1\addtocounter{totalpoints}{#1}}

\newcounter{problemset}
\setcounter{problemset}{8}

\subject{physics7501 -- Advanced Quantum Field Theory}
\ihead{physics7501 -- Problem Set \arabic{problemset}}

\title{Problem Set \arabic{problemset}}

\newcommand\thegroup{Tutor: Thorsten Schimannek}

\publishers{\thegroup}
\ofoot{\thegroup}

\author{
    Martin Ueding \\ \small{\href{mailto:mu@martin-ueding.de}{mu@martin-ueding.de}}
}
\ifoot{Martin Ueding}

\ohead{\rightmark}

\begin{document}

\maketitle

\vspace{3ex}

\begin{center}
    \begin{tabular}{rrr}
        Problem & Achieved points & Possible points \\
        \midrule
        \nameref{homework:1} & & \punkte{20} \\
        \midrule
        Total & & \arabic{totalpoints}
    \end{tabular}
\end{center}

\vspace{3ex}

\begin{center}
    \begin{small}
        This document consists of \pageref{LastPage} pages.
    \end{small}
\end{center}

\section{(Non-)abelian gauge invariance}
\label{homework:1}

In the past weeks I have reached another level in understanding the whole
\enquote{Gauge Theory} business \Laughey. You don't have to read this, I just
wanted to share how my understanding progresses as a lot comes from discussions
with you.

\begin{multicols}{2}
    In general relativity one has a \emph{connection}\footnote{%
        In physics that is usually called \enquote{covariant derivative} from
        what I read. I like to stick with the mathematical names since I first
        read those names in the book by \textcite{penrose-road_to_reality}.
    } $\vnabla$ with \emph{connection coefficients}\footnote{%
        And in physics the \emph{connection coefficients} are called
        \enquote{connection}. Who makes this stuff up?
    } $\Gamma^\gamma_{\alpha\beta}$. The connection acting on a vector is then
    given as
    \[
        \nabla_\mu X_\nu
        = \partial_\mu X_\nu + \Gamma^\gamma_{\mu\nu} X_\gamma \,,
    \]
    or at least very close to that.
    Then the equation of motion is
    \[
        m_i \left[ \ddot x^\mu_i + \Gamma_{\alpha\beta}^\mu \dot x_i^\alpha
        \dot x_i^\beta \right] = q_i F^\mu{}_\nu \dot x_i^\nu
    \]
    and the Lagrangian density contains a term with the Ricci scalar, which is a
    functional of the Christoffel symbols.

    So the Christoffel symbols, the connection coefficients, were introduced in
    GR in order to have a sensible differentiation as the (co)tangent spaces at
    different points of the manifold are not comparable directly any more. One
    needs to introduce some sort of \emph{connection} (literally) to compare
    the fibers at different points of the underlying manifold. In GR this is
    the tangent bundle which is something I can depict in my imagination. One
    chooses this connection to be the Levi-Civita-connection such that metric
    and affine geodesics coincide. So that is just a complicated way to say
    that the metric tensor is invariant under the connection which is a handy
    thing to have. I think there were more arguments to this specific choice of
    connection but I cannot recall that.

    The connection coefficients then take on a life on their own and contribute
    to the Lagrangian density in form of the Ricci scalar $\mathcal R$. This
    means that the Christoffel symbols obtain some sort of tangible physical
    role and also appear in the equation of motion.

    Now in quantum electromagnetism we start off with $\mathcal L = \bar\psi
    [\iup \slashed\partial - m] \psi$ and note that we have a local symmetry in
    the spinors. The symmetry is a U(1) symmetry which means that our $\psi$
    now is a cross section through the fiber bundle $\R^4 \times \mathrm U(1)$.
    By the way, $\mathrm U(1) \simeq S^1$ and that fiber bundle now looks
    awfully close to the little I know about Kaluza–Klein theory. Is that a
    coincidence? $\psi(\four x)$ is now a cross section through the fiber
    bundle. Here the bundle is not as easily imaginable as the tangent bundle,
    so it took me a while to wrap my head around that. We are in need of a
    \emph{connection} between the different points in spacetime as each point
    now has a different gauge. Therefore it does not make sense to write
    ordinary derivatives any more. We want a partial derivative in every
    spacetime direction, so we need four connection coefficients. Those are
    called $A_\mu$ knowing what the \enquote{life of its own} will make those
    coefficients.

    Having the connection coefficients, it is simple to write down the
    connection. That is then the \emph{covariant derivative}. Just as in the GR
    case one needs to have a Lagrangian density which transforms with the
    trivial representation of the symmetry group, i.e.\ a scalar. Therefore one
    needs to add terms such that this turns out. In GR it was shown how this
    Lagrangian has to look like. Here we just have to build a Lagrangian
    without any ordinary derivative and match $\bar\psi$ with $\psi$ to give
    Gauge invariant terms. The Lagrangian density therefore is just
    \[
        \mathcal L = \bar\psi[\iup\slashed\nabla - m] \psi - \frac14 F \wedge
        \hodge F \,,
    \]
    where $\nabla$ is the connection and $F$ is a two-form. From here it is
    evident that it is gauge invariant, at least under an abelian gauge
    transformation. Each summand is invariant on its own and that is rather
    beautiful.

    I never thought that I would learn something about electromagnetism by
    studying general relativity. But apparently it served as a very good
    analogy.

\end{multicols}

Thank you for all the discussions so far! \Smiley

\subsection{Transformation behavior}

\paragraph{Field strength tensor}

The field strength tensor can be defined proportional to the curvature,
\begin{align*}
    F_{\mu\nu} &= \frac{\iup}{e} [\Dif_\mu, \Dif_\nu] \,.
    \intertext{%
        We can also write this as idempotent antisymmetrization and have
    }
    &= \frac{2\iup}{e} \Dif_{[\mu} \Dif_{\nu]} \,.
    \intertext{%
        Inserting all the terms and factoring out gives us
    }
    &= \frac{2\iup}{e} \sbr{
    \partial_{[\mu} \partial_{\nu]}
    + \iup e \partial_{[\mu} A_{\nu]}
    + \iup e A_{[\mu} \partial_{\nu]}
    - e^2 A_{[\mu} A_{\nu]}
    }
    \intertext{%
        The first term vanishes as partial derivatives of differentiable
        functions are symmetric. The last term vanishes also since the
        components of the potential are just real numbers and commute. This
        will of course change later with non-abelian gauge groups but we are
        not there yet. So the middle terms remain only. These are
    }
    &= 2 \sbr{
    \partial_{[\mu} A_{\nu]}
    + A_{[\mu} \partial_{\nu]}
    } \,.
    \intertext{%
        The partial derivative is supposed to act on another function as well.
        So we apply the product rule in the first term and obtain
    }
    &= 2 \sbr{
    \sbr{\partial_{[\mu} A_{\nu]}}
    + A_{[\nu} \partial_{\mu]}
    + A_{[\mu} \partial_{\nu]}
    } \,.
    \intertext{%
        The first term is scoped such that the partial derivative only acts on
        the connection coefficient. The last two terms then cancel if we switch
        the indices to give a minus sign. The remainder is the usual definition
        of the field strength tensor,
    }
    &= 2 \sbr{\partial_{[\mu} A_{\nu]}} \,, \\
    &= F_{\mu\nu} \,.
\end{align*}

\paragraph{Transformation of field strength tensor}

The commutator of the connections is
\begin{align*}
    [\Dif_\mu, \Dif_\nu]
    &= 2 \sbr{
    \partial_{[\mu} \partial_{\nu]}
    + \iup e \partial_{[\mu} A_{\nu]}
    + \iup e A_{[\mu} \partial_{\nu]}
    - e^2 A_{[\mu} A_{\nu]} \,.
    }
    \intertext{%
        We apply the mapping and obtain
    }
    &\mapsto 2 \sbr{
    \partial_{[\mu} \partial_{\nu]}
    + \iup e \partial_{[\mu} U A_{\nu]} U\inv
    + \iup e U A_{[\mu} U\inv \partial_{\nu]}
    - e^2 U A_{[\mu} U\inv U A_{\nu]} U\inv
    }
    \intertext{%
        We can collapse the $U\inv U$ in the last term. Also we can push the
        partial derivative through all the other terms until it is at the end.
    }
    &= 2 \bigg[
        \partial_{[\mu} \partial_{\nu]}
        + \iup e U_{[,\mu} A_{\nu]} U\inv
        + \iup e U \partial_{[\mu} A_{\nu]} U\inv
        + \iup e U A_{[\nu,\mu]} U\inv
        + \iup e U A_{[\nu} \partial U\inv_{,\mu]}
        + \iup e U A_{[\nu} U\inv \partial_{\mu]}
        \\&\quad
        + \iup e U A_{[\mu} U\inv \partial_{\nu]}
        - e^2 U A_{[\mu} A_{\nu]} U\inv
    \bigg]
    \intertext{%
        The terms with the trailing derivatives will again cancel like above.
    }
    &= 2 \bigg[
        \partial_{[\mu} \partial_{\nu]}
        + \iup e U_{[,\mu} A_{\nu]} U\inv
        + \iup e U \partial_{[\mu} A_{\nu]} U\inv
        + \iup e U A_{[\nu,\mu]} U\inv
        + \iup e U A_{[\nu} \partial U\inv_{,\mu]}
        \\&\quad
        - e^2 U A_{[\mu} A_{\nu]} U\inv
    \bigg]
    \intertext{%
        As a next step we use that the inverse exponential map just has a
        negative argument. Therefore the differentiation of the inverse
        exponential map will just give the negative compared to the normal map.
        Therefore the two terms with derivatives of the exponential map will
        just cancel.
    }
    &= 2 \sbr{
        \partial_{[\mu} \partial_{\nu]}
        + \iup e U \partial_{[\mu} A_{\nu]} U\inv
        + \iup e U A_{[\nu,\mu]} U\inv
        - e^2 U A_{[\mu} A_{\nu]} U\inv
    }
    \intertext{%
        Most terms are now sandwiched in transformations. The first term does
        not have those terms. Since partial derivatives commute, we can just
        add $U U\inv$ on either side and move one half through the commutator.
        Then we obtain
    }
    &= 2 U \sbr{
        \partial_{[\mu} \partial_{\nu]}
        + \iup e \partial_{[\mu} A_{\nu]}
        + \iup e A_{[\nu,\mu]}
        - e^2 A_{[\mu} A_{\nu]}
    } U\inv \,,
    \intertext{%
        which is the desired expression
    }
    &= U [\Dif_\mu, \Dif_\nu] U\inv \,.
\end{align*}

\paragraph{Invariance}

\end{document}

% vim: spell spelllang=en tw=79
