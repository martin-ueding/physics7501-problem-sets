\documentclass[11pt, english, fleqn, DIV=15, headinclude]{scrartcl}

\usepackage[bibatend]{../header}
\usepackage{../my-boxes}

\usepackage{lastpage}
\usepackage{multicol}
\usepackage{simplewick}
\usepackage{multicol}
\usepackage{slashed}
\usepackage{subcaption}
\usepackage{cancel}
\usepackage{tikzsymbols}

\newcommand\timeorder{\mathscr T}
\newcommand\normorder{\mathscr N}
\newcommand\eye{\mat 1_4}
\newcommand\fourslash[1]{\slashed{\four{#1}}}
\newcommand\T{\mathrm T}

\hypersetup{
    pdftitle=
}

\graphicspath{{build/}}

\newcounter{totalpoints}
\newcommand\punkte[1]{#1\addtocounter{totalpoints}{#1}}

\newcounter{problemset}
\setcounter{problemset}{8}

\subject{physics7501 -- Advanced Quantum Field Theory}
\ihead{physics7501 -- Problem Set \arabic{problemset}}

\title{Problem Set \arabic{problemset}}

\newcommand\thegroup{Tutor: Thorsten Schimannek}

\publishers{\thegroup}
\ofoot{\thegroup}

\author{
    Martin Ueding \\ \small{\href{mailto:mu@martin-ueding.de}{mu@martin-ueding.de}}
}
\ifoot{Martin Ueding}

\ohead{\rightmark}

\begin{document}

\maketitle

\vspace{3ex}

\begin{center}
    \begin{tabular}{rrr}
        Problem & Achieved points & Possible points \\
        \midrule
        \nameref{homework:1} & & \punkte{20} \\
        \midrule
        Total & & \arabic{totalpoints}
    \end{tabular}
\end{center}

\vspace{3ex}

\begin{center}
    \begin{small}
        This document consists of \pageref{LastPage} pages.
    \end{small}
\end{center}

\section{(Non-)abelian gauge invariance}
\label{homework:1}

In the past weeks I have reached another level in understanding the whole
\enquote{Gauge Theory} business \Laughey. You don't have to read this, I just
wanted to share how my understanding progresses as a lot comes from discussions
with you.

\begin{multicols}{2}
    In general relativity one has a \emph{connection}\footnote{%
        In physics that is usually called \enquote{covariant derivative} from
        what I read. I like to stick with the mathematical names since I first
        read those names in the book by \textcite{penrose-road_to_reality}.
    } $\vnabla$ with \emph{connection coefficients}\footnote{%
        And in physics the \emph{connection coefficients} are called
        \enquote{connection}. Who makes this stuff up?
    } $\Gamma^\gamma_{\alpha\beta}$. The connection acting on a vector is then
    given as
    \[
        \nabla_\mu X_\nu
        = \partial_\mu X_\nu + \Gamma^\gamma_{\mu\nu} X_\gamma \,,
    \]
    or at least very close to that.
    Then the equation of motion is
    \[
        m_i \left[ \ddot x^\mu_i + \Gamma_{\alpha\beta}^\mu \dot x_i^\alpha
        \dot x_i^\beta \right] = q_i F^\mu{}_\nu \dot x_i^\nu
    \]
    and the Lagrangian density contains a term with the Ricci scalar, which is a
    functional of the Christoffel symbols.

    So the Christoffel symbols, the connection coefficients, were introduced in
    GR in order to have a sensible differentiation as the (co)tangent spaces at
    different points of the manifold are not comparable directly any more. One
    needs to introduce some sort of \emph{connection} (literally) to compare
    the fibers at different points of the underlying manifold. In GR this is
    the tangent bundle which is something I can depict in my imagination. One
    chooses this connection to be the Levi-Civita-connection such that metric
    and affine geodesics coincide. So that is just a complicated way to say
    that the metric tensor is invariant under the connection which is a handy
    thing to have. I think there were more arguments to this specific choice of
    connection but I cannot recall that.

    The connection coefficients then take on a life on their own and contribute
    to the Lagrangian density in form of the Ricci scalar $\mathcal R$. This
    means that the Christoffel symbols obtain some sort of tangible physical
    role and also appear in the equation of motion.

    Now in quantum electromagnetism we start off with $\mathcal L = \bar\psi
    [\iup \slashed\partial - m] \psi$ and note that we have a local symmetry in
    the spinors. The symmetry is a U(1) symmetry which means that our $\psi$
    now is a cross section through the fiber bundle $\R^4 \times \mathrm U(1)$.
    By the way, $\mathrm U(1) \simeq S^1$ and that fiber bundle now looks
    awfully close to the little I know about Kaluza–Klein theory. Is that a
    coincidence? $\psi(\four x)$ is now a cross section through the fiber
    bundle. Here the bundle is not as easily imaginable as the tangent bundle,
    so it took me a while to wrap my head around that. We are in need of a
    \emph{connection} between the different points in spacetime as each point
    now has a different gauge. Therefore it does not make sense to write
    ordinary derivatives any more. We want a partial derivative in every
    spacetime direction, so we need four connection coefficients. Those are
    called $A_\mu$ knowing what the \enquote{life of its own} will make those
    coefficients.

    Having the connection coefficients, it is simple to write down the
    connection. That is then the \emph{covariant derivative}. Just as in the GR
    case one needs to have a Lagrangian density which transforms with the
    trivial representation of the symmetry group, i.e.\ a scalar. Therefore one
    needs to add terms such that this turns out. In GR it was shown how this
    Lagrangian has to look like. Here we just have to build a Lagrangian
    without any ordinary derivative and match $\bar\psi$ with $\psi$ to give
    Gauge invariant terms. The Lagrangian density therefore is just
    \[
        \mathcal L = \bar\psi[\iup\slashed\nabla - m] \psi - \frac14 F \wedge
        \hodge F \,,
    \]
    where $\nabla$ is the connection and $F$ is a two-form. From here it is
    evident that it is gauge invariant, at least under an abelian gauge
    transformation. Each summand is invariant on its own and that is rather
    beautiful.

    I never thought that I would learn something about electromagnetism by
    studying general relativity. But apparently it served as a very good
    analogy.

\end{multicols}

Thank you for all the discussions so far! \Smiley

\subsection{Transformation behavior}

\paragraph{Field strength tensor}

The field strength tensor can be defined proportional to the curvature,
\begin{align*}
    F_{\mu\nu} &= \frac{\iup}{e} [\Dif_\mu, \Dif_\nu] \,.
    \intertext{%
        We can also write this as idempotent antisymmetrization and have
    }
    &= \frac{2\iup}{e} \Dif_{[\mu} \Dif_{\nu]} \,.
    \intertext{%
        Inserting all the terms and factoring out gives us
    }
    &= \frac{2\iup}{e} \sbr{
    \partial_{[\mu} \partial_{\nu]}
    + \iup e \partial_{[\mu} A_{\nu]}
    + \iup e A_{[\mu} \partial_{\nu]}
    - e^2 A_{[\mu} A_{\nu]}
    }
    \intertext{%
        The first term vanishes as partial derivatives of differentiable
        functions are symmetric. The last term vanishes also since the
        components of the potential are just real numbers and commute. This
        will of course change later with non-abelian gauge groups but we are
        not there yet. So the middle terms remain only. These are
    }
    &= 2 \sbr{
    \partial_{[\mu} A_{\nu]}
    + A_{[\mu} \partial_{\nu]}
    } \,.
    \intertext{%
        The partial derivative is supposed to act on another function as well.
        So we apply the product rule in the first term and obtain
    }
    &= 2 \sbr{
    \sbr{\partial_{[\mu} A_{\nu]}}
    + A_{[\nu} \partial_{\mu]}
    + A_{[\mu} \partial_{\nu]}
    } \,.
    \intertext{%
        The first term is scoped such that the partial derivative only acts on
        the connection coefficient. The last two terms then cancel if we switch
        the indices to give a minus sign. The remainder is the usual definition
        of the field strength tensor,
    }
    &= 2 \sbr{\partial_{[\mu} A_{\nu]}} \,, \\
    &= F_{\mu\nu} \,.
\end{align*}

\paragraph{Transformation of field strength tensor}

A nice thing that we will use a lot of times is the following:
\[
    U U\inv = 1
    \implies
    \partial_\mu U U\inv = 
    U_{,\mu} U\inv + U U\inv_{,\mu}
    = 0 \,.
\]

The commutator of the connections is
\begin{align*}
    [\Dif_\mu, \Dif_\nu]
    &= 2 \sbr{
    \partial_{[\mu} \partial_{\nu]}
    + \iup e \partial_{[\mu} A_{\nu]}
    + \iup e A_{[\mu} \partial_{\nu]}
    - e^2 A_{[\mu} A_{\nu]} \,.
    }
    \intertext{%
        We apply the mapping and obtain
    }
    &\mapsto 2 \sbr{
    \partial_{[\mu} \partial_{\nu]}
    + \iup e \partial_{[\mu} U A_{\nu]} U\inv
    + \iup e U A_{[\mu} U\inv \partial_{\nu]}
    - e^2 U A_{[\mu} U\inv U A_{\nu]} U\inv
    }
    \intertext{%
        We can collapse the $U\inv U$ in the last term. Also we can push the
        partial derivative through all the other terms until it is at the end.
    }
    &= 2 \bigg[
        \partial_{[\mu} \partial_{\nu]}
        + \iup e U_{[,\mu} A_{\nu]} U\inv
        + \iup e U \partial_{[\mu} A_{\nu]} U\inv
        + \iup e U A_{[\nu,\mu]} U\inv
        + \iup e U A_{[\nu} \partial U\inv_{,\mu]}
        + \iup e U A_{[\nu} U\inv \partial_{\mu]}
        \\&\quad
        + \iup e U A_{[\mu} U\inv \partial_{\nu]}
        - e^2 U A_{[\mu} A_{\nu]} U\inv
    \bigg]
    \intertext{%
        The terms with the trailing derivatives will again cancel like above.
    }
    &= 2 \bigg[
        \partial_{[\mu} \partial_{\nu]}
        + \iup e U_{[,\mu} A_{\nu]} U\inv
        + \iup e U \partial_{[\mu} A_{\nu]} U\inv
        + \iup e U A_{[\nu,\mu]} U\inv
        + \iup e U A_{[\nu} \partial U\inv_{,\mu]}
        \\&\quad
        - e^2 U A_{[\mu} A_{\nu]} U\inv
    \bigg]
    \intertext{%
        As a next step we use the derived trick and the two terms with
        derivatives of the exponential map will just cancel.
    }
    &= 2 \sbr{
        \partial_{[\mu} \partial_{\nu]}
        + \iup e U \partial_{[\mu} A_{\nu]} U\inv
        + \iup e U A_{[\nu,\mu]} U\inv
        - e^2 U A_{[\mu} A_{\nu]} U\inv
    }
    \intertext{%
        Most terms are now sandwiched in transformations. The first term does
        not have those terms. Since partial derivatives commute, we can just
        add $U U\inv$ on either side and move one half through the commutator.
        Then we obtain
    }
    &= 2 U \sbr{
        \partial_{[\mu} \partial_{\nu]}
        + \iup e \partial_{[\mu} A_{\nu]}
        + \iup e A_{[\nu,\mu]}
        - e^2 A_{[\mu} A_{\nu]}
    } U\inv \,,
    \intertext{%
        which is the desired expression
    }
    &= U [\Dif_\mu, \Dif_\nu] U\inv \,.
\end{align*}

\paragraph{Invariance}

We want that the transformation and the connection commute with each other.
This will make a lot of things easier. So we start with the transformation of
$\psi$,
\begin{align*}
    \Dif_\mu \psi
    &\mapsto \Dif_\mu U \psi \,.
    \intertext{%
        Then we expand the covariant derivative.
    }
    &= [\partial_\mu + \iup e A_\mu] U \psi
    \intertext{%
        The product rule tells us to let the partial derivative act on $U$ and
        on $\psi$.
    }
    &= [U_{,\mu} + U \partial_\mu + \iup e A_\mu U] \psi
    \intertext{%
        The partial derivative of the transformation is given by the chain
        rule. For the potential we can add another $U U\inv$ pair and then
        factor out the $U$ in front.
    }
    &= U [\iup \alpha_{,\mu} + \partial_\mu + \iup e U\inv A_\mu U] \psi
    \intertext{%
        Shuffling the terms give a suggestive form. Then we can identity this
        as a backwards transformation.
    }
    &= U [\partial_\mu + \underbrace{\iup e U\inv A_\mu U + \iup
    \alpha_{,\mu}}_{\iup e A'_\mu}] \psi
    \intertext{%
        So if we also let $A$ transform in the way given in Equation~(2) on the
        problem set this all gives the form we want.
    }
    &= U [\partial_\mu + \iup e A_\mu] \psi
    \intertext{%
        This simplifies to
    }
    &= U \Dif_\mu \psi \,.
\end{align*}
So $U$ and $\Dif_\mu$ commute in the abelian case which is nice.

\subsection{Invariant Lagrangian}

The Lagrangian is given as
\begin{align*}
    \bar\psi [\iup \slashed\Dif - m] \psi
    &= \iup \bar\psi \slashed\Dif \psi - \bar\psi \psi \,.
    \intertext{%
        We apply the transformation to the $\psi$.
    }
    &\mapsto \iup \bar\psi U\inv \slashed\Dif' U \psi - \bar\psi U\inv U \psi
    \intertext{%
        The second summand is invariant under this transformation. The first
        summand would be invariant with the previous result if we would still
        deal with an abelian gauge. This is no longer the case, so we have to
        show the above result again in more generality.
    }
    &= \iup \bar\psi U\inv [\partial_\mu + \iup g A'_\mu] \mat\gamma^\mu U \psi
    - \bar\psi \psi
    \intertext{%
        The transformed potential has to be inserted. It is crucial to get the
        scope of the partial derivative in the transformation of the potential
        right. It only works on the immediately following inverse
        transformation. So all in all we now have
    }
    &= \iup \bar\psi U\inv [\partial_\mu + \iup g U A_\mu U \inv + U
    U\inv_{,\mu}] \mat\gamma^\mu U \psi
    - \bar\psi \psi \,.
    \intertext{%
        We move the first $U\inv$ and the last $U$ into the bracket and let it
        act on all the terms.
    }
    &= \iup \bar\psi [U \inv \partial_\mu U + \iup g A_\mu \inv +
    U\inv_{,\mu} U] \mat\gamma^\mu  \psi
    - \bar\psi \psi \,.
    \intertext{%
        The first summand will give another product rule, this gives two terms.
    }
    &= \iup \bar\psi [U \inv U_{,\mu} + \partial_\mu + \iup g A_\mu \inv +
    U\inv_{,\mu} U] \mat\gamma^\mu  \psi
    - \bar\psi \psi \,.
    \intertext{%
        The first and last summand just add to zero. Remaining is just a single
        $\Dif$ so that we have an invariant Lagrangian.
    }
    &= \iup \bar\psi \slashed\Dif \psi - \bar\psi \psi \,.
\end{align*}

\subsection{Non-abelian field strength tensor}

\paragraph{Definition via curvature}

We have shown that the abelian field strength tensor can also be written as
the curvature of the connection. Now we have to repeat the same thing for the
non-abelian case. It is rather irritating that both the letters $A$ and $F$ are
used for the abelian and non-abelian case. In some lectures they add hats for
the non-abelian quantities. I also have seen $G$ for the gluon field strength
tensor. Anyway, we have
\begin{align*}
    F_{\mu\nu}
    &= \frac \iup g [\Dif_\mu, \Dif_\nu] \,.
    \intertext{%
        We expand the covariant derivative.
    }
    &= \frac \iup g [\partial_\mu, \partial_\nu] + [\partial_\mu, A_\nu] +
    [A_\mu, \partial_\nu] - \frac \iup g [A_\mu, A_\nu]
    \intertext{%
        The first term vanishes as usual. The second and third term give $\dif
        A$. The last term has to written with generators in order to get some
        structure constants.
    }
    &= A_{\nu,\mu} - A_{\mu,\nu} - \frac \iup g A_\mu^b A_\nu^c [t_b, t_c]
    \intertext{%
        The commutator is by definition given by the structure constants.
    }
    &= A_{\nu,\mu} - A_{\mu,\nu} + \frac 1g A_\mu^b A_\nu^c f^{abc} t_a
    \intertext{%
        We can now project all this on $t_a$ and have
    }
    F_{\mu\nu}^a
    &= A_{\nu,\mu}^a - A_{\mu,\nu}^a + \frac 1g A_\mu^b A_\nu^c f^{abc} \,.
\end{align*}
The compact way to write this would be $F = \dif A + A \wedge A$.

\paragraph{Transformation of the curvature}

This is a tedious calculation. Lots of little places to screw up and to get
some signs wrong.

So we start with the curvature again
\begin{align*}
    [\Dif_\mu, \Dif_\nu]
    &= [\partial_\mu, \partial_\nu] - \iup g [\partial_\mu, A_\nu] -
    \iup g [A_\mu, \partial_\nu] - g^2 [A_\mu, A_\nu]
    \intertext{%
        Again the term vanishes, so we will just drop that already. There will
        be enough terms shortly.
    }
    &= \iup g [\partial_\mu, A_\nu] + \iup g [A_\mu, \partial_\nu] - g^2 [A_\mu, A_\nu]
    \intertext{%
        Now we can apply the transformation. We always incur extra terms from
        the transformation of the non-abelian $A$.
    }
    &\mapsto
    - \iup g \sbr{\partial_\mu, U A_\nu U\inv + \frac ig U U\inv_{,\nu}}
    - \iup g \sbr{U A_\mu U\inv + \frac ig U U\inv_{,\mu}, \partial_\nu}
    \\&\qquad
    - g^2 \sbr{U A_\mu U\inv + \frac ig U U\inv_{,\mu}, U A_\nu U\inv + \frac ig U
    U\inv_{,\nu}}
    \intertext{%
        Next we break up all the commutators.
    }
    &=
    - \iup g \sbr{\partial_\mu, U A_\nu U\inv}
    + \sbr{\partial_\mu, U U\inv_{,\nu}}
    - \iup g \sbr{U A_\mu U\inv, \partial_\nu}
    + \sbr{U U\inv_{,\mu}, \partial_\nu}
    - g^2 \sbr{U A_\mu U\inv, U A_\nu U\inv}
    \\&\qquad
    - \iup g \sbr{U A_\mu U\inv, U U\inv_{,\nu}}
    - \iup g \sbr{U U\inv_{,\mu}, U A_\nu U\inv}
    + \sbr{U U\inv_{,\mu}, U U\inv_{,\nu}}
    \intertext{%
        It makes sense to reorder them a bit.
    }
    &=
    - \iup g \sbr{\partial_\mu, U A_\nu U\inv}
    - \iup g \sbr{U A_\mu U\inv, \partial_\nu}
    - \iup g \sbr{U A_\mu U\inv, U U\inv_{,\nu}}
    - \iup g \sbr{U U\inv_{,\mu}, U A_\nu U\inv}
    \\&\qquad
    + \sbr{\partial_\mu, U U\inv_{,\nu}}
    + \sbr{U U\inv_{,\mu}, \partial_\nu}
    + \sbr{U U\inv_{,\mu}, U U\inv_{,\nu}}
    - g^2 \sbr{U A_\mu U\inv, U A_\nu U\inv}
    \intertext{%
        There are eight commutators in the above expression, four in each row.
        We will now transform every one of them a bit in order to keep this
        reasonable in amount of dead trees.
        \textcircled 1 and \textcircled 2:
        The first two commutators contain a partial derivative. This means that
        we have to apply the product rule and throw away the last term which
        acts on the trailing function (which is omitted here). The result is on
        the first line in the next equation block.
        \textcircled 3 and \textcircled 4:
        These do not contain any derivatives and just need to be expanded. The
        result is on the second line.
        \textcircled 5 and \textcircled 6:
        Again we have to push the derivative through each factor.
        \textcircled 7:
        We just write out the commutator explicitly.
        \textcircled 8:
        That one reduces nicely to a transformed commutator of the potential
        alone. The $U U\inv$ pairs in the middle just collapse.
    }
    &=
    - \iup g \sbr{U_{,\mu} A_\nu U\inv + U A_{\nu,\mu} U\inv + U A_\nu
    U\inv_{,\mu}}
    + \iup g \sbr{U_{,\nu} \, A_\mu U\inv + U A_{\mu,\nu} U\inv - U A_\mu
    U\inv_{,\nu}}
    \\&\qquad
    - \iup g \sbr{U A_\mu U\inv U U\inv_{,\nu} - U U\inv_{,\nu} U A_\mu U\inv}
    - \iup g \sbr{U U\inv_{,\mu} U A_\nu U\inv - U U\inv_{,\mu} U A_\nu U\inv}
    \\&\qquad
    + \sbr{U_{,\mu} U\inv_{,\nu} + U U\inv_{,\nu,\mu}}
    - \sbr{U_{,\nu} U\inv_{,\mu} + U U\inv_{,\mu,\nu}}
    + \sbr{U U\inv_{,\mu} U U\inv_{,\nu} - U U\inv_{,\nu} U U\inv_{,\mu}}
    \\&\qquad
    - g^2 U \sbr{A_\mu, A_\nu} U\inv
    \intertext{%
        There are two things to be done here. $U U\inv$ and $U\inv U$ pairs
        need to be collapsed. Also we need to apply the trick to shift the
        differentiated $U$ in order to do some collapsing.
    }
    &=
    - \iup g \sbr{U_{,\mu} A_\nu U\inv + U A_{\nu,\mu} U\inv + U A_\nu
    U\inv_{,\mu}}
    + \iup g \sbr{U_{,\nu} \, A_\mu U\inv + U A_{\mu,\nu} U\inv - U A_\mu
    U\inv_{,\nu}}
    \\&\qquad
    - \iup g \sbr{U A_\mu U\inv_{,\nu} + U_{,\nu} A_\mu U\inv}
    - \iup g \sbr{- U_{,\mu} A_\nu U\inv + U_{,\mu} A_\nu U\inv}
    \\&\qquad
    + \sbr{U_{,\mu} U\inv_{,\nu} + U U\inv_{,\nu,\mu}}
    - \sbr{U_{,\nu} U\inv_{,\mu} + U U\inv_{,\mu,\nu}}
    + \sbr{- U_{,\mu} U\inv_{,\nu} + U_{,\nu} U\inv_{,\mu}}
    \\&\qquad
    - g^2 U \sbr{A_\mu, A_\nu} U\inv
    \intertext{%
        We again shuffle the terms around. Well, some minor mistake happened
        here and I did not have to fix that. In my notes it worked out …
    }
    &=
    \iup g \sbr{
          U A_{\mu,\nu} U\inv
        - U A_{\nu,\mu} U\inv
        + U A_\mu U\inv_{,\nu}
        - U A_\mu U\inv_{,\nu}
        + U A_\nu U\inv_{,\mu}
        - U A_\nu U\inv_{,\mu}
    }
    \\&\qquad
    + \iup g \sbr{
          U_{,\mu} A_\nu U\inv
        - U_{,\mu} A_\nu U\inv
        + U_{,\nu} A_\mu U\inv
        - U_{,\nu} A_\mu U\inv
    }
    \\&\qquad
    + \sbr{
          U_{,\mu} U\inv_{,\nu}
        - U_{,\mu} U\inv_{,\nu}
        + U U\inv_{,\nu,\mu}
        - U U\inv_{,\mu,\nu}
        + U_{,\nu} U\inv_{,\mu}
        - U_{,\nu} U\inv_{,\mu}
    }
    \\&\qquad
    - g^2 U \sbr{A_\mu, A_\nu} U\inv
    \intertext{%
        We see that pretty much all terms cancel. The only remaining terms are
    }
    &=
    \iup g \sbr{
          U A_{\mu,\nu} U\inv
        - U A_{\nu,\mu} U\inv
    }
    - g^2 U \sbr{A_\mu, A_\nu} U\inv
    \intertext{%
        Those can be grouped with
    }
    &=
    U \sbr{\iup g \sbr{A_{\mu,\nu} -  A_{\nu,\mu}} - g^2 \sbr{A_\mu, A_\nu}} U \inv
\end{align*}
We can now restore the commutator with the partial derivatives with the same
reasoning used in a previous problem and have the transformed curvature as to
be shown.

\paragraph{Trace}

Last we have to show that the trace of two contracted field strength tensors is
gauge invariant. At this point in the course we have so many different types of
indices:

\begin{tabular}{lll}
    Type & Range & Letters \\
    \midrule
    Lorentz & \numrange 03 & $\mu,\nu,\ldots$ \\
    Dirac or spinor & \numrange 14 & $\alpha,\beta,\ldots$ \\
    Color & \numrange 13 & $i, j,\ldots$ \\
    Gluon & \numrange 18 & $a, b,\ldots$ \\
\end{tabular}

Of course it is only a gluon when we have $\SU(3)$ and we are here in $\SU(N)$.
It will go to $N^2 - 1$ in all cases. I heard that the gluon indices are also
called color indices. So we have color indices from \numrange 13 and \numrange
18? If that is true, who came up with that?

And I have a hunch that we will obtain another index for the flavor as we have
$\SU(N)_\text{flavor}$ and $\SU(3)_\text{color}$. Since the strong and
electromagnetic interaction are agnostic to flavor, we have not seen those.
Perhaps it comes up in the weak interaction which can change the flavor? Man, I
thought I've seen a lot of indices in general relativity \Sey.

Anyway, what I was going go get at is that “trace” is now pretty ambiguous. And
we do not even have a thermodynamic trace here in high energy physics. So I
propose to use
\[
    \trcolor(x)
    \eqnsep
    \trgluon(x)
    \eqnsep
    \trdirac(x)
    \qquad\text{and}\qquad
    \trlorentz(x)
\]
to make clear what is summed over. Perhaps I only have to do this in the
beginning just as adding hats to the quantum mechanical operators in the fourth
semester.

To cut to the chase, we have
\begin{align*}
    \frac 12 \trcolor(F^{\mu\nu} F_{\mu\nu})
    &= \frac 12 \trcolor(F^{\mu\nu}_a t^a  F_{\mu\nu}^b t_b) \,.
    \intertext{%
        The field strength tensors with all their indices explicit are now just
        scalars. Therefore we can pull them out of the trace.
    }
    &= \frac 12 F^{\mu\nu}_a  F_{\mu\nu}^b \trcolor(t^a t_b)
    \intertext{%
        The generators in the fundamental representation have one gluon and two
        color indices. Therefore a color trace is the perfect thing. In the
        adjoint representation they have three gluon indices and therefore
        would need a gluon trace. We were given the normalization so we can now
        write this as
    }
    &= - \frac 14 F^{\mu\nu}_a  F_{\mu\nu}^b \delta^a_b
    \intertext{%
        We can get rid of one of the indices and obtain the final result of
    }
    &= - \frac 14 F^{\mu\nu}_a  F_{\mu\nu}^1 \,.
\end{align*}

\end{document}

% vim: spell spelllang=en tw=79
