\documentclass[11pt, english, fleqn, DIV=15, headinclude]{scrartcl}

\usepackage[bibatend]{../header}
\usepackage{../my-boxes}

\usepackage{lastpage}
\usepackage{multicol}
\usepackage{simplewick}
\usepackage{multicol}
\usepackage{adjustbox}
\usepackage{slashed}
\usepackage{subcaption}
\usepackage{cancel}
\usepackage{tikzsymbols}
\usepackage{placeins}

\newcommand\timeorder{\mathscr T}
\newcommand\normorder{\mathscr N}
\newcommand\eye{\mat 1_4}
\newcommand\fourslash[1]{\slashed{\four{#1}}}
\newcommand\T{\mathrm T}

\hypersetup{
    pdftitle=
}

\graphicspath{{build/}}

\newcounter{totalpoints}
\newcommand\punkte[1]{#1\addtocounter{totalpoints}{#1}}

\newcounter{problemset}
\setcounter{problemset}{10}

\subject{physics7501 -- Advanced Quantum Field Theory}
\ihead{physics7501 -- Problem Set \arabic{problemset}}

\title{Problem Set \arabic{problemset}}

\newcommand\thegroup{Tutor: Thorsten Schimannek}

\publishers{\thegroup}
\ofoot{\thegroup}

\author{
    Martin Ueding \\ \small{\href{mailto:mu@martin-ueding.de}{mu@martin-ueding.de}}
}
\ifoot{Martin Ueding}

\ohead{\rightmark}

\begin{document}

\maketitle

\vspace{3ex}

\begin{center}
    \begin{tabular}{rrr}
        Problem & Achieved points & Possible points \\
        \midrule
        \nameref{homework:1} & & \punkte{25} \\
        \midrule
        Total & & \arabic{totalpoints}
    \end{tabular}
\end{center}

\vspace{3ex}

\begin{center}
    \begin{small}
        This document consists of \pageref{LastPage} pages.
    \end{small}
\end{center}

\section{Renormalization of $\phi^3$-theory in six dimensions}
\label{homework:1}

\newcommand\mdim M

\subsection{Dimensionality}

\paragraph{Four dimensions}

The action is dimensionless as it is fed into the path integral. There in the
exponential it must not have any unit. As the Lagrangian density is integrated
over the whole four dimensional space-time and space has negative energy
dimension, it must have energy dimension of four.

\paragraph{Six dimensions}

In our case here we are in \emph{six} dimensions. Therefore the Lagrangian
density must have mass dimension of six. We denote the mass/energy dimension
with the function $\mdim$. We now need $\mdim(\mathcal L) = d$ where $d = 6$ as
the integration over $\dif^6 x$ because $\mdim(\int \dif^6 x) = -6$.

Looking at the bilinear mass term and taking $\mdim(m) = 1$ for granted, we
must have $\mdim(\phi) = 2$. From that we conclude $\mdim(c) = 4$ and $\mdim(g)
= 0$. This is all combined in Table~\ref{tab:mass_dimensions}.

\paragraph{Real dimension}

Now we will go into $d = 6 - 2 \epsilon$ dimensions. The integration measure is
now of $-d$ mass dimensions and the Lagrangian density has to be of mass
dimension $d$. Starting again with the mass term we see that
\[
    \mdim(m^2 \phi^2) = \mdim(m^2) + \mdim(\phi^2) = 2 + 2 \mdim(\phi) \overset != d \,.
\]
From there we must have $\mdim(\phi) = 2 - \epsilon$. With that we can look at
$c$. We have
\[
    \mdim(c \phi) = \mdim(c) + \mdim(\phi) = \mdim(c) + 2 - \epsilon \overset != d \,.
\]
We conclude that $\mdim(c) = 4 - \epsilon$. Finally we can look at $g$. Here it
is
\[
    \mdim(g \phi^3) = \mdim(g) + 3 \mdim(\phi) = \mdim(g) + 6 - 3 \epsilon \overset != d \,.
\]
Solving for $\mdim(g)$ gives us $\mdim(g) = \epsilon$. We now have to introduce
the $\mu$ to absorb the mass dimension of $g$ and make $g$ a dimensionless
constant again. We assume $\mdim(\mu) = 1$ as $\mu$ has been used as a
regulating photon mass before. We need $\mdim(\mu^x g) = \epsilon$ such that
$\mdim(g)$ can stay zero. This brings us to
\[
    \mdim(\mu^x g) = x \mdim(\mu) + \mdim(g) = x + 0 \overset != \epsilon
\]
and we conclude that $x = \epsilon$.

All the mass dimensions for both cases are collected in
Table~\ref{tab:mass_dimensions} for easy review.

\begin{table}
    \centering
    \begin{tabular}{lSc}
        \toprule
        Term & {$\mdim$ for $d = 6$} & {$\mdim$ for $d = 6 - 2\epsilon$} \\
        \midrule
        $\int \dif^d x$ & -6 & $-6 + 2 \epsilon$ \\
        $\mathcal L$ & 6 & $6 - 2 \epsilon$ \\
        $\partial$ & 1 & 1 \\
        \midrule
        $m$ & 1 & 1 \\
        $\phi$ & 2 & $2 - \epsilon$ \\
        $c$ & 4 & $4 - \epsilon$ \\
        \midrule
        $g$ & 0 & 0 \\
        $\mu$ & 1 & 1 \\
        $\mu^x$ & 0 & $\epsilon$ \\
        \bottomrule
    \end{tabular}
    \caption{Mass dimensions}
    \label{tab:mass_dimensions}
\end{table}

\subsection{Feynman rules and superficial divergence}

\paragraph{Feynman rules}

There are two interactions in the Lagrangian. Those are
\begin{align*}
    - \frac{\mu^x g}{3!} \phi^3
    &\leadsto &
    \vcenter{\hbox{\includegraphics{triple}}}
    &= \iup \mu^x g \,,
    \\
    - c\phi
    &\leadsto &
    \vcenter{\hbox{\includegraphics{tadpole}}}
    &= \iup c \,.
    \\
\end{align*}
We mark the vertices with a dot in order to distinguish the tadpole diagram
from the simple propagator.

\paragraph{Degree of divergence}

Each propagator should give a mass dimension of $-2$ as there is a $q^2$ in the
denominator. At each loop we have to integrate over $d$ momenta and therefore
add $d$ mass dimensions there. Each $g$-vertex gives us $\mdim(g) = \epsilon$
and each $c$-vertex gives us $\mdim(c) = 4 - \epsilon$. Combining all this we
can give the mass dimension of a diagram. This directly is the superficial
degree of divergence,
\[
    D = - 2 N_\text P + d N_\text L + \epsilon N_g + [4 - \epsilon] N_c \,.
\]

This expression is not that helpful yet as we are supposed to express this
degree of divergence in terms of the number of propagators and vertices only.
Hence we need some expression to get rid of the number of loops.


\begin{table}
    \newcommand\tdiagram[1]{\adjustbox{valign=c}{\includegraphics[scale=0.78]{#1}}}
    \centering
    \begin{tabular}{cc*5{m{1em}}}
        \toprule
        Nickname
        & Diagram
        & {$N_\text e$}
        & {$N_\text L$}
        & {$N_\text P$}
        & {$N_g$}
        & {$N_c$} \\
        \midrule
        triple tadpole & \tdiagram{antennas} & 3 & 0 & 6 & 4 & 3 \\ \midrule[0.02em]
        one loop & \tdiagram{1-loop} & 2 & 1 & 2 & 2 & 0 \\ \midrule[0.02em]
        two loops & \tdiagram{2-loop} & 2 & 2 & 5 & 4 & 0 \\ \midrule[0.02em]
        double loop & \tdiagram{loop-creation2} & 2 & 2 & 5 & 4 & 0 \\ \midrule[0.02em]
        fork and loop & \tdiagram{fork-loop} & 3 & 1 & 3 & 3 & 0 \\ \midrule[0.02em]
        tree & \tdiagram{tree} & 4 & 0 & 1 & 2 & 0 \\ \midrule[0.02em]
        two antennas &\tdiagram{two-antenna} & 2 & 0 & 3 & 2 & 2 \\ \midrule[0.02em]
        three antennas & \tdiagram{three-antenna} & 2 & 0 & 5 & 3 & 3 \\ \midrule[0.02em]
        propagator & \tdiagram{propagator} & 2 & 0 & 1 & 0 & 0 \\
        \bottomrule
    \end{tabular}
    \caption{%
        Diagrams with their respective number of external lines $N_\text e$,
        loops $N_\text L$, propagators $N_\text P$, $g$-vertices $N_g$ and
        $c$-vertices $N_c$. External legs are labelled with a particle to
        clearly distinguish them from the $c$-vertices.
    }
    \label{tab:diagrams}
\end{table}


\begin{appendix}
    \section{Questions} 

\begin{question}
    In general relativity one writes $\partial$ for the normal partial
    derivative and $\nabla$ for the covariant derivative (or connection as it
    is called in mathematics). In gauge theory one also starts with $\partial$
    but calls the covariant derivative $\Dif$. Why is that? To me the $\Dif$
    looks like a more general form of $\dif$, the exterior derivative that acts
    on differential forms. In some way we have $\dif f \simeq \vec\partial f$.

    Is there a reason at all, is it just convention or is it to distinguish the
    gauge covariant connection from field theory from the Levi-Civita
    connection of general relativity?
\end{question}

\end{appendix}

\end{document}

% vim: spell spelllang=en tw=79
