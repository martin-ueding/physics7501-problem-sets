\documentclass[11pt, english, fleqn, DIV=15, headinclude]{scrartcl}

\usepackage[bibatend]{../header}
\usepackage{../my-boxes}

\usepackage{lastpage}
\usepackage{multicol}
\usepackage{simplewick}
\usepackage{multicol}
\usepackage{adjustbox}
\usepackage{slashed}
\usepackage{subcaption}
\usepackage{cancel}
\usepackage{tikzsymbols}
\usepackage{placeins}

\newcommand\timeorder{\mathscr T}
\newcommand\normorder{\mathscr N}
\newcommand\eye{\mat 1_4}
\newcommand\fourslash[1]{\slashed{\four{#1}}}
\newcommand\T{\mathrm T}

\hypersetup{
    pdftitle=
}

\graphicspath{{build/}}

\newcounter{totalpoints}
\newcommand\punkte[1]{#1\addtocounter{totalpoints}{#1}}

\newcounter{problemset}
\setcounter{problemset}{10}

\subject{physics7501 -- Advanced Quantum Field Theory}
\ihead{physics7501 -- Problem Set \arabic{problemset}}

\title{Problem Set \arabic{problemset}}

\newcommand\thegroup{Tutor: Thorsten Schimannek}

\publishers{\thegroup}
\ofoot{\thegroup}

\author{
    Martin Ueding \\ \small{\href{mailto:mu@martin-ueding.de}{mu@martin-ueding.de}}
}
\ifoot{Martin Ueding}

\ohead{\rightmark}

\begin{document}

\maketitle

\vspace{3ex}

\begin{center}
    \begin{tabular}{rrr}
        Problem & Achieved points & Possible points \\
        \midrule
        \nameref{homework:1} & & \punkte{25} \\
        \midrule
        Total & & \arabic{totalpoints}
    \end{tabular}
\end{center}

\vspace{3ex}

\begin{center}
    \begin{small}
        This document consists of \pageref{LastPage} pages.
    \end{small}
\end{center}

\section{Renormalization of $\phi^3$-theory in six dimensions}
\label{homework:1}

\newcommand\mdim M

\subsection{Dimensionality}

\paragraph{Four dimensions}

The action is dimensionless as it is fed into the path integral. There in the
exponential it must not have any unit. As the Lagrangian density is integrated
over the whole four dimensional space-time and space has negative energy
dimension, it must have energy dimension of four.

\paragraph{Six dimensions}

In our case here we are in \emph{six} dimensions. Therefore the Lagrangian
density must have mass dimension of six. We denote the mass/energy dimension
with the function $\mdim$. We now need $\mdim(\mathcal L) = d$ where $d = 6$ as
the integration over $\dif^6 x$ because $\mdim(\int \dif^6 x) = -6$.

Looking at the bilinear mass term and taking $\mdim(m) = 1$ for granted, we
must have $\mdim(\phi) = 2$. From that we conclude $\mdim(c) = 4$ and $\mdim(g)
= 0$. This is all combined in Table~\ref{tab:mass_dimensions}.

\paragraph{Real dimension}

Now we will go into $d = 6 - 2 \epsilon$ dimensions. The integration measure is
now of $-d$ mass dimensions and the Lagrangian density has to be of mass
dimension $d$. Starting again with the mass term we see that
\[
    \mdim(m^2 \phi^2) = \mdim(m^2) + \mdim(\phi^2) = 2 + 2 \mdim(\phi) \overset != d \,.
\]
From there we must have $\mdim(\phi) = 2 - \epsilon$. With that we can look at
$c$. We have
\[
    \mdim(c \phi) = \mdim(c) + \mdim(\phi) = \mdim(c) + 2 - \epsilon \overset != d \,.
\]
We conclude that $\mdim(c) = 4 - \epsilon$. Finally we can look at $g$. Here it
is
\[
    \mdim(g \phi^3) = \mdim(g) + 3 \mdim(\phi) = \mdim(g) + 6 - 3 \epsilon \overset != d \,.
\]
Solving for $\mdim(g)$ gives us $\mdim(g) = \epsilon$. We now have to introduce
the $\mu$ to absorb the mass dimension of $g$ and make $g$ a dimensionless
constant again. We assume $\mdim(\mu) = 1$ as $\mu$ has been used as a
regulating photon mass before. We need $\mdim(\mu^x g) = \epsilon$ such that
$\mdim(g)$ can stay zero. This brings us to
\[
    \mdim(\mu^x g) = x \mdim(\mu) + \mdim(g) = x + 0 \overset != \epsilon
\]
and we conclude that $x = \epsilon$.

All the mass dimensions for both cases are collected in
Table~\ref{tab:mass_dimensions} for easy review.

\begin{table}
    \centering
    \begin{tabular}{lSc}
        \toprule
        Term & {$\mdim$ for $d = 6$} & {$\mdim$ for $d = 6 - 2\epsilon$} \\
        \midrule
        $\int \dif^d x$ & -6 & $-6 + 2 \epsilon$ \\
        $\mathcal L$ & 6 & $6 - 2 \epsilon$ \\
        $\partial$ & 1 & 1 \\
        \midrule
        $m$ & 1 & 1 \\
        $\phi$ & 2 & $2 - \epsilon$ \\
        $c$ & 4 & $4 - \epsilon$ \\
        \midrule
        $g$ & 0 & 0 \\
        $\mu$ & 1 & 1 \\
        $\mu^x$ & 0 & $\epsilon$ \\
        \bottomrule
    \end{tabular}
    \caption{Mass dimensions}
    \label{tab:mass_dimensions}
\end{table}

\subsection{Feynman rules and superficial divergence}

\paragraph{Feynman rules}

There are two interactions in the Lagrangian. Those are
\begin{align*}
    - \frac{\mu^x g}{3!} \phi^3
    &\leadsto &
    \vcenter{\hbox{\includegraphics{triple}}}
    &= \iup \mu^x g \,,
    \\
    - c\phi
    &\leadsto &
    \vcenter{\hbox{\includegraphics{tadpole}}}
    &= \iup c \,.
    \\
\end{align*}
We mark the vertices with a dot in order to distinguish the tadpole diagram
from the simple propagator.

\paragraph{Degree of divergence}

Each propagator should give a mass dimension of $-2$ as there is a $q^2$ in the
denominator. At each loop we have to integrate over $d$ momenta and therefore
add $d$ mass dimensions there. Each $g$-vertex gives us $\mdim(g) = \epsilon$
and each $c$-vertex gives us $\mdim(c) = 4 - \epsilon$. Combining all this we
can give the mass dimension of a diagram. This directly is the superficial
degree of divergence,
\[
    D = - 2 N_\text P + d N_\text L + \epsilon N_g + [4 - \epsilon] N_c \,.
\]

This expression is not that helpful yet as we are supposed to express this
degree of divergence in terms of the number of propagators and vertices only.
Hence we need some expression to get rid of the number of loops.
Table~\ref{tab:diagrams} shows a couple of diagrams with their respective
number of external lines $N_\text e$, loops $N_\text L$, propagators $N_\text
P$, $g$-vertices $N_g$ and $c$-vertices $N_c$. We make the ansatz
\[
    a
    + \tilde N_\text e N_\text e
    + \tilde N_\text L N_\text L
    + \tilde N_\text P N_\text P
    + \tilde N_g N_g
    + \tilde N_c N_c
    = 0
\]
where the $\tilde N$ are the coefficients in this equation. They are to be
determined. The numbers in the table provide a system of equations. Including
all diagrams makes it solvable with a one-dimensional solution space. We choose
a normalization and therefore obtain the equation
\[
    2
    - N_\text e
    -2 N_\text L
    + N_g
    - N_c
    = 0 \,.
\]
This relation contains the number of loops, so we can express that number in
terms of the other quantities. Interestingly the number of propagators does not
appear in the relation. There is a problem that the system of equations only
has a one-dimensional solution space as we only extract one equation.

Using the one relation we can eliminate $N_\text L$ from the superficial degree
of divergence. This gets us to
\begin{align*}
    D
    &= - 2 N_\text P + \frac d2 \sbr{
        2 - N_\text e + N_g - N_c
    } + \epsilon N_g + [4 - \epsilon] N_c \,.
    \intertext{%
        We can simplify further by inserting $d = 6 - 2 \epsilon$.
    }
    &= - 2 N_\text P + [3 - \epsilon] \sbr{
        2 - N_\text e + N_g - N_c
    } + \epsilon N_g + [4 - \epsilon] N_c \,.
    \intertext{%
        And then factor out to get
    }
    &= 6 - 2 \epsilon - 2 N_\text P
    - [3 - \epsilon] N_\text e
    + 3 N_g
    + 1 N_c \,.
\end{align*}


\begin{table}[tbp]
    \newcommand\tdiagram[1]{\adjustbox{valign=c}{\includegraphics[scale=0.78]{#1}}}
    \centering
    \begin{tabular}{cc*5{m{1em}}}
        \toprule
        Nickname
        & Diagram
        & {$N_\text e$}
        & {$N_\text L$}
        & {$N_\text P$}
        & {$N_g$}
        & {$N_c$} \\
        \midrule
        triple tadpole & \tdiagram{antennas} & 3 & 0 & 6 & 4 & 3 \\ \midrule[0.02em]
        one loop & \tdiagram{1-loop} & 2 & 1 & 2 & 2 & 0 \\ \midrule[0.02em]
        two loops & \tdiagram{2-loop} & 2 & 2 & 5 & 4 & 0 \\ \midrule[0.02em]
        double loop & \tdiagram{loop-creation2} & 2 & 2 & 5 & 4 & 0 \\ \midrule[0.02em]
        fork and loop & \tdiagram{fork-loop} & 3 & 1 & 3 & 3 & 0 \\ \midrule[0.02em]
        tree & \tdiagram{tree} & 4 & 0 & 1 & 2 & 0 \\ \midrule[0.02em]
        two antennas &\tdiagram{two-antenna} & 2 & 0 & 3 & 2 & 2 \\ \midrule[0.02em]
        three antennas & \tdiagram{three-antenna} & 2 & 0 & 5 & 3 & 3 \\ \midrule[0.02em]
        propagator & \tdiagram{propagator} & 2 & 0 & 1 & 0 & 0 \\
        \bottomrule
    \end{tabular}
    \caption{%
        Diagrams with their respective number of external lines $N_\text e$,
        loops $N_\text L$, propagators $N_\text P$, $g$-vertices $N_g$ and
        $c$-vertices $N_c$. External legs are labelled to clearly distinguish
        them from the $c$-tadpoles.
    }
    \label{tab:diagrams}
\end{table}

\subsection{Superficially divergent diagrams}

In this part of the problem we are asked to draw all diagrams with are
one-particle-irreducible and only consist of one loop \emph{and} are
superficially divergent.

All diagrams containing a $c$-vertex are reducible. One can always cut off the
$c$-tadpole (\includegraphics{tadpole}) as the $c$-tadpole is a valid diagram
in itself. It is a particle disappearing into the vacuum which is quite a
strange thing.

In condensed matter theory such a diagram is the scattering with an impurity.
Due to momentum conservation no momentum can be exchanged and it only
contributes in $\four q = 0$. Applying a Fourier transform this gives the
monopole moment and therefore is the averaged impurity density. As far as I can
recall we took the tadpole as an irreducible diagram in order to obtain a
self-energy correction due to exactly this impurity scattering.

Figure~\ref{fig:loop} contains a couple of different irreducible diagrams with
exactly one loop. There is not so much one can do without the $c$-tadpoles.

\begin{figure}
    \centering
    \begin{subfigure}[c]{0.3\linewidth}
        \centering
        \includegraphics{loop-0}
        \caption{%
            No legs
        }
        \label{fig:loop/0}
    \end{subfigure}
    \hfill
    \begin{subfigure}[c]{0.3\linewidth}
        \centering
        \includegraphics{loop-1}
        \caption{%
            One leg
        }
        \label{fig:loop/1}
    \end{subfigure}
    \hfill
    \begin{subfigure}[c]{0.3\linewidth}
        \centering
        \includegraphics{loop-2}
        \caption{%
            Two legs
        }
        \label{fig:loop/2}
    \end{subfigure}
    \\
    \begin{subfigure}[c]{0.3\linewidth}
        \centering
        \includegraphics{loop-3}
        \caption{%
            Three legs
        }
        \label{fig:loop/3}
    \end{subfigure}
    \hspace{2em}
    \begin{subfigure}[c]{0.3\linewidth}
        \centering
        \includegraphics{loop-4}
        \caption{%
            Four legs
        }
        \label{fig:loop/4}
    \end{subfigure}
    \caption{%
        1-PI diagrams with only one loop.
    }
    \label{fig:loop}
\end{figure}

We can then look at the superficial degree of divergence using either form of
the formula derived above. One has to be a bit careful as both diagrams with no
and one external leg consist of one propagator only. Our results are summarized
in Table~\ref{tab:divergence}.

\begin{table}
    \centering
    \begin{tabular}{lSSS}
        \toprule
        Figure & {$N_\text e$} & {$D$} & {$S$} \\
        \midrule
        \ref{fig:loop/0} & 0 & 4  & 1 \\
        \ref{fig:loop/1} & 1 & 4  & 2 \\
        \ref{fig:loop/2} & 2 & 2  & 4 \\
        \ref{fig:loop/3} & 3 & 0  & 6 \\
        \ref{fig:loop/4} & 4 & -2 & 8 \\
        \bottomrule
    \end{tabular}
    \caption{%
        Superficial divergences of the diagrams shown in Figure~\ref{fig:loop}.
        The symmetry factor is also included.
    }
    \label{tab:divergence}
\end{table}

In the table we can see that only the first four diagrams are superficially
divergent. The more external lines we add, the less divergent (superficially)
the diagram becomes. At some point the diagrams are superficially convergent.

The diagram in Figure~\ref{fig:loop/0} without any external legs is a vacuum
bubble and will be cancled in any obersabled by the expansion of the
denominator. Therefore this diagram exists but it is canceled. We are not sure
whether this is to be taken acount. If it is not to be included, we are lacking
one superficially divergent diagram.

The next step is the symmetry factor for all the diagrams. Their symmetry
groups are the dihedral groups as they have the same symmetry as a regular
polygon. Each of these groups looks like this:
\[
    D_n = \bigcup_{i = 0}^{n-1} \set{c^i, bc^i}
    = \set{1, c, c^2, \ldots, c^{n-1}, b, bc, bc^2, \ldots, bc^{n-1}} \,,
\]
where $1$ is the identity, $c$ is the rotation around $2\piup/n$ and $b$ is the
reflection along some arbitrary but fixed axis. In each case the group has
cardinality $2n$. The symmetry group for the vacuum diagram seems to be
$\SO(2)$ from the diagram (if it hadn't been drawn like an ellipse). This will
probably lead to the wrong result. We have no contractions in this diagram so
the symmetry factor is probably just one.

\subsection{Taylor expansion}

\paragraph{No legs}

Without any legs, we have no external momentum. Therefore we just have
\[
    \iup \mathcal M
    = \iup \int \frac{\dif^d p}{[2 \piup]^d} \frac{1}{p^2 - m^2} \,.
\]
This is divergent by $D = d - 2$, which seems to be quartic for $\epsilon = 0$.

\paragraph{One leg}

One external leg still does not give us any additional momentum in the loop.
Only the loop momentum can go around in the loop, no momentum can be
transported into the $g$-tadpole. Therefore we just have
\[
    \iup \mathcal M
    = - \mu^x g \int \frac{\dif^d p}{[2 \piup]^d} \frac{1}{\sbr{p^2 - m^2}^2}
\]
which has a divergence of $D = d - 4$ which is quadratic.

\paragraph{Two legs}

The diagram with two legs has one loop momentum. Momentum can be transfered
through the loop. We choose the following convention for the momenta:

\hspace{\mathindent}\includegraphics{loop-2p}

We can write down the amplitude for this diagram:
\[
    \iup \mathcal M
    = (\mu^x g)^2 \int \frac{\dif^d p}{[2 \piup]^d}
    \frac{1}{p^2 - m^2}
    \frac{1}{[p + k]^2 - m^2}
\]
Now we can expand this around $\four k = 0$ in the next step.

Just to savor this moment: We have an integral which is logarithmically
divergent (for $\epsilon = 0$). A divergence already means that we are on the
very edge of validity for this theory. And to cure this divergence we now
expand the integrand into an infinite power series. Then we exchange the
summation and integration in order to get a series of integrals. This exchange
of limits should be shown carefully. That the integral diverges in the first
place should be a big fat hint that the exchange will be highly dubious. To
make it even worse, we claim that the integral does not really diverge for
large enough $\epsilon \in \R$.

At the next opportunity I will tell some mathematicians about this escalation
of things and see how they cringe. However, I assume that this can be treated
rigorously (just like dimensional regularization in polar coordinates) and is
not as bad as it really sounds.

To make the expansion more visible, we define the $k$-dependent term to be
$f(k)$. Then we can make the following tableau:
\begin{align*}
    f(k) &= \frac{1}{[p + k]^2 - m^2}
    &
    f(0) &= \frac{1}{p^2 - m^2}
    \\
    f'(k) &= - \frac{2 [p + k]}{\sbr{[p + k]^2 - m^2}^2}
    &
    f'(0) &= - \frac{2 p}{\sbr{p^2 - m^2}^2}
    \\
    f''(k) &=
    \frac{8[p + k]^2}{\sbr{[p + k]^2 - m^2}^3}
    - \frac{2}{\sbr{[p + k]^2 - m^2}^2}
    &
    f''(0) &=
    \frac{8p^2}{\sbr{p^2 - m^2}^3}
    - \frac{2}{\sbr{p^2 - m^2}^2}
\end{align*}

From this we can assemble the power series expansion. We have included the
degree of divergence~$D$ for each term. The first fraction outside of the
bracket is included in this as well. We obtain
\[
    \iup \mathcal M
    = (\mu^x g)^2 \int \frac{\dif^d p}{[2 \piup]^d}
    \frac{1}{p^2 - m^2}
    \bigg[
        \underbrace{\frac{1}{p^2 - m^2}}_{D = d -4}
        - \underbrace{\frac{\four p \cdot \four k}{\sbr{p^2 - m^2}^2}}_{D = d - 5}
        + \underbrace{\sbr{\frac{4p^2}{\sbr{p^2 - m^2}^3}- \frac{1}{\sbr{p^2 -
        m^2}^2}}}_{D = d - 6} k^2
        + \mathrm O(k^4)
    \bigg] \,.
\]
It is apparent that in each term the power of $p$ decreases. The terms odd in
$\four p$ drop out due to the symmetric integration bounds. This means that we
only have dependence on $k^2$, the Lorentz-invariant magnitude, as before. The
first term is badly divergent for $d = 6$. The last two terms are just
logarithmically divergent. All higher terms (omitted in the above expression)
are convergent in the $p$-integration. There seem to be two
divergent terms here, one quadratic and one logarithmic term.

Together with the two divergent terms from the previous two diagrams, we
already have four divergent terms now.

\paragraph{Three legs}

The diagram with three external legs still has only one loop momentum but now
two external momenta that enter the loop:

\hspace{\mathindent}\includegraphics{loop-3p}

The amplitude is just like in the case with two external legs, there is just an
additional propagator and vertex in the amplitude. It is
\[
    \iup \mathcal M
    = (\mu^x g)^3 \int \frac{\dif^d p}{[2 \piup]^d}
    \frac{1}{p^2 - m^2}
    \frac{1}{[p + k_1]^2 - m^2}
    \frac{1}{[p + k_1 + k_2]^2 - m^2} \,.
\]
Luckily we already have a divergence of $D = d - 6$ in the first term without
any expansion. This means that the only diverging term is the one we already
have. We can still expand in $\four k_1$ and $\four k_2$ and obtain the
following result:
\[
    \iup \mathcal M
    = (\mu^x g)^3 \int \frac{\dif^d p}{[2 \piup]^d}
    \frac{1}{p^2 - m^2}
    \sbr{
        \frac{1}{p^2 - m^2}
        \frac{1}{p^2 - m^2}
        + \mathrm O\sbr{k_1^2 + k_2^2}
    } \,.
\]
The terms linear in $\four k_1$ and $\four k_2$ drop out since those terms are
also odd in $\four p$. Due to the symmetric integration bounds this will drop
out.

This is our fifth divergent term. We are still missing one term. Perhaps the
term odd in $\four p$ of the diagram with two legs is to be included in the
counting to six?

In order to get down to four we need to drop one or two terms depending on the
counting. Which one shall it be? The formula for the dimensional regularization
on the problem set does not include the $p^2$ dependency in the numerator at
all. This means that this term is not included or that we have to find that
formula ourselves?

\subsection{Dimensional regularization}

\subsection{Counterterms}

\subsection{Linear term}

\begin{appendix}
    \section{Questions} 

\begin{question}
    In general relativity one writes $\partial$ for the normal partial
    derivative and $\nabla$ for the covariant derivative (or connection as it
    is called in mathematics). In gauge theory one also starts with $\partial$
    but calls the covariant derivative $\Dif$. Why is that? To me the $\Dif$
    looks like a more general form of $\dif$, the exterior derivative that acts
    on differential forms. In some way we have $\dif f \simeq \vec\partial f$.

    Is there a reason at all, is it just convention or is it to distinguish the
    gauge covariant connection from field theory from the Levi-Civita
    connection of general relativity?
\end{question}

\end{appendix}

\end{document}

% vim: spell spelllang=en tw=79
